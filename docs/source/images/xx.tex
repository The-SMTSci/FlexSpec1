\documentclass[crop,border=5,tikz,convert={outext=.svg,command=\unexpanded{pdf2svg
    \infile\space\outfile}},multi=false]{standalone}
\usetikzlibrary{math,calc,through,shapes,patterns}
\usepackage{graphicx}
\usepackage{subcaption}

\newcommand{\markangle}[6][0.25]{
  \begin{scope}
    \clip (#2)--(#3)--(#4);
    \node (ANG) [draw, circle, minimum size=#5] at (#3){};
  \end{scope}
  \coordinate (E1) at (intersection 0 of ANG and #3--#2);
  \coordinate (E2) at (intersection 0 of ANG and #3--#4);
  %\draw[thick] (E1) to [bend right=50] (E2);
  \coordinate (EM) at ($ (E1)!0.5!(E2) $);
  \coordinate (EM) at ($ (EM)!#1!(#3) $);
  \node at (EM) {#6};
}


 
\begin{document}


\begin{figure}[!h]
\centering
\begin{subfigure}[b]{0.45\textwidth}
\def\myta{60}
\begin{tikzpicture}[>=latex,scale=1]
  \fill[white] (-3,0) rectangle (3,3);
  \fill[gray!70] (-3,-1) rectangle (3,0);
%  \fill[pattern = north east lines] (-4,-0.5) rectangle (4,0);
  \draw[thick, dashed] (0,-0.85) -- (0,2.5);
  \coordinate (A) at (180-\myta:3);
  \coordinate (B) at (\myta:3);
  \coordinate (O) at (0,0);
  \coordinate (L) at (0,3);
  \draw[blue, ultra thick] (A) -- (O) --  (B);
  \draw[blue, ultra thick,->] (A) -- ($ (A)!0.5!(O) $);
  \draw[blue, ultra thick,->] (O) -- ($ (O)!0.6!(B) $);
  \markangle{A}{O}{L}{122}{$\alpha$}
  \markangle{L}{O}{B}{125}{$\alpha'$}
  \markangle{0,-1}{O}{3,0}{35}{$\cdot$}
  \node[fill=white,draw=red,ultra thick,rectangle,inner sep=5] at
  (-1.9,0.7){\Large $\alpha = \alpha'$};
\end{tikzpicture}

\caption{Total Reflection}
\end{subfigure}
~
\begin{subfigure}[b]{0.45\textwidth}
\input{refractionlaw2.tikz}
\caption{TODO}
\end{subfigure}

\begin{subfigure}[b]{0.45\textwidth}
\def\myalphac{30}
\def\mybetac{30}
\def\mync{1.8}

\tikzmath{\mybetac = asin(sin(\myalphac)*\mync);}

\begin{tikzpicture}[>=latex,scale=1]
  \fill[white] (-3,0) rectangle (3,3);
  \fill[gray!70] (-3,-3) rectangle (3,0);
%  \fill[pattern = north east lines] (-4,-0.5) rectangle (4,0);
  \draw[thick, dashed] (0,-2.5) -- (0,2.5);
  \coordinate (A) at (-90-\myalphac:3);
  \coordinate (B) at (-90+\myalphac:3);
  \coordinate (C) at (90-\mybetac:2.8);
  \coordinate (O) at (0,0);
  \coordinate (L) at (0,-3);
  \draw[blue, ultra thick] (A) -- (O) --  (C);
  \draw[blue, ultra thick,->] (A) -- ($ (A)!0.5!(O) $);
  \draw[blue, ultra thick,->] (O) -- ($ (O)!0.5!(C) $);
  \draw[blue] (O) -- (B);
  \draw[blue, ->] (O) -- ($ (O)!0.6!(B) $);
  \markangle{A}{O}{L}{126}{$\alpha$}
  \markangle{L}{O}{B}{126}{$\alpha$}
  \markangle{0,3}{O}{C}{126}{$\beta$}
  \markangle{0,1}{O}{-3,0}{30}{$\cdot$}
  \node () at (-2,0.5){$n_2$};
  \node () at (2,-1.5){\footnotesize $n_1 > n_2$};
  \node () at (-2,-0.5){$n_1$};
  \node[fill=white,draw=red,ultra thick,rectangle,inner sep=5] at
  (-1.6,1.7){\Large $\frac{\sin\alpha}{\sin\beta} = \frac{n_2}{n_1}$};
\end{tikzpicture}

\caption{TODO}
\end{subfigure}
~
\begin{subfigure}[b]{0.45\textwidth}
\def\myt{60}
\def\myalphad{50}
\def\mybetad{90}
\def\mync{1.8}
\tikzmath{\myalphad = asin(1/\mync);}
\begin{tikzpicture}[>=latex,scale=1]
  \fill[white] (-3,0) rectangle (3,3);
  \fill[gray!70] (-3,-3) rectangle (3,0);
  % \fill[pattern = north east lines] (-4,-0.5) rectangle (4,0);
  \draw[thick, dashed] (0,-2.5) -- (0,2.5);
  \coordinate (A) at (-90-\myalphad:3);
  \coordinate (B) at (-90+\myalphad:3);
  \coordinate (C) at (90-\mybetad:3);
  \coordinate (O) at (0,0);
  \coordinate (L) at (0,-3);
  \draw[blue, ultra thick] (A) -- (O);
  \draw[blue, ultra thick, dotted] (O) -- (C);
  \draw[blue, ultra thick,->] (A) -- ($ (A)!0.5!(O) $);
  \draw[blue, ultra thick,->] (O) -- ($ (O)!0.6!(B) $);
  \draw[blue, ultra thick] (O) -- (B);
  \markangle{A}{O}{L}{126}{$\alpha_{\rm t}$}
  \markangle{L}{O}{B}{126}{$\alpha_{\rm t}$}
  \markangle{0,1}{O}{3,0}{30}{$\cdot$}
  \node () at (-2,0.5){$n_2$};
  \node () at (2,-1.5){\footnotesize $n_1 > n_2$};
  \node () at (-2,-0.5){$n_1$};
  \node[fill=white,draw=red,ultra thick,rectangle,inner sep=5] at
  (-1.6,1.7){\large $\sin\alpha_{\rm t} > \frac{n_2}{n_1}$};
\end{tikzpicture}

\caption{TODO}
\end{subfigure}

\caption{Cases of light through Ovio Slit.}
\end{figure}

\end{document}
