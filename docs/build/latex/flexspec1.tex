%% Generated by Sphinx.
\def\sphinxdocclass{report}
\documentclass[letterpaper,10pt,english,openany,oneside]{sphinxmanual}
\ifdefined\pdfpxdimen
   \let\sphinxpxdimen\pdfpxdimen\else\newdimen\sphinxpxdimen
\fi \sphinxpxdimen=.75bp\relax

\PassOptionsToPackage{warn}{textcomp}
\usepackage[utf8]{inputenc}
\ifdefined\DeclareUnicodeCharacter
% support both utf8 and utf8x syntaxes
  \ifdefined\DeclareUnicodeCharacterAsOptional
    \def\sphinxDUC#1{\DeclareUnicodeCharacter{"#1}}
  \else
    \let\sphinxDUC\DeclareUnicodeCharacter
  \fi
  \sphinxDUC{00A0}{\nobreakspace}
  \sphinxDUC{2500}{\sphinxunichar{2500}}
  \sphinxDUC{2502}{\sphinxunichar{2502}}
  \sphinxDUC{2514}{\sphinxunichar{2514}}
  \sphinxDUC{251C}{\sphinxunichar{251C}}
  \sphinxDUC{2572}{\textbackslash}
\fi
\usepackage{cmap}
\usepackage[T1]{fontenc}
\usepackage{amsmath,amssymb,amstext}
\usepackage{babel}



\usepackage{times}
\expandafter\ifx\csname T@LGR\endcsname\relax
\else
% LGR was declared as font encoding
  \substitutefont{LGR}{\rmdefault}{cmr}
  \substitutefont{LGR}{\sfdefault}{cmss}
  \substitutefont{LGR}{\ttdefault}{cmtt}
\fi
\expandafter\ifx\csname T@X2\endcsname\relax
  \expandafter\ifx\csname T@T2A\endcsname\relax
  \else
  % T2A was declared as font encoding
    \substitutefont{T2A}{\rmdefault}{cmr}
    \substitutefont{T2A}{\sfdefault}{cmss}
    \substitutefont{T2A}{\ttdefault}{cmtt}
  \fi
\else
% X2 was declared as font encoding
  \substitutefont{X2}{\rmdefault}{cmr}
  \substitutefont{X2}{\sfdefault}{cmss}
  \substitutefont{X2}{\ttdefault}{cmtt}
\fi


\usepackage[Bjarne]{fncychap}
\usepackage{sphinx}

\fvset{fontsize=\small}
\usepackage{geometry}


% Include hyperref last.
\usepackage{hyperref}
% Fix anchor placement for figures with captions.
\usepackage{hypcap}% it must be loaded after hyperref.
% Set up styles of URL: it should be placed after hyperref.
\urlstyle{same}

\addto\captionsenglish{\renewcommand{\contentsname}{Contents:}}

\usepackage{sphinxmessages}
\setcounter{tocdepth}{2}



\title{FlexSpec 1}
\date{Apr 21, 2022}
\release{0.1}
\author{Jerrold L. Foote\and Wayne Green\and Greg Jones\and Frank Parks\and Anthony Rodda\and Forrest Sims\and Thomas C. Smith\and Clarke Yeager}
\newcommand{\sphinxlogo}{\vbox{}}
\renewcommand{\releasename}{Release}
\makeindex
\begin{document}

\pagestyle{empty}
\sphinxmaketitle
\pagestyle{plain}
\sphinxtableofcontents
\pagestyle{normal}
\phantomsection\label{\detokenize{index::doc}}


\begin{figure}[htbp]
\centering
\capstart

\noindent\sphinxincludegraphics{{FS1_Assembly_with_Baffle}.jpeg}
\caption{CAD Layout for optical components of the FlexSpec 1.}\label{\detokenize{index:id1}}\end{figure}

\sphinxAtStartPar
Ramifications of 3D printing leading to FlexSpec 1 (FS1):
Introducing the Flexible Spectrograph I
Small Telescope Spectrograph.

\sphinxAtStartPar
This is documentation, the project and our aspirations for developing
a science data quality 3D spectrograph that is ‘Flexible’.

\sphinxAtStartPar
It is a continuous in progress. Bear with us. Better yet, collaborate with us.


\chapter{ABSTRACT}
\label{\detokenize{abstract:abstract}}\label{\detokenize{abstract::doc}}
\sphinxAtStartPar
\sphinxstylestrong{Title}:  Lessons Learned from the FlexSpec1 Spectrograph

\sphinxAtStartPar
(Accepted TBD, Proceedings of the Small Telescope Symposium 2022, Society
for Astronomical Sciences).
\begin{description}
\item[{Authors}] \leavevmode{[}Jerrold L. Foote, Wayne Green, Greg Jones, Frank Parks, Anthony{]}
\sphinxAtStartPar
Rodda, Forrest Sims, Thomas C. Smith, Clarke Yeager

\end{description}

\sphinxAtStartPar
\sphinxstylestrong{Abstract}:
This final paper of the FlexSpec1 series presents lessons learned and applied to 3D\sphinxhyphen{}Printed/Hybrid spectrographs for small telescope science. We present a detailed analysis of the photon\sphinxhyphen{}path from source to sensor and the steps within the spectrograph to mitigate instrumentation issues.  We present the caliberation unit, the “Kzin” ring, as a stand\sphinxhyphen{}alone aspect of the project suited to all spectrographs. A full implementation is in the puiblic domain via Github. Public documentation is available via ReadTheDocs.io. The full design is available via GitHub, the documentation is at ‘readthedocs.io flexspec1 \textless{}\sphinxurl{https://flexspec1.readthedocs.io/en/latest}\textgreater{}\_.

\sphinxAtStartPar
\sphinxstylestrong{Note to Editor}:

\sphinxAtStartPar
The GitHub and readthedocs.io contents are currently being maintained
at the ‘private’ level until release.


\chapter{Download FlexSpec1 Files}
\label{\detokenize{flexspec1:download-flexspec1-files}}\label{\detokenize{flexspec1::doc}}
\sphinxAtStartPar
FlexSpec1 is managed and distributed using the super zip\sphinxhyphen{}like
*cough* environment managed by the free \sphinxstylestrong{git} source code management (SCM)
system. Git will manage anything, not just computer code. So, we use
it! We distribute the STL files, and their original package’s
sources. SLDPRT for Solidworks for example. We also distribute code
for the Raspberry Pi, Arduinos, documentation, python utilities.  We
greatly appreciate SolidWorks support of community education projects.
We use KiCAD (pronounced “Kee CAD”) for EDA (Electronic Design
Automation) schematics. We may use other CAD programs to make
files. FreeCad is the latest adventure. Our general rule is to use
free or accessible software so everyone may participate. This means
any age, anywhere, anytime.

\sphinxAtStartPar
We are sharing a working dynamic release of the files and
documentation we develop for ourselves. Our files are your files. We
use the MIT license. This license boils down to copy\sphinxhyphen{}center. Take our
stuff to the copy\sphinxhyphen{}center and make all the copies you want. You can not
claim our work or limit our ownership. We can offer to share with you
and we do so freely. See the license for the liability statements etc.

\sphinxAtStartPar
\sphinxstylestrong{OK I want to build a spectrograph.}

\begin{sphinxVerbatim}[commandchars=\\\{\}]
\PYG{c+c1}{\PYGZsh{} Use PowerShell for Windows:}
mkdir C:\PYG{l+s+se}{\PYGZbs{}g}it
\PYG{n+nb}{cd} C:\PYG{l+s+se}{\PYGZbs{}g}it
git clone https://github.com/dxwayne/FlexSpec1
\end{sphinxVerbatim}

\sphinxAtStartPar
Or with \sphinxstylestrong{linux}:

\begin{sphinxVerbatim}[commandchars=\\\{\}]
\PYG{c+c1}{\PYGZsh{} Use bash for Mac/Linux}
mkdir \PYG{n+nv}{\PYGZdl{}HOME}/git
\PYG{n+nb}{cd} \PYG{n+nv}{\PYGZdl{}HOME}/git
git clone https://github.com/dxwayne/FlexSpec1
\end{sphinxVerbatim}

\sphinxAtStartPar
TaDa you have what we have.

\sphinxAtStartPar
\sphinxstylestrong{What did I get into?}

\sphinxAtStartPar
Under FlexSpec1/docs/source you find a bunch of .rst files. These
are used to make our \sphinxhref{https://flexspec1.readthedocs.io/en/latest/}{readthedocs.io files}
site. \sphinxstylestrong{Read All About It}.

\sphinxAtStartPar
In the lower\sphinxhyphen{}right of the page, you should find a little box:
“v:latest”. Hit that little down\sphinxhyphen{}caret and you can get a PDF.

\sphinxAtStartPar
This documentation is made with Sphinx documentation management, coordinated
with shared credentials with readthedocs.io.


\chapter{Open Questions}
\label{\detokenize{openquestions:open-questions}}\label{\detokenize{openquestions::doc}}
\sphinxAtStartPar
This is a very fluid section. Open questions are addressed here.

\sphinxAtStartPar
9 June 2021 Tony Rodda reported the fact that shifting the focus of
a commercial (Pentax) lens between he optical infinity mark and the
little ‘dot’ representing the infinity focus for infrared, the focus
changed over the dispersion axis.

\sphinxAtStartPar
\sphinxstylestrong{First and foremost:} we already know using a parabolic mirror as
a collimating lens is the best approach. The camera lens is
an expedient way to form the image.

\sphinxAtStartPar
This traces back to the fact that an achromat is balanced for two
of the Fraunhofer bands. An Apochromat is balanced for three. Lens
are basically prisms and focus is a function of wavelength.

\sphinxAtStartPar
\sphinxstylestrong{We determined}: The normal for the camera lens has to be tilted
a very small amount to compensate for spectral focus shift across
the dispersion axis projected onto the sensor.

\sphinxAtStartPar
\sphinxstylestrong{We note}: most inexpensive cameras do not maintain enough simple
quality control to make the sensor normal to the optical axis. Since
after\sphinxhyphen{}market measures are needed to compensate for the tilt, might
as well “feature it”. The difference between a “Bug” and a “Feature”
is the necktie!

\begin{figure}[htbp]
\centering

\noindent\sphinxincludegraphics{{BugFeature}.png}
\end{figure}

\sphinxAtStartPar
The use of a commercial camera lens also helps in achieving a
flatter field for the science camera (which then improves
calibration) and assists greatly with maintaining overall
collimation.


\chapter{OVIO Slit}
\label{\detokenize{ovio:ovio-slit}}\label{\detokenize{ovio::doc}}
\sphinxAtStartPar
The OVIO slit is a chrome\sphinxhyphen{}plated 1.5mm glass substrate and has 12
positions/widths:


\begin{savenotes}\sphinxattablestart
\centering
\sphinxcapstartof{table}
\sphinxthecaptionisattop
\sphinxcaption{Ovio Slit}\label{\detokenize{ovio:id1}}
\sphinxaftertopcaption
\begin{tabular}[t]{|\X{10}{60}|\X{50}{60}|}
\hline
\sphinxstyletheadfamily 
\sphinxAtStartPar
Position
&\sphinxstyletheadfamily 
\sphinxAtStartPar
Size
\\
\hline
\sphinxAtStartPar
1
&
\sphinxAtStartPar
10 microns
\\
\hline
\sphinxAtStartPar
2
&
\sphinxAtStartPar
20 microns
\\
\hline
\sphinxAtStartPar
3
&
\sphinxAtStartPar
30 microns
\\
\hline
\sphinxAtStartPar
4
&
\sphinxAtStartPar
40 microns
\\
\hline
\sphinxAtStartPar
5
&
\sphinxAtStartPar
50 microns
\\
\hline
\sphinxAtStartPar
6
&
\sphinxAtStartPar
70 microns
\\
\hline
\sphinxAtStartPar
7
&
\sphinxAtStartPar
100 microns
\\
\hline
\sphinxAtStartPar
8
&
\sphinxAtStartPar
150 microns
\\
\hline
\sphinxAtStartPar
9
&
\sphinxAtStartPar
200 microns
\\
\hline
\sphinxAtStartPar
10
&
\sphinxAtStartPar
300 microns
\\
\hline
\sphinxAtStartPar
11
&
\sphinxAtStartPar
500 (Shutter)
\\
\hline
\sphinxAtStartPar
12
&
\sphinxAtStartPar
700 microns
\\
\hline
\end{tabular}
\par
\sphinxattableend\end{savenotes}

\sphinxAtStartPar
The 500 micron (next to largest size) is blocked to be used as a
simple shutter to allow taking zero and dark imagess during the
daytime.

\sphinxAtStartPar
The slit disc thickness is 1.5 mm.

\sphinxAtStartPar
The slit length is 6mm which is larger than most commercial
spectrographs. Benefits are easier placement of targets in the
guide camera field of view and greater coverage of extended
objects.

\sphinxAtStartPar
We briefly investigated the idea of reversing the slit in its
holder such that the mirrored side faces the incoming beam.
We found that the guide image was noticeably improved in
briightness and that secondary images associated with bright
stars were removed.

\sphinxAtStartPar
We did not detect any degradation in the recorded spectra,
however, we have maintained the convestion of a “reversed” OVIO
wheel for the present until we can investigate further.

\sphinxAtStartPar
The following images show a white light bench test of both
orientations.

\begin{figure}[htbp]
\centering

\noindent\sphinxincludegraphics{{GuideImageStandard}.jpeg}
\end{figure}

\begin{figure}[htbp]
\centering

\noindent\sphinxincludegraphics{{GuideImageReversed}.jpeg}
\end{figure}

\sphinxAtStartPar
We also investigated the Fraunhofer diffraction pattern as it
emerges from the spectrograph side of the slit because theory
predicts that this will cause a degradation in overall spectra
SNR due to stray light patterns withion the spectrograph body.

\sphinxAtStartPar
INSERT IMAGE \sphinxhyphen{} Waynes diffraction pattern.

\sphinxAtStartPar
This is shown in the attached image which is a highly
magnified view of the slit taken from the science camera
with the grating replaced by a mirror in the zeroeth order.

\begin{figure}[htbp]
\centering

\noindent\sphinxincludegraphics{{FS1FrFringe}.png}
\end{figure}

\sphinxAtStartPar
The fringe pattern can clearly be seen. Approximately
95\% of the signal is contained within the first lobe.
Additional lobes will cause noise in the signal through
loss of collimation and focus.

\sphinxAtStartPar
We remedied this by inserting a baffle behind the OVIO
mounting such that the width of the baffle exit pupil
removed the unwanted lobes.  This appreciably improved
the sharpness of the spectral calibration lines.

\sphinxAtStartPar
Since this effect is clearly dependent upon wavelength
and the focal ratio of the incoming beam a compromise
baffel exit pupil width must be allowed to accomodate
both the red and blue ends of the spectrum.

\sphinxAtStartPar
In addition, a baffle must be made for each desired
telescope focal ratio.  The baffles are easy to modify
and fit as a 3D printed part.

\begin{figure}[htbp]
\centering

\noindent\sphinxincludegraphics{{Fraunhofer_baffle}.png}
\end{figure}

\sphinxAtStartPar
The position of the baffle is shown in the following
figure.  It can be seen as the red protrusion
attached to the rear of the OVIO wheel mount.

\begin{figure}[htbp]
\centering

\noindent\sphinxincludegraphics{{FS1_Assembly_with_Baffle}.jpeg}
\end{figure}

\sphinxAtStartPar
A development for the FS2 will be a motorised selectable
slit wheel.


\chapter{Care of Optics}
\label{\detokenize{optics:care-of-optics}}\label{\detokenize{optics::doc}}
\sphinxAtStartPar
When handling optics uses gloves, nitrile or pure clean untreated
cotton gloves. This will protect the optical elements and protect
you from harsh cleaning chemicals. Optics should be stored in an
airtight container to avoid reaction with pollutants.
\begin{enumerate}
\sphinxsetlistlabels{\arabic}{enumi}{enumii}{}{.}%
\item {} 
\sphinxAtStartPar
\sphinxstylestrong{Never} use silcone based adhesives or silicone based lubricants any
where near optics. They outgas and condense on the optical surfaces.

\item {} 
\sphinxAtStartPar
The optics should never be touched with your bare hands.  Always
use clean powder free latex or nitrile gloves. Finger prints leave
oils, acids and other volatiles on the surfaces which in the cases of
gratings can not be removed. On lenses or mirrors they may etch the
surface coatings.

\item {} 
\sphinxAtStartPar
If after careful handling a finger print gets on the surface a
micro fiber cloth may remove it if used gently.  DON’T try to wipe a
grating it will scratch.

\item {} 
\sphinxAtStartPar
Front surface mirrors are also fragile. Usually they are over
coated for protection but the coating may be very thin.

\item {} 
\sphinxAtStartPar
Don’t use \sphinxstylestrong{Windex} ® to clean optics. It contains ammonia which
can remove aluminum coating. Use Everclear ® grain alcohol, 95\%
ethanol alcohol or commercial lens cleaning solution as a cleaning
fluid. Us it sparingly. Grain alcohol does not contain denaturing
agents and should not leave a fine white powder on optics.

\sphinxAtStartPar
Like most alcohols, isopropyl alcohol reacts with active metals
such as potassium to form alkoxides that can be called
isopropoxides. The reaction with aluminum (initiated by a trace
of mercury) is used to prepare the catalyst aluminum
isopropoxides. Most IPA isopropyl alcohol has denaturing
agents added and will leave a fine white powder.

\item {} 
\sphinxAtStartPar
Gratings are very susceptible to handling problems. Even with
gloves never touch the ruled surface.  Any scratches or finger prints
will introduce scattered light in to the system. A spectrograph is
always starved for light and to scatter some of the light will reduce
the throughput.

\end{enumerate}


\section{Tips for optics in a spectrograph}
\label{\detokenize{optics:tips-for-optics-in-a-spectrograph}}\begin{enumerate}
\sphinxsetlistlabels{\arabic}{enumi}{enumii}{}{.}%
\item {} 
\sphinxAtStartPar
Grating orientation: if a commercial blazed grating is used it
will have an arrow drawn on the top, bottom or both edges that are
perpendicular to the grooves.  This arrow shows the blaze
direction. The arrow should point toward the incoming beam.  This
orientation maximizes the energy into the first order. If the
arrow is missing it is possible to look into the spectrographs
camera port, without the camera lens in place, while light is
shinning on the slit, rotate the grating so that the zero’th order
can be seen. Then rotate the grating further to pick up the
spectrum starting with the blue end. Further grating rotation will
the bring up the second order. The first order should be very much
brighter than the second order.  If it isn’t the grating
orientation in the holder should be reversed.

\item {} 
\sphinxAtStartPar
If the dispersion along the image from the sensor is not
horizontal (follows a ‘row’ of pixels), the camera needs to be
rotated such that it is.

\item {} 
\sphinxAtStartPar
If the calibration lamp emission lines are not vertical with
respect to the grating; the OVIO slit needs to be rotated in it’s
holder to make them vertical. This can be fiddly sometimes to get
right. Also the vertical slit must be co\sphinxhyphen{}linear with the grating
grooves.

\item {} 
\sphinxAtStartPar
If lenses are not a snug fit into their holders a small
application of white glue on the holder will hold them in place
but allow later removal if needed. Don’t get a glue string on the
lens.

\item {} 
\sphinxAtStartPar
If the lens is a snug fit into the holder make sure that you don’t
apply too much pressure to insert it. Some lower quality achromat
lens elements are not glued in a way that is precisely centered
and while pushing it in an edge might get caught and split the
elements apart.

\item {} 
\sphinxAtStartPar
Depending on many factors (lens focal length, aperture, quality)
the blue and red ends of the spectra may be out of focus while the
center is sharp. This is normal with simple achromat lenses used
for the camera lens. If a commercial photographic lens is used for
the camera lens this is much less of a problem as the multiple
elements in the commercial lens are designed to flatten the field.

\item {} 
\sphinxAtStartPar
When combined with large chips, short focal length lenses, with
short radii may require the sensors plane to be shifted. This is
easily achieved with after\sphinxhyphen{}marked sensor/camera tilt correctors.

\item {} 
\sphinxAtStartPar
A Cheshire eyepiece (laser collimator) is a very handy tool to get
things lined up in the spectrograph, particularly the guide mirror
and main mirror. It is also an easy way to check if the grating is
mounted with it’s surface parallel with the rotation stage. If it
is a green laser and the grating stage rotated till the beam is
centered in the camera port, then the first spectra will be close
to centered. If red then adjust the grating angle such that it is
where H\sphinxhyphen{}alpha emission is expected on the image.

\item {} 
\sphinxAtStartPar
Achromat lens orientation. When mounting lenses the side with the
bigger “bulge” should be oriented toward the grating. Both for the
collimating lens as well as the camera lens.

\end{enumerate}


\chapter{Optical Experiments}
\label{\detokenize{experiments:optical-experiments}}\label{\detokenize{experiments::doc}}
\sphinxAtStartPar
There are several optical experiments to consider.
\begin{enumerate}
\sphinxsetlistlabels{\arabic}{enumi}{enumii}{}{.}%
\item {} 
\sphinxAtStartPar
Orientation of grating.

\item {} 
\sphinxAtStartPar
Focus of the camera/collimator.

\end{enumerate}


\section{Orientation of Grating}
\label{\detokenize{experiments:orientation-of-grating}}
\sphinxAtStartPar
Commercial spectrographs have been received with the grating
facing the wrong way. This is a simple experiment to determine
the direction of the blazed order:

\sphinxAtStartPar
Materials:
\begin{enumerate}
\sphinxsetlistlabels{\arabic}{enumi}{enumii}{}{.}%
\item {} 
\sphinxAtStartPar
A dimmer laser pointer.

\end{enumerate}

\sphinxAtStartPar
\#. A simple littrow prism from Surplus Shed. (L2003D) or a Herschel wedge
solar ‘filter’.
\begin{enumerate}
\sphinxsetlistlabels{\arabic}{enumi}{enumii}{}{.}%
\item {} 
\sphinxAtStartPar
White Index Card

\end{enumerate}

\sphinxAtStartPar
Open the spectrograph just enough to insert the card, and shine the laser beam
onto the grating. The angle is not that important for this test a large angle
will be OK \textendash{} as long as it is not too large.

\sphinxAtStartPar
The littrow prism/Herschel wedge reduces the laser brightness by about 95 percent.
If the laser pointer is very dim, it is not needed.

\sphinxAtStartPar
Place a white card to receive the reflection of a dim laser beam. The brightest
dot is the specular reflection (non\sphinxhyphen{}dispersed order\sphinxhyphen{}0 beam). The brightest
spot will be the ‘blazed’ beam. This is most often the +1 order. On rare
occasions (surplus gratings for example) may have the blaze set to a different
order.


\section{Focus of the Camera/Collimator}
\label{\detokenize{experiments:focus-of-the-camera-collimator}}
\sphinxAtStartPar
In order to achieve and maintain accurate collimation to and from
the grating we find that focusing the science camera on infinity at the
outset and solely focusing the collimationg mirror during operation
results in the best possible set up.  This may seem obvious but most
commercial spectrographs allow the user to change the science camera
focus during an observing session (because the collimating mirror is
usually inaccessible).  This may be necessary because of thermal
changes or flexure in the spectrograph body.  The result is
a badly collimated system.

\sphinxAtStartPar
We suggest the following method of set up.

\sphinxAtStartPar
Position a camera (a DSLR will work) with its lens focused for
infinity using a very distant point source. A far away
street\sphinxhyphen{}light or a bright star works. Most commercial camera
lenses will focus IR light, hence the full stop is NOT
focused for visible light.

\sphinxAtStartPar
Place a front surface mirror at the grating position, and the camera
at the exit ‘camera’ port of the spectrograph. Then focus the slit’s
image using the collimating mirror. This is a ‘real’ close
approximation for the collimator lens focus.

\sphinxAtStartPar
The FlexSpec has a remote collimating mirror adjuster driven by
a stepper motor and eccentri cam arrangement for fine focus during
an observing session.  This means the FlexSpec can be focused
without touching.

\sphinxAtStartPar
Future FlexSpec will have a remote\sphinxhyphen{}selectable grating changer. Plans
are to include a front\sphinxhyphen{}surface mirror.


\chapter{OVERVIEW}
\label{\detokenize{overview:overview}}\label{\detokenize{overview::doc}}
\sphinxAtStartPar
This documentation has all {\hyperref[\detokenize{bom:flexspec1-bom}]{\sphinxcrossref{\DUrole{std,std-ref}{bill\sphinxhyphen{}of\sphinxhyphen{}materials}}}} and in\sphinxhyphen{}depth discussion of
the concepts, development, technical and scientific aspects of
Flexspec 1. “Flexible” is “Flexible”. A concrete example
of the Flexspec 1 is presented but the ideas, CAD drawings,
software and documentation are available in the public\sphinxhyphen{}domain
for vendors and fellow small telescope scientists to use and
enjoy.

\sphinxAtStartPar
During an after\sphinxhyphen{}meeting in Flagstaff, AZ in late 2018 a few of the
authors met and developed the idea of leveraging Paul Gerlach’s
LOWSPEC 3D Printed spectrograph as a means of gaining more data from
the small telescope community. In early 2019, at the Sacramento
Mountains Spectroscopy Workshop II in Las Cruces NM the decision was
taken to refine 3D printed spectrographs. This collaboration will
continue beyond this paper.

\sphinxAtStartPar
The Flexspec 1 small instrument operating as a full peer within a
network of sites with other instruments.  Flexspec employs multiple
control components and can share its state\sphinxhyphen{}of\sphinxhyphen{}affairs with other
interested control elements throughout its associated networks. For
example, one does not start a deep spectral exposure while a neighbor
in a roll\sphinxhyphen{}off building is taking dome flats. We made an attempt to
design a control messaging architecture to address the full
control/scheduler issue. Those implementation details will be
addressed in future. See Section {\hyperref[\detokenize{controls:fs1-control}]{\sphinxcrossref{\DUrole{std,std-ref}{Controls}}}}.

\sphinxAtStartPar
Adding automated control elements such as motors, sensors, and
switches to a spectrograph were driven by experience and issues with
other instruments.

\sphinxAtStartPar
Issues include:
\begin{itemize}
\item {} 
\sphinxAtStartPar
\sphinxstylestrong{Thermal Expansion:} The complete thermal range is from daytime 118F to nighttime \sphinxhyphen{}20F depending on the season. 3D Plastics have high thermal expansion coefficients.

\item {} 
\sphinxAtStartPar
\sphinxstylestrong{Flexture:} PLA/Plastics are highly flexible. Threads are unreliable.

\item {} 
\sphinxAtStartPar
\sphinxstylestrong{Focus:} Temperature changes across a night, bench setup.

\item {} 
\sphinxAtStartPar
\sphinxstylestrong{Seeing:} Seeing conditions are variable as small telescope work is done deep in the atmosphere.

\item {} 
\sphinxAtStartPar
\sphinxstylestrong{Parallactic Angle:} Always a concern. The Arduino Nano Sense 33 IoT (BLE) provides rotator feedback.

\item {} 
\sphinxAtStartPar
\sphinxstylestrong{Remote use:} Here Office to backyard was the goal, architecture was global.

\end{itemize}

\sphinxAtStartPar
During 2020 we became aware of thermal and structural issues. Yeager
did the bulk of these experiments in the the cold Colorado winter.
This year’s work concentrated on the math, physics, optics, and
engineering aspects of the spectrograph. There have been requests for
accommodating apertures beyond the nominal 11\sphinxhyphen{}inch SCT telescopes, to
include short\sphinxhyphen{}focus (f/5 \sphinxhyphen{} f/8) instruments; and apertures in the
ranges of 36\sphinxhyphen{}60cm and beyond.

\index{issues@\spxentry{issues}!thermal@\spxentry{thermal}}\index{issues@\spxentry{issues}!flexure@\spxentry{flexure}}\ignorespaces 
\sphinxAtStartPar
Matching the slit size to focal length and seeing. TODO Move this table.
\begin{quote}


\begin{savenotes}\sphinxattablestart
\centering
\begin{tabular}[t]{|*{8}{\X{1}{8}|}}
\hline

\sphinxAtStartPar
Aperture
&
\sphinxAtStartPar
f/ratio
&
\sphinxAtStartPar
fl
&
\sphinxAtStartPar
Magic
&\sphinxstartmulticolumn{4}%
\begin{varwidth}[t]{\sphinxcolwidth{4}{8}}
\sphinxAtStartPar
Seeing (arcsec)
\par
\vskip-\baselineskip\vbox{\hbox{\strut}}\end{varwidth}%
\sphinxstopmulticolumn
\\
\hline
\sphinxAtStartPar
(inch)
&
\sphinxAtStartPar
(inch)
&\begin{enumerate}
\sphinxsetlistlabels{\roman}{enumi}{enumii}{(}{)}%
\setcounter{enumi}{1999}
\item {} 
\end{enumerate}
&
\sphinxAtStartPar
“/\(\mu{m}\)
&
\sphinxAtStartPar
1
&
\sphinxAtStartPar
2
&
\sphinxAtStartPar
3
&
\sphinxAtStartPar
4
\\
\hline
\sphinxAtStartPar
6
&
\sphinxAtStartPar
5
&
\sphinxAtStartPar
735
&
\sphinxAtStartPar
0.28063
&
\sphinxAtStartPar
3.56338
&
\sphinxAtStartPar
7.12676
&
\sphinxAtStartPar
10.69014
&
\sphinxAtStartPar
14.25352
\\
\hline
\sphinxAtStartPar
6
&
\sphinxAtStartPar
6.7
&
\sphinxAtStartPar
984.9
&
\sphinxAtStartPar
0.20943
&
\sphinxAtStartPar
4.77493
&
\sphinxAtStartPar
9.54986
&
\sphinxAtStartPar
14.32479
&
\sphinxAtStartPar
19.09972
\\
\hline
\sphinxAtStartPar
6
&
\sphinxAtStartPar
8
&
\sphinxAtStartPar
1176
&
\sphinxAtStartPar
0.17540
&
\sphinxAtStartPar
5.70141
&
\sphinxAtStartPar
11.40282
&
\sphinxAtStartPar
17.10423
&
\sphinxAtStartPar
22.80564
\\
\hline
\sphinxAtStartPar
8
&
\sphinxAtStartPar
5
&
\sphinxAtStartPar
980
&
\sphinxAtStartPar
0.21047
&
\sphinxAtStartPar
4.75117
&
\sphinxAtStartPar
9.50235
&
\sphinxAtStartPar
14.25352
&
\sphinxAtStartPar
19.00470
\\
\hline
\sphinxAtStartPar
8
&
\sphinxAtStartPar
6
&
\sphinxAtStartPar
1176
&
\sphinxAtStartPar
0.17540
&
\sphinxAtStartPar
5.70141
&
\sphinxAtStartPar
11.40282
&
\sphinxAtStartPar
17.10423
&
\sphinxAtStartPar
22.80564
\\
\hline
\sphinxAtStartPar
8
&
\sphinxAtStartPar
10
&
\sphinxAtStartPar
1960
&
\sphinxAtStartPar
0.10524
&
\sphinxAtStartPar
9.50235
&
\sphinxAtStartPar
19.00470
&
\sphinxAtStartPar
28.50705
&
\sphinxAtStartPar
38.00939
\\
\hline
\sphinxAtStartPar
11
&
\sphinxAtStartPar
5
&
\sphinxAtStartPar
1347.5
&
\sphinxAtStartPar
0.15307
&
\sphinxAtStartPar
6.53286
&
\sphinxAtStartPar
13.06573
&
\sphinxAtStartPar
19.59859
&
\sphinxAtStartPar
26.13146
\\
\hline
\sphinxAtStartPar
11
&
\sphinxAtStartPar
11
&
\sphinxAtStartPar
2964.5
&
\sphinxAtStartPar
0.06958
&
\sphinxAtStartPar
14.37230
&
\sphinxAtStartPar
28.74460
&
\sphinxAtStartPar
43.11691
&
\sphinxAtStartPar
57.48921
\\
\hline
\sphinxAtStartPar
12
&
\sphinxAtStartPar
4
&
\sphinxAtStartPar
1176
&
\sphinxAtStartPar
0.17540
&
\sphinxAtStartPar
5.70141
&
\sphinxAtStartPar
11.40282
&
\sphinxAtStartPar
17.10423
&
\sphinxAtStartPar
22.80564
\\
\hline
\sphinxAtStartPar
12
&
\sphinxAtStartPar
13.5
&
\sphinxAtStartPar
3969
&
\sphinxAtStartPar
0.05197
&
\sphinxAtStartPar
19.24226
&
\sphinxAtStartPar
38.48451
&
\sphinxAtStartPar
57.72677
&
\sphinxAtStartPar
76.96902
\\
\hline
\sphinxAtStartPar
14
&
\sphinxAtStartPar
6.7
&
\sphinxAtStartPar
2298.1
&
\sphinxAtStartPar
0.08975
&
\sphinxAtStartPar
11.14150
&
\sphinxAtStartPar
22.28301
&
\sphinxAtStartPar
33.42451
&
\sphinxAtStartPar
44.56601
\\
\hline
\sphinxAtStartPar
14
&
\sphinxAtStartPar
8
&
\sphinxAtStartPar
2744
&
\sphinxAtStartPar
0.07517
&
\sphinxAtStartPar
13.30329
&
\sphinxAtStartPar
26.60658
&
\sphinxAtStartPar
39.90986
&
\sphinxAtStartPar
53.21315
\\
\hline
\sphinxAtStartPar
16
&
\sphinxAtStartPar
4
&
\sphinxAtStartPar
1568
&
\sphinxAtStartPar
0.13155
&
\sphinxAtStartPar
7.60188
&
\sphinxAtStartPar
15.20376
&
\sphinxAtStartPar
22.80564
&
\sphinxAtStartPar
30.40752
\\
\hline
\sphinxAtStartPar
16
&
\sphinxAtStartPar
13.5
&
\sphinxAtStartPar
5292
&
\sphinxAtStartPar
0.03898
&
\sphinxAtStartPar
25.65634
&
\sphinxAtStartPar
51.31268
&
\sphinxAtStartPar
76.96902
&
\sphinxAtStartPar
102.62536
\\
\hline
\sphinxAtStartPar
24
&
\sphinxAtStartPar
4
&
\sphinxAtStartPar
2352
&
\sphinxAtStartPar
0.08770
&
\sphinxAtStartPar
11.40282
&
\sphinxAtStartPar
22.80564
&
\sphinxAtStartPar
34.20845
&
\sphinxAtStartPar
45.61127
\\
\hline
\sphinxAtStartPar
24
&
\sphinxAtStartPar
8
&
\sphinxAtStartPar
4704
&
\sphinxAtStartPar
0.04385
&
\sphinxAtStartPar
22.80564
&
\sphinxAtStartPar
45.61127
&
\sphinxAtStartPar
68.41691
&
\sphinxAtStartPar
91.22255
\\
\hline
\end{tabular}
\par
\sphinxattableend\end{savenotes}

\sphinxAtStartPar
Table of focal lengths matching slit widths with seeing.
\end{quote}

\index{seeing@\spxentry{seeing}!slit width@\spxentry{slit width}}\ignorespaces 

\chapter{Physics}
\label{\detokenize{physics:physics}}\label{\detokenize{physics::doc}}
\sphinxAtStartPar
This section contains the basic mathematical underpinnings used in our
analysis of the basic DIY spectroscope.

\sphinxAtStartPar
The optical \index{sign conventions@\spxentry{sign conventions}}sign conventions used here follow Schroeder’s (1987)
“Cartesian” approach. The plane for the optics (“lens/mirror surfaces”)
are in the XY plane, the optical axis follows the Z axis.  Along the Z
axis, rays pass from the negative (lesser) to right
(greater). Reflections reverse the sign. (I.e.: photons move in the
positive direction along the Z axis.)

\sphinxAtStartPar
There are at least 6 dispersion techniques to consider:
\begin{itemize}
\item {} \begin{enumerate}
\sphinxsetlistlabels{\arabic}{enumi}{enumii}{(}{)}%
\item {} 
\sphinxAtStartPar
Fresnel (nearfield),

\end{enumerate}

\item {} \begin{enumerate}
\sphinxsetlistlabels{\arabic}{enumi}{enumii}{(}{)}%
\item {} 
\sphinxAtStartPar
Fraunhofer (slit) diffraction,

\end{enumerate}

\item {} \begin{enumerate}
\sphinxsetlistlabels{\arabic}{enumi}{enumii}{(}{)}%
\setcounter{enumi}{1}
\item {} 
\sphinxAtStartPar
Reflection and transmission diffraction gratings,

\end{enumerate}

\item {} \begin{enumerate}
\sphinxsetlistlabels{\arabic}{enumi}{enumii}{(}{)}%
\setcounter{enumi}{1}
\item {} 
\sphinxAtStartPar
Prisms and a grism (combination prism/transmission grating).

\end{enumerate}

\end{itemize}

\sphinxAtStartPar
A lot of the mathematical manipulations used here are accessible via
the \sphinxhref{https://www.wolframalpha.com}{Wolfram Alpha web site} , freely accessible and with paid
subscriptions.


\section{BOTE Computation}
\label{\detokenize{physics:bote-computation}}
\sphinxAtStartPar
To assist with basic design and to develop intuition, “Back of The
Envelope” \index{BOTE@\spxentry{BOTE}}BOTE computation is in
order. Astronomers utilize the “Centimeter Gram Seconds” (CGS)
system. Here are a few guide pointers:
\begin{enumerate}
\sphinxsetlistlabels{\arabic}{enumi}{enumii}{}{.}%
\item {} \begin{description}
\item[{Use the same units, here centimeters;}] \leavevmode\begin{itemize}
\item {} 
\sphinxAtStartPar
\(5 \times 10^{-5}\) cm for green light. \(5 \times 10^{-5}\) cm for blue and \(6 \times 10^{-5}\) cm for red. We use 300 and 600 l/mm gratings and the math cancels!

\item {} 
\sphinxAtStartPar
Translate grating \(l/mm\; to\l\) \(l/cm\)

\end{itemize}

\end{description}

\item {} 
\sphinxAtStartPar
Remember sign conventions, and get the right equation.

\item {} 
\sphinxAtStartPar
Follow the general spectrograph equation chain. Slit/grating/sensor.

\end{enumerate}


\section{Light}
\label{\detokenize{physics:light}}
\sphinxAtStartPar
A wavelength is a cycle, its length is a function of the energy of the
associated photon. An image has a ‘wavefront’ and that wavefront has a
quality, expressed in fractions of a wavelength (variation of 1/10th
of a wavelength \(2\times\pi/10 = 0.328\) (360 degrees for a full
cycle) amounts to a shift of around 36 degrees! A lot of error. Using
a ‘folding’ mirror \index{mirror@\spxentry{mirror}!folding@\spxentry{folding}}mirror;folding adds error. This same
wavefront error seeps in with irregular coatings and lens surfaces.  It
creeps in with the parallelism agreement between the front and back of a
lens; with the co\sphinxhyphen{}alignment of lenses; and with the flatness of a
grating’s plane.


\section{The Behavior of Lenses}
\label{\detokenize{physics:the-behavior-of-lenses}}
\sphinxAtStartPar
In theory, lenses work well. In practice there is the issue of the
agreement between the curve at the front of the lens and that at the
back of the lens (a wedge or a prism\sphinxhyphen{}astigmatism effect). The distance
between a surface at the edge adds additional chances for
dispersion of the light as it passes through the lens. The cheaper the
lens the more likely the lens has issues.

\sphinxAtStartPar
Glass refracts, deferentially, as a function of energy (wavelength).


\section{Lens Makers Equation}
\label{\detokenize{physics:lens-makers-equation}}
\sphinxAtStartPar
The general form of the \index{lens makers equation@\spxentry{lens makers equation}}lens makers equation:
\begin{equation*}
\begin{split}\frac{1}{f} = (n-1) \left[ \frac{1}{R_1} - \frac{1}{R_2} + \frac{(n-1)d}{n R_1 R_2} \right]\end{split}
\end{equation*}
\sphinxAtStartPar
where:
\begin{itemize}
\item {} 
\sphinxAtStartPar
\(n\)  is the index of refraction,

\item {} 
\sphinxAtStartPar
\(f\)   the lens’ focal lengths,

\item {} 
\sphinxAtStartPar
\(R_1\) is the \sphinxstylestrong{radius of curvature} of the surface towards the object,

\item {} 
\sphinxAtStartPar
\(R_2\) is the radius of curvature of the surface towards the sensor, and

\item {} 
\sphinxAtStartPar
\(d\)   is the thickness of the lens

\end{itemize}

\sphinxAtStartPar
See \sphinxhref{https://en.wikipedia.org/wiki/Lens\#Lensmaker's\_equation}{Wikipedia’s Lens Makers Equation}.

\sphinxAtStartPar
Thin Lens Equation:

\sphinxAtStartPar
In practice the thin lens equation:
\begin{equation}\label{equation:physics:thinlens}
\begin{split}\frac{1}{f} = \frac{1}{S_1} + \frac{1}{S_2}\end{split}
\end{equation}

\section{Snell’s Law}
\label{\detokenize{physics:snell-s-law}}\begin{equation*}
\begin{split}\frac{\sin\theta_2}{\sin\theta_1} =  \frac{n_1}{n_2}
 =\frac{v_2}{v_1}\end{split}
\end{equation*}
\sphinxAtStartPar
where:

\sphinxAtStartPar
TODO: This needs to be verified the image names match the labels.
\begin{itemize}
\item {} 
\sphinxAtStartPar
\(\theta_x\) are the angles involved.

\item {} 
\sphinxAtStartPar
\(n\)  is the index of refraction,

\item {} 
\sphinxAtStartPar
\(v\)  is the speed of propagation

\end{itemize}


\section{Reflection}
\label{\detokenize{physics:reflection}}
\sphinxAtStartPar
The \index{OVIO slit@\spxentry{OVIO slit}}OVIO slit is not an air slit. Internal reflection
tends to scramble the beam at this point.

\begin{figure}[htbp]
\centering
\capstart

\noindent\sphinxincludegraphics[scale=0.4]{{xxrefractionlaw1}.png}
\caption{Total reflection (mirror).}\label{\detokenize{physics:id3}}\end{figure}

\begin{figure}[htbp]
\centering
\capstart

\noindent\sphinxincludegraphics[scale=0.4]{{xxrefractionlaw2}.png}
\caption{Reflection from a glass surface, portion of the I:subscript:\sphinxtitleref{0} is lost.}\label{\detokenize{physics:id4}}\end{figure}

\begin{figure}[htbp]
\centering
\capstart

\noindent\sphinxincludegraphics[scale=0.4]{{xxrefractionlaw3}.png}
\caption{Exit angle refraction. (The other side of the slit substrate.)}\label{\detokenize{physics:id5}}\end{figure}

\begin{figure}[htbp]
\centering
\capstart

\noindent\sphinxincludegraphics[scale=0.4]{{xxrefractionlaw4}.png}
\caption{Total internal reflection.}\label{\detokenize{physics:id6}}\end{figure}

\sphinxAtStartPar
TODO: refine this section.

\sphinxAtStartPar
The \sphinxhref{http://hyperphysics.phy-astr.gsu.edu/hbase/phyopt/freseq.html}{calculator} at:

\sphinxAtStartPar
helps to calculate the pertinent values. A f/5 beam is the arctan(1/5)
= 11.30 degrees. Divided by 2 gives 5.65 degrees \textendash{} from the optical
axis to the side of the converging cone. With n$_{\text{1}}$ =
1.0001 (air) and n$_{\text{2}}$ = 1.53 (BK7\sphinxhyphen{}ish).

\sphinxAtStartPar
s\sphinxhyphen{}polarized:
\begin{equation*}
\begin{split}R_\mathrm{s} = \left|\frac{Z_2 \cos \theta_\mathrm{i} - Z_1 \cos \theta_\mathrm{t}}{Z_2 \cos \theta_\mathrm{i} + Z_1 \cos \theta_\mathrm{t}}\right|^2\end{split}
\end{equation*}
\sphinxAtStartPar
p\sphinxhyphen{}polarized
\begin{equation*}
\begin{split}I = I_0 \frac{ 1+\cos^2 \theta }{2 R^2} \left( \frac{ 2 \pi }{ \lambda } \right)^4 \left( \frac{ n^2-1}{ n^2+2 } \right)^2 \left( \frac{d}{2} \right)^6\end{split}
\end{equation*}
\sphinxAtStartPar
TODO: \sphinxurl{https://en.wikipedia.org/wiki/Total\_internal\_reflection\#/media/File:Total\_internal\_reflection\_by\_fluorescence.jpg}

\sphinxAtStartPar
Using the \sphinxhref{https://www.schott.com/d/advanced\_optics/c36214d9-13c4-468c-bf40-8d438b89f532/1.30/schott-optical-glass-pocket-catalog-jan-2020-row.pdf}{Shott Pocket Catalog}

\sphinxAtStartPar
A point source (star) receives information from the entire aperture with
a slight distortion due to its off\sphinxhyphen{}axis relationship. For spectroscopy,
careful centering and alignment of the ‘target’ position on the slit
onto a co\sphinxhyphen{}aligned optical axis keeps the noise symmetric. Off\sphinxhyphen{}axis
star means asymmetric line profiles at the sensor. Slight, but there.

\sphinxAtStartPar
In considering the \sphinxhref{https://en.ovio-optics.com/media/pim/assets/DocumentsPDF/std.lang.all/2-/en/Notice-Ovio-204012-EN.pdf}{OVIO} slit with a beam consisting a pencil of rays
(bundle of rays \textendash{} each with a different approach angle to the
surface). Using the above equations it is easy to see that the f/ratio
is altered significantly as each ray enters, passes, and exits the
slit’s substrate. There is a condition where internal reflection adds
an additional ray with changed geometry.  This amounts to noise in the
system. It is not a true caustic.


\section{Grating Equation}
\label{\detokenize{physics:grating-equation}}\begin{equation}\label{equation:physics:grating}
\begin{split}\frac{m\lambda}{d} = sin(\alpha) \pm sin(\beta)\end{split}
\end{equation}
\sphinxAtStartPar
where \(\lambda\) is the wavelength, \(\alpha\) is the angle
of incidence, \(\beta\) is the diffracted angle (the dependent
variable of real interest). The sign \(\pm\) is positive for
reflection gratings and negative for transmission grating. The angle
\(\phi = |\alpha| + |\beta|\) defines the main geometry of the
system.


\section{Spot Size and Slit Selection}
\label{\detokenize{physics:spot-size-and-slit-selection}}
\sphinxAtStartPar
In practice, the focal length drives the platescale eq: \eqref{equation:physics:pixelscale}
and aperture drives resolution \eqref{equation:physics:RayleighEquation}.

\sphinxAtStartPar
In general, the \index{spot\sphinxhyphen{}size@\spxentry{spot\sphinxhyphen{}size}}spot\sphinxhyphen{}size is independent of the aperture, and
wholly dependent on the focal length. Since f/ratio is a derived value
\textendash{} equation eq: \eqref{equation:physics:pixelscale} bypasses the aperture and gives a result
as a fraction of an arcsecond per micron at the focal plane:
\begin{equation}\label{equation:physics:pixelscale}
\begin{split}\frac{206204.8}{fl\times 1000} = \frac{x^"}{1 \mu{m}}\end{split}
\end{equation}
\sphinxAtStartPar
This is a elementary school ratio problem: where \(206204.8\) is
the number of arc\sphinxhyphen{}seconds in a radian; \(fl\) is the focal length
in mm (traditionally stated) and multiplied by 1000 to turn into
microns; \(x^"\) is “x” in arc\sphinxhyphen{}seconds; the goal of \(1 \mu{m}\)
relates to both pixel size and slit width.

\sphinxAtStartPar
In spectroscopy, the Rayleigh equation eq: \eqref{equation:physics:RayleighEquation} provides
a critical estimate of the ability to separate the light from two
close objects.
\begin{equation}\label{equation:physics:RayleighEquation}
\begin{split}\theta = 1.22\;\frac{\lambda}{D}\end{split}
\end{equation}
\sphinxAtStartPar
where \(\theta\) is the resolution IN RADIANS; \(1.22\) is the
first Bessel J coefficient; \(\lambda\) is the wavelength in units
of D; and \(D\) is the diameter of the aperture. For
back\sphinxhyphen{}of\sphinxhyphen{}the\sphinxhyphen{}envelope estimates, use 5e\sphinxhyphen{}5 (\(5\times 10^{-5}\)) as
the wavelength in {[}cm{]} and express D in matching units {[}cm{]}.

\sphinxAtStartPar
\sphinxstylestrong{Observation:} The width of the slit dominates resolution within the
spectroscope. The aperture requirement stresses the need for flux over
resolution. A short focal length stresses the need for a small “spot
size”, allowing a smaller slit width; in turn driving higher resolution.

\sphinxAtStartPar
The so\sphinxhyphen{}called ‘diffraction limited’ requirement of one arcsecond for
stations deep in the atmosphere only requires an aperture of around
12.58 cm (4.95 inches). This is all that is satisfy most amateur
needs under above average seeing conditions.


\section{Sellmeier Equation}
\label{\detokenize{physics:sellmeier-equation}}
\sphinxAtStartPar
The \index{Sellmeier Equation@\spxentry{Sellmeier Equation}}Sellmeier Equation, uses an empirical equation to model the
dispersion of light in medium with a given refractive index.

\sphinxAtStartPar
Values may be found at \sphinxcite{references:polyanskiy-2021}: \sphinxhref{https://refractiveindex.info/}{Refractive Index Database}
\begin{equation}\label{equation:physics:sellmeier1}
\begin{split}n^2(\lambda) = 1 + \sum_i \frac{B_i \lambda^2}{\lambda^2 - C_i}\end{split}
\end{equation}
\sphinxAtStartPar
In practice, a small handful of empirical constants determines a
workable solution to path\sphinxhyphen{}times for lenses:
\begin{equation}\label{equation:physics:sellmeier2}
\begin{split}\lambda = 1 + \frac{B_1 \lambda^2 }{ \lambda^2 - C_1}
   + \frac{B_2 \lambda^2 }{ \lambda^2 - C_2}
   + \frac{B_3 \lambda^2 }{ \lambda^2 - C_3}\end{split}
\end{equation}
\sphinxAtStartPar
Coefficients may be had from several sources:
\begin{quote}


\begin{savenotes}\sphinxattablestart
\centering
\begin{tabulary}{\linewidth}[t]{|T|T|}
\hline

\sphinxAtStartPar
B\_1
&
\sphinxAtStartPar
1.03961212
\\
\hline
\sphinxAtStartPar
B\_2
&
\sphinxAtStartPar
0.231792344
\\
\hline
\sphinxAtStartPar
B\_3
&
\sphinxAtStartPar
1.01046945
\\
\hline
\sphinxAtStartPar
C\_1
&
\sphinxAtStartPar
6.00069867 x 10\textasciicircum{} 3 microns
\\
\hline
\sphinxAtStartPar
C\_2
&
\sphinxAtStartPar
2.00179144 x 10\textasciicircum{}\sphinxhyphen{}2 microns
\\
\hline
\sphinxAtStartPar
C\_3
&
\sphinxAtStartPar
1.03560653 x 10\textasciicircum{} 2 microns
\\
\hline
\end{tabulary}
\par
\sphinxattableend\end{savenotes}
\end{quote}


\section{Scattering}
\label{\detokenize{physics:scattering}}
\sphinxAtStartPar
In optical spectroscopy the wavelength ranges from 3300\sphinxhyphen{}12,000 Angstroms
or .33 to 1.2 microns. Particles or defects in the optics may be
modeled by \sphinxcode{\sphinxupquote{\textbackslash{}sigma = 2\textbackslash{}pi\textbackslash{}r/\textbackslash{}lambda}}. Rayleigh scattering occurs when
\sphinxcode{\sphinxupquote{\textbackslash{}sigma \textless{}\textless{} 1}} and \sphinxcode{\sphinxupquote{\textbackslash{}sigma \textless{}\textless{} \textbackslash{}lambda/10}}.
\begin{equation}\label{equation:physics:RayleighIntensity}
\begin{split}I &= I_0 \frac{ 1+\cos^2 \theta }{2 R^2} \left( \frac{ 2 \pi }{ \lambda } \right)^4 \left( \frac{ n^2-1}{ n^2+2 } \right)^2 \left( \frac{d}{2} \right)^6\end{split}
\end{equation}
\sphinxAtStartPar
\textless{}ref\textgreater{}Seinfeld, John H. and Pandis, Spyros N. (2006) ‘’Atmospheric Chemistry and Physics, 2nd Edition’’, John Wiley and Sons, New Jersey, Chapter 15.1.1, \{\{ISBN|0471720186\}\}\textless{}/ref\textgreater{}
\begin{equation}\label{equation:physics:RayleighScattering}
\begin{split}\sigma_\text{s} &= \frac{ 2 \pi^5}{3} \frac{d^6}{\lambda^4} \left( \frac{ n^2-1}{ n^2+2 } \right)^2\end{split}
\end{equation}
\sphinxAtStartPar
\sphinxcode{\sphinxupquote{Siegel, R., Howell, J.R., (2002). {[}https://cds.cern.ch/record/1505325/files/9781439805336\_TOC.pdf \textquotesingle{}\textquotesingle{}Thermal radiation heat transfer\textquotesingle{}\textquotesingle{}{]}. p. 480. New York, NY: Taylor \& Francis. \{\{ISBN|1560329688\}\}}}


\section{Lambertian Reflection:}
\label{\detokenize{physics:lambertian-reflection}}
\sphinxAtStartPar
One \index{calibration lamp@\spxentry{calibration lamp}}calibration lamp method involves flooding the guider/slit chamber with
light from an off\sphinxhyphen{}axis lamp. Experiments (Rodda/Yeager) showed variation
in lamp placement had a significant impact in calibration line placement
on the spectrum. Here is the simple case of light from a bulb radiating
outwards. If the lamp is replaced with a patch of the reflecting area
inside the chamber; and considering all contributions from the entire
area of the chamber \textendash{} it is easy to see where the off\sphinxhyphen{}axis contributions
originate.  TODO Add reference to the ‘Greg’s device’.

\begin{figure}[htbp]
\centering
\capstart

\noindent\sphinxincludegraphics[scale=0.3]{{576px-Lambert_Cosine_Law_1}.png}
\caption{Relationship of reflection to solid angle \(\Omega\). (Wikipedia)}\label{\detokenize{physics:id7}}\end{figure}
\begin{equation}\label{equation:physics:Lambert-1}
\begin{split}F_{tot} &= \int\limits_0^{2\pi}\;\int\limits_0^{\pi/2}\cos(\theta)I_{max}\; \sin(\theta)\;\mathbf{d}\theta\;\mathbf{d}\phi \\
&= 2\pi\cdot I_{max}\int\limits_0^{\pi/2}\cos(\theta)\sin(\theta)\;\mathbf{d}\theta \\
&= 2\pi\cdot I_{max}\int\limits_0^{\pi/2}\frac{\sin(2\theta)}{2}\;\mathbf{d}\theta\end{split}
\end{equation}
\sphinxAtStartPar
Using \sphinxhref{https://www.wolframalpha.com}{Wolfram Alpha web site} :

\begin{sphinxVerbatim}[commandchars=\\\{\}]
integrate sin (2x / 2) dx from x = 0 to pi/2
\end{sphinxVerbatim}
\begin{equation*}
\begin{split}\int_0^{\pi/2} sin (2x / 2)\; \mathbf{d}x = 1\end{split}
\end{equation*}

\section{Thermal Expansion}
\label{\detokenize{physics:thermal-expansion}}
\sphinxAtStartPar
In general the degree of thermal expansion is \(\propto\) inverse
of the melting point. In practice the ODE
\begin{equation}\label{equation:physics:ThermalExpansionODE}
\begin{split}\alpha_{L} = \frac{1}{L}\; \frac{\mathbf{d}L}{\mathbf{d}T}\end{split}
\end{equation}
\sphinxAtStartPar
where \(L\) is the length \(T\) is in units of temperature.
FlexSpec 1 uses MKS, so coefficients here are in \(\mu{m}/m/K\).


\begin{savenotes}\sphinxattablestart
\raggedright
\sphinxcapstartof{table}
\sphinxthecaptionisattop
\sphinxcaption{Material Thermal Coefficients}\label{\detokenize{physics:id8}}
\sphinxaftertopcaption
\begin{tabular}[t]{|\X{50}{65}|\X{15}{65}|}
\hline
\sphinxstyletheadfamily 
\sphinxAtStartPar
Standard Material
&\sphinxstyletheadfamily 
\sphinxAtStartPar
Coefficient
\\
\hline
\sphinxAtStartPar
Aluminum
&
\sphinxAtStartPar
23.1
\\
\hline
\sphinxAtStartPar
Brass
&
\sphinxAtStartPar
19
\\
\hline
\sphinxAtStartPar
Carbon steel
&
\sphinxAtStartPar
10.8
\\
\hline
\sphinxAtStartPar
Invar
&
\sphinxAtStartPar
1.2
\\
\hline
\sphinxAtStartPar
Douglas\sphinxhyphen{}fir
&
\sphinxAtStartPar
3.5
\\
\hline&\\
\hline
\sphinxAtStartPar
3D FILAMENTS
&
\sphinxAtStartPar
3D FIBER COEFFICIENTS
\\
\hline
\sphinxAtStartPar
ABS
&
\sphinxAtStartPar
90
\\
\hline
\sphinxAtStartPar
ASA
&
\sphinxAtStartPar
98
\\
\hline
\sphinxAtStartPar
Carbon Fiber Filled
&
\sphinxAtStartPar
57.5
\\
\hline
\sphinxAtStartPar
Flexible
&
\sphinxAtStartPar
157
\\
\hline
\sphinxAtStartPar
HIPS
&
\sphinxAtStartPar
80
\\
\hline
\sphinxAtStartPar
Metal Filled
&
\sphinxAtStartPar
33.75
\\
\hline
\sphinxAtStartPar
Nylon
&
\sphinxAtStartPar
95
\\
\hline
\sphinxAtStartPar
PETG
&
\sphinxAtStartPar
60
\\
\hline
\sphinxAtStartPar
PLA
&
\sphinxAtStartPar
68
\\
\hline
\sphinxAtStartPar
PVA
&
\sphinxAtStartPar
85
\\
\hline
\sphinxAtStartPar
Polycarbonate
&
\sphinxAtStartPar
69
\\
\hline
\sphinxAtStartPar
Polyproplyne
&
\sphinxAtStartPar
150
\\
\hline
\sphinxAtStartPar
Wood Filled
&
\sphinxAtStartPar
30.5
\\
\hline
\end{tabular}
\par
\sphinxattableend\end{savenotes}

\sphinxAtStartPar
\sphinxhref{https://www.simplify3d.com/support/materials-guide/properties-table/}{Simply3D Table} has lots of properties related to 3D printer filaments.

\sphinxAtStartPar
TODO: Young’s Modulus.
Same issues in every spectrograph

\sphinxAtStartPar
Things we did to minimize.

\sphinxAtStartPar
Binding within spectrograph. \sphinxhyphen{}\textgreater{} distortion.

\sphinxAtStartPar
Answer is do lots of cals.

\sphinxAtStartPar
\sphinxurl{https://www.simplify3d.com/support/materials-guide/properties-table/}


\chapter{Mathematics}
\label{\detokenize{mathematics:mathematics}}\label{\detokenize{mathematics::doc}}
\sphinxAtStartPar
While exploring the physics of optics for the FlexSpec1 we encountered
aspects of mathematics tailored to optics. This section is a collection
of notes that guided our intuition to close the gaps between the theory
and the images we obtained with practice.
\begin{itemize}
\item {} 
\sphinxAtStartPar
\sphinxstylestrong{“Conventions”} get in the way of the pure math.

\item {} 
\sphinxAtStartPar
\sphinxstylestrong{Beware}! Trig \sphinxstylestrong{folds} quadrants!

\item {} 
\sphinxAtStartPar
Angles are measured as positive in a counter\sphinxhyphen{}clockwise direction, and negative in a clockwise direction using the “\index{right\sphinxhyphen{}hand\sphinxhyphen{}rule@\spxentry{right\sphinxhyphen{}hand\sphinxhyphen{}rule}}right\sphinxhyphen{}hand\sphinxhyphen{}rule”.

\end{itemize}

\sphinxAtStartPar
Most calculators and the human brain operate within the social
convention of “degrees”. Computer languages, for the most part, work
with radians. High school trigonometry has left most people with a
morbid fear of radians. When radians are used in their simplest terms
they offer a powerful shortcut to intuition for
back\sphinxhyphen{}of\sphinxhyphen{}the\sphinxhyphen{}envelope (\index{BOTE@\spxentry{BOTE}}BOTE) work.

\sphinxAtStartPar
The slope\sphinxhyphen{}intercept equation from high school algebra
\index{equation@\spxentry{equation}!slope\sphinxhyphen{}intercept@\spxentry{slope\sphinxhyphen{}intercept}}equation;slope\sphinxhyphen{}intercept uses a “rise\sphinxhyphen{}over\sphinxhyphen{}run”
\index{mathematics@\spxentry{mathematics}!rise\sphinxhyphen{}over\sphinxhyphen{}run@\spxentry{rise\sphinxhyphen{}over\sphinxhyphen{}run}}mathematics;rise\sphinxhyphen{}over\sphinxhyphen{}run paradigm that amounts to the
tangent: \(sin/cos\). The sin/cos projection winds up inside the
unit circle at the \(cos(\theta)\) position, while the tangent is
outside the unit circle and goes to infinity as \(cos(\theta)
\rightarrow\;0\). The radian measure follows the unit circle and
represents an exact value. The tangent starts at 0 degrees (and
radians) and follows a line tangent to the outside of the unit circle.


\section{Radians}
\label{\detokenize{mathematics:radians}}
\sphinxAtStartPar
With a spectrograph, the separation between two wavelengths, \(\lambda\)
and \(\lambda+\Delta{\lambda}\) may be calculated by simply (ha!)
differentiating the \index{grating equation@\spxentry{grating equation}}grating equation. The goal is to state
dispersion in units of scale instead of units of wavelength:
\begin{equation}\label{equation:mathematics:ChainRule1}
\begin{split}\frac{\tt{d}\lambda}{\tt{d}x} &= \frac{\tt{d}\lambda}{\tt{d}\beta} \\
            &= \frac{\tt{d}\lambda}{\tt{d}\beta} \frac{\tt{d}\beta}{\tt{d}x}\end{split}
\end{equation}
\sphinxAtStartPar
where \(\tt{d}x\) is the distance from the grating towards the
sensor.

\sphinxAtStartPar
The farther the camera lens is from the grating, the higher the resolution
at the cost of dimmer light.


\section{Angles}
\label{\detokenize{mathematics:angles}}
\sphinxAtStartPar
Angles \index{Angles@\spxentry{Angles}!measurement@\spxentry{measurement}}Angles;measurement are measured in a plane, taken as a
counter\sphinxhyphen{}clockwise rotation around a normal vector to the plane. In the
Cartesian coordinate system/plane, values start at zero at the
\(X\) axis and increase as the rotation goes from \(X\),
through \(Y\). The angle value wraps at zero and repeats. Negative
angles are simply those that move clockwise from zero.

\sphinxAtStartPar
Trigonometric identities “fold values” in their quadrants and pose a
number of computational problems for algorithms. For example,
\(Y=sin(X) = 0.5\) occurs at both positive and negative
\(\pi/6\) or 30 degrees. Similarly \(Y=cos(X)=0.5\) occurs
at \(\pi/3\) or 60 degrees. In other words, ambiguous results
unless one accounts for circumstances. The cross product returns
a vector normal to the plane of rotation, and with a sign representing
the direction of rotation. The magnitude of the vector can be turned
into exactly one angle. The magnitude of the angle is between \(0\) and
\(pi\) BUT the sign allows for the angle to move backwards from zero!
thus \(2\pi\) \sphinxhyphen{} alpha can take on value \(\pi < \alpha < 2\pi\)
without confusion.

\sphinxAtStartPar
This angle ambiguity is critical to computing solutions in spectroscopy.

\sphinxAtStartPar
To further complicate things


\section{Small Angle Approximation}
\label{\detokenize{mathematics:small-angle-approximation}}
\sphinxAtStartPar
Kahan \sphinxcite{references:kahan-1999} recommends the use of vectors over trig for the computation
of small angles.

\sphinxAtStartPar
Estimating accuracy of the \index{small angle approximation@\spxentry{small angle approximation}}small angle approximation, taking a
McClaurian series at face value (Taylor’s Theorem around \(x=0\)),
\begin{equation}\label{equation:mathematics:SmallAngleApproximation}
\begin{split} \sin \theta &= \sum^{\infty}_{n=0} \frac{(-1)^n}{(2n+1)!} \theta^{2n+1} \\
    &= x - \frac{x^3}{3!} + \frac{x^5}{5!} - \cdots\quad\text{ for all } x\! \\\end{split}
\end{equation}
\sphinxAtStartPar
Setting the second term \(\;\theta^3\;/\;3! > 1\;/\;206264.8\)
gives an error approximation for 1 arcsecond. (There are 206264.8 arc\sphinxhyphen{}seconds in a radian.)
Looking for an error of 1 arcsecond and solving for \(\theta\):


\section{Gaussian}
\label{\detokenize{mathematics:gaussian}}
\sphinxAtStartPar
Normalized curves are used to match and measure effective width of
spectral lines. The process of producing a line is Gaussian in nature,
dominated by Maxwell\sphinxhyphen{}Boltzmann distribution but influenced by local
physics. Other curves include the Lorenzian and the Voigt profiles.
The Voigt profile is a combination of the Gaussian and lorenzian.
The math related to the Gaussian has aspects that make it attractive
for computation.

\sphinxAtStartPar
The Gaussian, and a few derivatives are:
\begin{equation}\label{equation:mathematics:Gaussian1}
\begin{split} y(x) &= a exp^\Big(  (\frac{x-\bar{x})^{2}}{\sigma}  \Big) \\
 y^{\prime}(x) &= -\frac{2a(x-\bar{x}}{\sigma}exp^\Big(  (\frac{x-\bar{x})^{2}}{\sigma}  \Big) \\
 y^{\prime\prime}(x) &= 2\frac{a}{\sigma^{2}} \left(-\sigma + 2(x-\bar{x})^{2} \right) exp^\Big(  (\frac{x-\bar{x})^{2}}{\sigma}  \Big)\end{split}
\end{equation}
\sphinxAtStartPar
where \(a\) is the maximum occurring at \(x = \bar{x}\), \(\sigma\) is
the variance.

\sphinxAtStartPar
The inflection points are minimum at \(\bar{x}\) and
maximum at \(x=\bar{x}\pm\sqrt{\sigma/2}\).

\sphinxAtStartPar
In general fitting a Gaussian to a PSF for a star, allows cosmic ray
and sensor defects to be corrected \textendash{} remembering to account for the
error associated with those corrections.

\sphinxAtStartPar
Derivative:
\begin{equation}\label{equation:mathematics:GaussianDerivative}
\begin{split}\frac{d}{dx} e^{\frac{-ax}{\sigma}}  =  -a/c e^{\frac{-ax}{\sigma}}\end{split}
\end{equation}

\section{Fourier Transform}
\label{\detokenize{mathematics:fourier-transform}}\begin{equation}\label{equation:mathematics:Fourier1}
\begin{split}f(x) &= \int_{-\infty}^{\infty} f(\xi) e^{2 \pi i} \textrm{d}\xi\\\end{split}
\end{equation}
\sphinxAtStartPar
Note:
\begin{equation}\label{equation:mathematics:Fourier2}
\begin{split}e^{2\; \pi i x} = e^{ax}\end{split}
\end{equation}
\sphinxAtStartPar
where \(a = 2 \pi i\).


\section{Parallactic Angle}
\label{\detokenize{mathematics:parallactic-angle}}
\sphinxAtStartPar
The essential aspect of compensating for the parallactic angle is
handled in FlexSpec1 by using the Inertial Measurement Unit of an
include Nano 33 BLE Sense (or IoT) processor. By choosing an axis,
we simply make it ‘balanced’ w.r.t. the Earth’s gravity vector.
A little vector mathematics tells us the minimal direction to rotate
and some book\sphinxhyphen{}keeping tells us to avoid cable wrap.


\chapter{FlexSpec 1 Guiding}
\label{\detokenize{guiding:flexspec-1-guiding}}\label{\detokenize{guiding::doc}}
\begin{figure}[htbp]
\centering
\capstart

\noindent\sphinxincludegraphics{{m29_onesecodn_astrometry_net}.jpeg}
\caption{Example 1\sphinxhyphen{}second raw guider exposure of M29, taken 2021\sphinxhyphen{}04\sphinxhyphen{}15, A. Rodda. Platesolved by nova.astrometry.net, image from ds9.}\label{\detokenize{guiding:id1}}\label{\detokenize{guiding:rawguide1}}\end{figure}

\begin{figure}[htbp]
\centering
\capstart

\noindent\sphinxincludegraphics{{ZoomDistortion1}.jpeg}
\caption{Zoom into a guider image, showing astigmatism arising from the misalignment of the slit and guider mirror.}\label{\detokenize{guiding:id2}}\end{figure}

\sphinxAtStartPar
The \index{FS1 guiding module@\spxentry{FS1 guiding module}}FS1 guiding module addresses matching wide field of view
surrounding the slit. This provides the opportunity to acquire the
target with an imprecise slew, plate\sphinxhyphen{}solve the image for precise
positioning and to increase the chances of having a brighter off\sphinxhyphen{}slit
star for alternative guiding. We applied the lens\sphinxhyphen{}makers formula to
create a compound 2\sphinxhyphen{}element “transfer lens” solution.  The equation
\begin{equation*}
\begin{split}\frac{206264.8}{\tt{fl}\times 1000} = \frac{\tt{Magic Number}}{1 \mu{m}}\end{split}
\end{equation*}
\sphinxAtStartPar
yields the “\index{Magic Number@\spxentry{Magic Number}}Magic Number” in units of arc\sphinxhyphen{}seconds per micron for a
telescope with a focal length of \({\tt{fl}}\) in units of mm. This
equation allows computing/planning the \index{slit width@\spxentry{slit width}!matching@\spxentry{matching}}slit width;matching in terms of
arc\sphinxhyphen{}seconds of seeing. Matching the slit width to the OTA magnification
and seeing is critical to acquire the best signal, limit the sky
contributions and to mitigate the effects of atmospheric refraction
(see Parallactic Angle and Rotation ref\{sec:Rotator\}.

\sphinxAtStartPar
In imaging, multiplying the number by the pixel size in microns,
yields the arc\sphinxhyphen{}seconds per pixel. Further multiplication by the size of
the chip yields the total field of view for the image. This number
should be permanently affixed to the OTA.

\index{calculation@\spxentry{calculation}!pixel size@\spxentry{pixel size}}\index{calculation@\spxentry{calculation}!slit width@\spxentry{slit width}}\ignorespaces 
\sphinxAtStartPar
The \index{lens makers equation@\spxentry{lens makers equation}}lens makers equation:

\sphinxAtStartPar
(DIAGRAM)
\begin{equation*}
\begin{split}\frac{1}{fl} = \frac{1}{s_{1}} + \frac{1}{s_{2}} \label{E:LensMakersEquation}\end{split}
\end{equation*}
\sphinxAtStartPar
is chained together for each lens in the set to match the computed
platescale on the slit to the chip size of the camera.

\sphinxAtStartPar
(EXAMPLE IMAGE).


\section{Preliminaries}
\label{\detokenize{guiding:preliminaries}}
\sphinxAtStartPar
Take a small series \index{calibration exposures@\spxentry{calibration exposures}}calibration exposures, flats, zeros and
darks for the guide camera. \index{FS1 slit carousel@\spxentry{FS1 slit carousel}}FS1 slit carousel has one position
filled to serve as a general \index{shutter@\spxentry{shutter}}shutter for dark exposures.
\begin{enumerate}
\sphinxsetlistlabels{\arabic}{enumi}{enumii}{}{.}%
\item {} 
\sphinxAtStartPar
Take zeros and darks, flats and make masters

\item {} 
\sphinxAtStartPar
Solve for defects: \((x,y)\).

\item {} 
\sphinxAtStartPar
Slew to field

\item {} 
\sphinxAtStartPar
Take a few long exposures (10sec?) to reach mag 15

\item {} 
\sphinxAtStartPar
Correct images darks/flats/defects

\item {} 
\sphinxAtStartPar
Identify main stars and platesolve.

\item {} 
\sphinxAtStartPar
Illuminate the slit from behind.

\item {} 
\sphinxAtStartPar
Solve the slit position (\(y = mx+b\))

\end{enumerate}

\sphinxAtStartPar
For the slit:
\begin{enumerate}
\sphinxsetlistlabels{\arabic}{enumi}{enumii}{}{.}%
\item {} 
\sphinxAtStartPar
The slit has rotated.

\item {} 
\sphinxAtStartPar
Position a point\sphinxhyphen{}source onto the ‘old’ best position.

\item {} 
\sphinxAtStartPar
Science image.

\item {} 
\sphinxAtStartPar
Solve the ‘Y’ offset of point source.

\item {} 
\sphinxAtStartPar
Recalculate the guide \((x,y)\) for point source to put the dispersion axis back onto original points.

\end{enumerate}

\sphinxAtStartPar
Science Camera:
\begin{enumerate}
\sphinxsetlistlabels{\arabic}{enumi}{enumii}{}{.}%
\item {} 
\sphinxAtStartPar
Take zero/dark/flats

\item {} 
\sphinxAtStartPar
Determine the defect map \((x,y)\).

\item {} 
\sphinxAtStartPar
Determine an optimal position \((y)\) for dispersion axis based on defects.

\item {} 
\sphinxAtStartPar
Record the position.

\end{enumerate}


\chapter{Remote Protocol}
\label{\detokenize{protocol:remote-protocol}}\label{\detokenize{protocol::doc}}
\sphinxAtStartPar
The control structure is being developed in a platform agnostic network
neutral fashion leveraging web applications (web apps) via bokeh and
other tools.

\sphinxAtStartPar
The bokeh server is a Raspberry Pi and is be considered to be the
‘main’ computer, a peer in an asynchronous distributed network of
peers. Any SBC/Desktop computer with a decent operating system may be
used for the server. The ODroid, BeagleBone, Intel NUC etc. Some SBC
\index{SBC@\spxentry{SBC}!Configuration@\spxentry{Configuration}}SBC;Configuration configuration will be required.

\sphinxAtStartPar
Arduino class devices are designated as an SBM is a “Single Board
Microcontroller”.

\sphinxAtStartPar
This approach sticks with the Arduino control system (boots out of the
box) and does not use an after\sphinxhyphen{}marker RTOS or PyCharm.

\sphinxAtStartPar
The essential parts:
\begin{itemize}
\item {} 
\sphinxAtStartPar
The actual hardware at the interface of the SBM and the real world,
where the if/then/else/switch/maths part is next to the
hardware. Timing may exist here.

\item {} 
\sphinxAtStartPar
The logical element within the Arduino for a part: A ‘StepperMotor’ may
have several ways to do things, there may be several different steppers
in a device. Handles the commands and is a communications terminus
or originator of status etc back to its manager.

\item {} 
\sphinxAtStartPar
The Raspberry Pi/Odroid/Linux box that owns the serial wires. Actual
wires.

\item {} \begin{description}
\item[{The user interface:}] \leavevmode\begin{itemize}
\item {} 
\sphinxAtStartPar
People via interaction a browser with buttons etc…

\item {} 
\sphinxAtStartPar
A controller program like a scheduler

\item {} 
\sphinxAtStartPar
(both at the same time)

\end{itemize}

\end{description}

\end{itemize}

\sphinxAtStartPar
One hard part is the timing required. We don’t have any real sense
of a “timer”. We have hacked up something to help, but that control
loop gets complicated with more than one set of things to do.

\sphinxAtStartPar
The other main division is in how messages flow.
\begin{itemize}
\item {} 
\sphinxAtStartPar
A message shall only be a JSON structure. The structure will be
created by a program and therefore checked before being sent. This
requirement removes all the checking requirements in the
Arduino. The JSON message will be a single string of ASCII
characters consisting divided into two classes of characters: Case
1) special \sphinxhyphen{} the braces, colon, comma and double quote mark \{\}:,”
Case 2) The characters A\sphinxhyphen{}Za\sphinxhyphen{}z0\sphinxhyphen{}9.\_ and noting else.

\end{itemize}
\begin{description}
\item[{\sphinxstylestrong{Case 1} are filtered when messages are parsed. This makes matching the close}] \leavevmode
\sphinxAtStartPar
brace to the open brace very small in the Arduino.

\item[{\sphinxstylestrong{Case 2} Means any illegal character \textendash{} that have a remote possibility}] \leavevmode
\sphinxAtStartPar
of occurring due to a bit\sphinxhyphen{}flip in coms, can act as a trigger to reset.

\item[{A JSON structure is a “dictionary” (special collection or set) of a}] \leavevmode
\sphinxAtStartPar
“key” and a “value”. A key is restricted to being only a quoted
string and a value may be one of three things: a quoted string,
ASCII integer representation, float in the form of
whole\sphinxhyphen{}dot\sphinxhyphen{}fraction expressed.

\end{description}

\sphinxAtStartPar
The Serial communications assumes more than on SBM on a wire pair.
All listen to the wire and parse a single JSON dictionary that matches
their name. The SBM parses this dictionary and dispatches any enclosed
value JSON dictionaries to each “widget”. The widget is responsible
for understanding and acting on the content of that dictionary.
A Key is considered a class of command and its values are considered
the parameters for that action.

\sphinxAtStartPar
This next part is a trick.

\sphinxAtStartPar
There will be two “mirrored” classes: Python for the Raspberry Pi
and C++ for the SBM.

\sphinxAtStartPar
The class (Python) in the Raspberry Pi has to mirror the class in the
Arduino exactly. This allows them to communicate in their own agreed
upon way to assure proper operation. This requirement alleviates
the need for a universal grammar or command set.

\sphinxAtStartPar
The ‘format’ of the message will be 1) JSON BUT 2) the content is left
up to both sides of that class.  The Python will have Bokeh/GUI
stuff. The Arduino implements all the tedious logic to make things act
and/or move. The paradigm is one of a Navy Captain and the Chief of
the Boat: The Captain knows he wants to make a certain speed to
a target and no idea how that happens. He ‘orders’ the Chief
to make 15 knots NNE. The Cheif tells the engine room to make
a certain number of ‘revolutions’.

\sphinxAtStartPar
The GUI user does not want to think in terms of stepper\sphinxhyphen{}motor
steps, directions. They want to move to a logical position.
The Arduino receives a JSON key/value pair that might say \textendash{}
in this scenario \textendash{} \{ … “speed” : “23{[}knots{]}” … \} and
parses the value as it sees fit. The units may be implicit, as
the two classes understand how that message’s value needs to
be handled.

\sphinxAtStartPar
The Raspberry Pi Python class communicates with the ‘user’
(person/algorithm). In this case a Python web\sphinxhyphen{}app built on Bokeh, or
will simply accept and pass\sphinxhyphen{}thru messages from an algorithm user. It
lives on a Raspberry Pi, and has lots of program space to do all the
logic and format checks to guarantee the messages sent to the SBM is
correct.  (Remember remove checks from the Arduino to the degree
possible).

\sphinxAtStartPar
The Raspberry Pi owns the serial lines: RS\sphinxhyphen{}232 two\sphinxhyphen{}wire, I2C, BLE
device.  This means Serial will be a server. It is provides one access
point to all classes/programs the electrical system. It will accept
and queue messages and route them by electrical protocol.

\sphinxAtStartPar
In the code repository the Slit may look like:

\sphinxAtStartPar
FS\_Slit.h
FS\_SLit.cpp
FS\_SLit.py

\sphinxAtStartPar
They will have dependencies in other classes within the architecture,
to handle the Bokeh (Python) and the messaging (Python and Arduino).


\section{Summary:}
\label{\detokenize{protocol:summary}}
\sphinxAtStartPar
The controls/sensors may be part of the Raspberry Pi itself, using a
separate RS\sphinxhyphen{}232/I2C/Wireless/Ethernet channel to the physical bits;
They may go through RS\sphinxhyphen{}232/I2C/BLE to a Arduino SBM. Tying these
together will be in Python/C++ on the Raspberry Pi. Doing the work is a
problem left somewhere else \textendash{} but required to use this communications
scheme.

\sphinxAtStartPar
Differences with ASCOM:

\sphinxAtStartPar
We do the GUI parts inside the chain, and do not rely on templates
as shims between components. This makes fast/direct coding possible.


\section{Kzin Ring Example}
\label{\detokenize{protocol:kzin-ring-example}}
\sphinxAtStartPar
The Kzin ring example:


\begin{wrapfigure}{l}{0pt}
\centering
\noindent\sphinxincludegraphics{{KzinArduinoCommand}.png}
\caption{The Kzin ring has the ability to turn on one or more of the lamps within its purview as needed. Here NeAr, Osram (Argon only), H\sphinxhyphen{}Alpha (LED), O{[}III{]} (LED), Flat (Compo of lamps) and Blue (additional LEDs) may be chosen. The yellow area is a debug mode window showing the text to be sent by serial port to the registered LED.}\label{\detokenize{protocol:id1}}\end{wrapfigure}

\sphinxAtStartPar
The nested\sphinxhyphen{}dictionary is nested inside a distribution to the “Postmaster”
that owns the spectrograph with the Kzin ring. The “Postmaster” on the
particular Arduino dispatches to an instance of a Kzin class with
the name (Kzin) the class sends the settings to to the class using
the “Process” method. The fine details of “Process” are implemented
to set the internal state and do what is necessary to conform that internal
state to the operation of that ring. In other words, if ‘osram’ had been on
at the time “Process” was called; it will be turned off. I ‘near’ was
off it is turned on \textendash{} all within the context of this one ‘command’ to
the Kzin ring.

\sphinxAtStartPar
The details of routing are not shown, as the Bokeh class only contributes
this message and that is the text we see.


\chapter{FlexSpec 1 Ecosystem}
\label{\detokenize{controls:flexspec-1-ecosystem}}\label{\detokenize{controls:fs1-ecosystem}}\label{\detokenize{controls::doc}}
\sphinxAtStartPar
Use Case: Develop a basic GUI to Arduino Test case that will be the
basis for all Arduinos.

\begin{figure}[htbp]
\centering
\capstart

\noindent\sphinxincludegraphics{{PiggyBackBllinky}.png}
\caption{PiggyBack is the name of an Arduino, on an OTA, “piggybacked” onto
the main spectrgraph OTA. The On/Off radio buttons originate
a message down to the device. It advertises text in a SandStone
coloured box, and allows a integer value to be entered for the
Pause Rate and Cycle Count for the Blinky Object Chain.}\label{\detokenize{controls:id1}}\end{figure}


\section{Control Overview}
\label{\detokenize{controls:control-overview}}
\sphinxAtStartPar
The FlexSpec 1 is designed to live with other devices under the umbrella
of a main scheduler. This requires partitioning each subsystem, and its
components in a way that information can be shared at within the instrument,
between other instruments, shared resources at a site and between other
layers within a master scheduler. The architecture of FlexSpec 1 is open.

\begin{figure}[htbp]
\centering
\capstart

\noindent\sphinxincludegraphics{{FlexSpec_Comm_Regional}.png}
\caption{Seen from the point\sphinxhyphen{}of\sphinxhyphen{}view of the Main OTA: The Raspberry PI has three means of communication: Ethernet with the world; Serial (1.8V/Voltage management) with Arduino SBMs; I2C for short haul with SMBs and stand alone sensors. Each sub\sphinxhyphen{}system is managed by its own collection of C++/Python classes.}\label{\detokenize{controls:id2}}\end{figure}

\sphinxAtStartPar
The best way to approach the control structure is to start at the distal
end of the control message chain \textendash{} the actual component being controlled.

\sphinxAtStartPar
The component is implemented using a C++ class within the arduino. There
may be a few implementations of these classes \textendash{} depending on the
architectural type of the Arduino/other device used. C++ classes
are used, as several of the same component may appear in one FlexSpec 1
instrument. They are disambiguated using a name.

\sphinxAtStartPar
Classes consist of “data+methods”. The data is always present when
included in a compiled program. However the data portion is only
created when a class is instantiated.

\begin{figure}[htbp]
\centering
\capstart

\noindent\sphinxincludegraphics{{GUI_Hardware}.png}
\caption{The Ethernet provides a potential connection to the world. The Browser in the world serves a HTML DIV for each of the instantiated devices, shown here with a darker green background. Here we have a power block “Blinky Power” \textendash{} with lots of switches; A LowRez configuration spectrograph with its slit positioner, its grating selector and a Calibration Lamp for NeAr.}\label{\detokenize{controls:id3}}\end{figure}

\sphinxAtStartPar
In the figure, each “instantiated device”. In the case of the Slit
Positioner “Slit LoRez” a motor, that is permitted to make full
rotations. It is controlled by an instance of a C++ class running
within its controlling Arduino. It is physically wired and resides on
a multi\sphinxhyphen{}drop line. The “shim” that connects this physical device to
the outside world is an instance of a Bokeh class that translates the
HTML/JSNode events from the Browser into JSON strings to be shipped to
its associated widget within one of the arduinos attached to this
system. This is a common widely used protocol \textendash{} borrowed mostly from
the Data Visualization Community. It is extremely platform independent.

\sphinxAtStartPar
You can control your system with any enabled browser, on any device,
anywhere you are allowed to go: including NASA’s Deep Space Network.

\sphinxAtStartPar
a C++ class instance running on an arduino connected by a “RS\sphinxhyphen{}232” type interface to its controlling Arduino residing on a multi\sphinxhyphen{}drop

\sphinxAtStartPar
Scenario: Consider temperature sensors.

\begin{sphinxVerbatim}[commandchars=\\\{\}]
The generic temperature sensor: FlexTemperature
Data: string Name
Methods:
  FlexTemperature() // constructor
  FlexTemperature(const string \PYGZam{}ref)
  void status(Message\PYGZlt{}x,y\PYGZgt{} \PYGZam{}response)
  static FlexTemperature *factory(const string \PYGZam{}ref)
\end{sphinxVerbatim}

\sphinxAtStartPar
In general a non\sphinxhyphen{}initialized constructor is only supported for
development. It costs nothing in terms of the code.

\sphinxAtStartPar
The construction with initialization capability of the form
FlexTemperature(const string \&ref). The string is a narrow
subset of a JSON structure. The syntax is described

\sphinxAtStartPar
This spectrograph uses several small single board computers with a serial
protocol to control:
\begin{itemize}
\item {} 
\sphinxAtStartPar
Slit selection

\item {} 
\sphinxAtStartPar
Grating selection

\item {} 
\sphinxAtStartPar
Focus of the colliminating lens

\item {} 
\sphinxAtStartPar
Enabling one or more calibration lamps
\begin{itemize}
\item {} 
\sphinxAtStartPar
High voltage lamps require a duty cycle

\end{itemize}

\end{itemize}
\begin{quote}
\begin{itemize}
\item {} 
\sphinxAtStartPar
High intensity (boosting blue) requires short cycle times

\end{itemize}
\begin{itemize}
\item {} 
\sphinxAtStartPar
Colour LEDs to support gross grating position

\end{itemize}
\end{quote}
\begin{itemize}
\item {} 
\sphinxAtStartPar
A slit back\sphinxhyphen{}illumination lamp

\item {} 
\sphinxAtStartPar
Sensors for temperature and other aspects

\item {} 
\sphinxAtStartPar
Orientation for parallactic angle measurement

\end{itemize}

\sphinxAtStartPar
The control elements essentially boil down to:
\begin{itemize}
\item {} 
\sphinxAtStartPar
Motor

\item {} 
\sphinxAtStartPar
Sensor

\item {} 
\sphinxAtStartPar
Switch

\end{itemize}


\section{Control System}
\label{\detokenize{controls:control-system}}\label{\detokenize{controls:fs1-control}}
\sphinxAtStartPar
The critical part of the control system is the device to be control.
The devices essentially fall into three main categories:
\begin{enumerate}
\sphinxsetlistlabels{\arabic}{enumi}{enumii}{}{.}%
\item {} 
\sphinxAtStartPar
Sensor

\item {} 
\sphinxAtStartPar
Switch

\item {} 
\sphinxAtStartPar
Motor

\end{enumerate}


\section{FlexJSON \textendash{} Narrow JSON Description}
\label{\detokenize{controls:flexjson-narrow-json-description}}\label{\detokenize{controls:flexjson}}
\sphinxAtStartPar
This section was created to address the use of JSON on Arduino/
C++ environments that have very limited CPUs and small memory.
The choice of the Nano BLE Sense 33 and the Xiao do not have
these restrictions.

\sphinxAtStartPar
The character set for the JSON permitted include only
Latin\sphinxhyphen{}7 characters; useful for encoding while preserving
some “in\sphinxhyphen{}band” testing.

\sphinxAtStartPar
\sphinxhref{https://en.wikipedia.org/wiki/JSON}{JSON} (Javascript Object
Notation) provides a stateless, real\sphinxhyphen{}time peer\sphinxhyphen{}peer communications
protocol. It defines the “payload” of any communications wrapper.

\sphinxAtStartPar
JSON is well modeled by a Python Dictionary. A “dictionary” consists
of a Key:Value pair.  In the JSON world, a general restriction is to
have all characters within the string to be plain text (no ‘binary’
characters.) Integers are represented by their textual representation,
floats etc.  A JSON value may be a string, integer, float, list or
another dictionary.  It is the nested nature of the dictionary that
supports FlexSpec interoperability.

\sphinxAtStartPar
The Arduino Nano 33 BLE Sense and IoT have sufficient capability
to support the ArduinoJson library. This is well crafted to be
memory efficient.


\section{Motors}
\label{\detokenize{controls:motors}}
\sphinxAtStartPar
ELEGOO 5 Sets 28BYJ\sphinxhyphen{}48 ULN2003 5V Stepper Motor + ULN2003 Driver Board Compatible with Arduino

\sphinxAtStartPar
ULN2803 Darlington driver


\chapter{Communications}
\label{\detokenize{communication:communications}}\label{\detokenize{communication::doc}}
\sphinxAtStartPar
The main scenario \index{scenario@\spxentry{scenario}!communications@\spxentry{communications}}scenario;communications is one of:
\begin{itemize}
\item {} 
\sphinxAtStartPar
An observing network of sites (observatories)

\item {} 
\sphinxAtStartPar
Each observatory has one or more buildings

\item {} 
\sphinxAtStartPar
Each building has one or more piers

\item {} 
\sphinxAtStartPar
Each pier has its elements, and one or more OTAs

\item {} 
\sphinxAtStartPar
Each OTA has lots of things. One or more of them may be Flex Spec 1 spectrographs.

\end{itemize}

\sphinxAtStartPar
This project implements its OTAs with piggy\sphinxhyphen{}back Raspberry Pi processors
on the OTA with one power cable and one Ethernet cable. This makes wire
maintenance very easy. Since the Raspberry Pi is already deployed, with
Astroberry KStars/INDI/Ekos \textendash{} there is spare room for our instrument’s
control Single Board Microcontrollers (SBMs).

\sphinxAtStartPar
In spectroscopy it is desirable to have a piggy\sphinxhyphen{}back system for simultaneous
photometry of the target.

\sphinxAtStartPar
Each OTA may have one or more Raspberry computers. Each computer may have one or
more serial connections (multi\sphinxhyphen{}drop destinations using a serial interface)
to one or more Arduino\sphinxhyphen{}class SBMs.

\sphinxAtStartPar
Remembering the mantra, “Flexible is Flexible” \textendash{} how this gets built and
orchestrated makes for a very interesting problem.

\sphinxAtStartPar
The message origination may be a GUI like interface or an intelligent
scheduler. Ideally the scheduler is inter\sphinxhyphen{}operable with other schedulers
to maximize the observing output of its network. This is the “lofty” future goal.
Now is present\sphinxhyphen{}time for architecture.

\sphinxAtStartPar
There are two paradigms to help with understanding the FlexSpec
communication:
\begin{itemize}
\item {} 
\sphinxAtStartPar
The Captain of a Ship concept: Here directions are supplied to the crew in the form of orders in common terms (rotate n degrees) and the crew translates that into a compass heading.

\item {} 
\sphinxAtStartPar
The matryoshkas \index{matryoshkas@\spxentry{matryoshkas}}matryoshkas (nested\sphinxhyphen{}doll) model: messages within messages.

\item {} 
\sphinxAtStartPar
Postal Delivery System, with a set of inter\sphinxhyphen{}related postmasters. The postmaster accepts packages or envelopes, with a return address.

\end{itemize}

\sphinxAtStartPar
The implementation is a nested set of restricted JSON dictionaries.

\sphinxAtStartPar
A dictionary is a “key” : “value” pair. The key is restricted to always be a quoted\sphinxhyphen{}string
without spaces.  The “value” may be a quoted string, an integer (no quotes), a float
(no quotes and not using exponential notation) or \textendash{} yup a dictionary.

\sphinxAtStartPar
The idea is to have a set of “matched classes” for each widget utilized. They
are matched as:


\section{Character Set}
\label{\detokenize{communication:character-set}}
\sphinxAtStartPar
The JSON communications strings are taken from a restricted ASCII
character set ANSI X3.4\sphinxhyphen{}1986 \index{communication@\spxentry{communication}!ASCII ANSI
X3.4\sphinxhyphen{}1986@\spxentry{ASCII ANSI
X3.4\sphinxhyphen{}1986}}communication;ASCII ANSI
X3.4\sphinxhyphen{}1986 with an 8\sphinxhyphen{}bit data frame, MSB defined to be zero. This
permits traditional 8\sphinxhyphen{}bit serial terminal code (PuTTY) \index{PuTTY@\spxentry{PuTTY}}PuTTY
to monitor lines, permits using out of band characters as an alarm
for corrupted serial communications, allows special characters
(ASCII control characters) to serve allow in\sphinxhyphen{}band synchronization
and special “in\sphinxhyphen{}band” control.

\sphinxAtStartPar
Here the characters \sphinxcode{\sphinxupquote{\{\}"":,}} are strictly reserved for JSON use
to promote very restricted hand\sphinxhyphen{}coded JSON capability for lesser
machines.


\chapter{Serial Communication}
\label{\detokenize{serial:serial-communication}}\label{\detokenize{serial::doc}}

\section{Raspberry Pi}
\label{\detokenize{serial:raspberry-pi}}
\begin{figure}[htbp]
\centering
\capstart

\noindent\sphinxincludegraphics[scale=0.4]{{RPi-GPIO-Pinout-Diagram-2}.png}
\caption{Reference for GPIO pins. (Raspberrypi.org)}\label{\detokenize{serial:id1}}\end{figure}

\sphinxAtStartPar
Make sure serial is enabled via the \sphinxstylestrong{raspi\sphinxhyphen{}config} file.

\begin{sphinxVerbatim}[commandchars=\\\{\}]
sudo raspi\PYGZhy{}config
Would you like a login shell to be accessible over serial? no
Would you like the serial port hardware to be enabled? YES
Save and reboot.
\end{sphinxVerbatim}

\begin{sphinxVerbatim}[commandchars=\\\{\}]
minicom \PYGZhy{}b 115200 \PYGZhy{}o \PYGZhy{}D \PYGZlt{}Port\PYGZus{}Name\PYGZgt{}
\end{sphinxVerbatim}

\sphinxAtStartPar
\textless{}Port\_Name\textgreater{} is either /dev/ttyAMA0 or /dev/serial0

\sphinxAtStartPar
Connect pin 14 and 15 (loopback) test.
minicom or similar will work.

\sphinxAtStartPar
To check what ttys are available (Astroberry example here):

\begin{sphinxVerbatim}[commandchars=\\\{\}]
dmesg \PYG{p}{|} grep enabled \PYG{p}{|} grep \PYGZhy{}o \PYGZhy{}\PYGZhy{} \PYG{l+s+s2}{\PYGZdq{}tty[0\PYGZhy{}9]\PYGZbs{}+\PYGZdq{}}
\end{sphinxVerbatim}

\sphinxAtStartPar
Get a terminal emulator for assistance with development.

\begin{sphinxVerbatim}[commandchars=\\\{\}]
sudo apt install putty
\end{sphinxVerbatim}

\sphinxAtStartPar
Run PuTTy, and the program starts a small GUI. In the case
of the Arduino IDE, use the Tools \sphinxhyphen{}\textgreater{} Serial \sphinxhyphen{}\textgreater{} Monitor
and, in one case, the port is “/dev/ttyACM” and the rate
is “115200”.

\sphinxAtStartPar
Save the session.

\sphinxAtStartPar
“Load” a named session. Then change the font FreeMono font\sphinxhyphen{}size to 14,
background (light green to sorta imply Arduino), text color to black,
cursor to black

\sphinxAtStartPar
You can change the font sizes, colors etc
via the PuTTy GUI and save with the session. Good idea to
use different color as needed.

\sphinxAtStartPar
Attach a small Arduino, running a script to print “Hello World”
every second, and you should output. This verifies a “closed
loop” solution has been achieved with PuTTy!


\begin{wrapfigure}{l}{0pt}
\centering
\noindent\sphinxincludegraphics[scale=0.5]{{Putty}.png}
\caption{Screenshot of Putty, under Ubuntu 18.04 Xfce. Note: the Category column \textendash{} where Window Color, Fonts etc may be found and changed. Note: Saved sessions offer the ability to name a session. One may achieve a pleasing appearance by selecting TitanSerialACM0\_Arduino, then Load; change the options and resave.}\label{\detokenize{serial:id2}}\end{wrapfigure}

\sphinxAtStartPar
WARNING:

\sphinxAtStartPar
COMMON VOLTAGES FOR PINS ON Small Board Computers range across
1.8, 3.3 and 5 VOLTS. MIXING SIGNALS CAN PERMANENTLY DAMAGE
HARDWARE.

\sphinxAtStartPar
FlexSpec software supports I2C and RS\sphinxhyphen{}232 and RS\sphinxhyphen{}485 as the communication
spec. I2C is a common scheme between boards located in very close
proximity. RS\sphinxhyphen{}485 is a differential signal scheme, range just over 1
kilometer. RS\sphinxhyphen{}485 wraps a 0\sphinxhyphen{}5V signal within a common\sphinxhyphen{}mode noise
voltage range of \sphinxhyphen{}7 to 12 V ( \sphinxhyphen{}7 to (0:5 +7) = 12 V). Long hauls,
multidrop capability and noise immunity are hallmarks of the RS\sphinxhyphen{}485
protocol.

\sphinxAtStartPar
RS232, ( Telecommunications Industry Association TIA\sphinxhyphen{}232\sphinxhyphen{}F)
\begin{itemize}
\item {} 
\sphinxAtStartPar
0 (space)     Asserted        +3 to +15 V    “Computer zero”

\item {} 
\sphinxAtStartPar
1 (mark)      Passive \sphinxhyphen{}15 to \sphinxhyphen{}3 V    “Computer one”

\end{itemize}

\sphinxAtStartPar
There is a “hysteresis” of 6V absolute (\sphinxhyphen{}3 to 3) between valid
signals to differentiate noise.

\sphinxAtStartPar
The maximum voltage range is from \sphinxhyphen{}25 to 25 volts.

\begin{figure}[htbp]
\centering
\capstart

\noindent\sphinxincludegraphics[scale=0.75]{{Bokeh1Hack}.png}
\caption{The Bokeh widget uses a “has\sphinxhyphen{}a” relationship to one or more FlexServers. The FlexServer may be local to same machine or remote.  The message is passed to any subclass \textendash{} here a FlexSerial class    which owns the interface to it’s local machines serial port where  Arduino/other SBMs reside.}\label{\detokenize{serial:id3}}\end{figure}


\chapter{User Software}
\label{\detokenize{software:user-software}}\label{\detokenize{software::doc}}
\sphinxAtStartPar
The overall goal of the project is to utilize free (as in money) software
that is platform agnostic (Capable of running or being made to run) on
all operating systems. One main sub\sphinxhyphen{}goal is to create an simple architecture
using basic web enabled applications, inexpensive Single Board Computers (SBCs)
(Raspberry Pi running the Astroberry suite), inexpensive Single Board
Microprocessors (SBMs). Another sub\sphinxhyphen{}goal is to leverage the best tools
for collaboration including Github, and Sphinx \textendash{} the package that
creates this documentation tied to Readthedocs.io.

\sphinxAtStartPar
The basic philosophy is to used ‘mirrored’ object\sphinxhyphen{}oriented classes written
in Python/Bokeh for managing the instructins from its users, relaying
those instructions via JSON to C/C++ classes running on the SBM. We use
the Raspberry Pi mounted to the telescope’s OTA to offer the web interface
and to direct messages to other Pi’s and arduinos within the full instrumentation
for the Pier and its OTAs.


\section{Remote Desktop}
\label{\detokenize{software:remote-desktop}}
\sphinxAtStartPar
Many use the VNC Desktop ®, TeamViewer ®, Microsoft Remote
Desktop ® and Chromeremote ® are packages we have encountered
in the wild.

\sphinxAtStartPar
KStars/INDI/Ekos:
\begin{itemize}
\item {} 
\sphinxAtStartPar
Windows

\item {} 
\sphinxAtStartPar
Linux:
\begin{itemize}
\item {} 
\sphinxAtStartPar
Desktop

\item {} 
\sphinxAtStartPar
Raspberry/PI

\item {} 
\sphinxAtStartPar
ODroid

\end{itemize}

\item {} 
\sphinxAtStartPar
Apple

\item {} 
\sphinxAtStartPar
SAOImage/ds9 (Platform Agnostic)

\item {} 
\sphinxAtStartPar
Astrometry.net (program and )

\item {} 
\sphinxAtStartPar
Sextractor (image to data conversion)

\end{itemize}


\section{Networking}
\label{\detokenize{software:networking}}
\sphinxAtStartPar
X11/R6 The X Consortium dissolved at the end of 1996, producing a
final revision, X11R6.3. Pre\sphinxhyphen{}1983, known as “W” \textendash{} move to X in
1984, settled down to the X11 release in 1986.

\sphinxAtStartPar
Wayland, X Org people (2010\sphinxhyphen{}Current), with XWayland to tie back
to legacy systems.


\section{Serial Port Management}
\label{\detokenize{software:serial-port-management}}
\sphinxAtStartPar
The chapter on Serial Communications covers details, Putty is the
package of choice as it is platform independent.


\section{End\sphinxhyphen{}user Software}
\label{\detokenize{software:end-user-software}}
\sphinxAtStartPar
TODO Add hyperlinks

\sphinxAtStartPar
The 3D design work was done with SolidWorks ®, Autodesk Fusion/360 ®,
TinkerCad ® and special modifications made with Cura ®.


\section{Computer Languages}
\label{\detokenize{software:computer-languages}}
\sphinxAtStartPar
The two languages of this project are Python and C++. The C++ is the
GNU toolchain \textendash{} very consistent across platforms. The GNU compiler
suite has excellent language libraries. Some external libraries
for the Arduino are not the best in the world \textendash{} be careful there.

\sphinxAtStartPar
The Pythonic environment of choice include Anaconda and VSCode. The
Anaconda environment is widely used in Astronomy and the VSCode widely
used on Win10 environments. Both have advantages and disadvantages.
The main thing with Python is to match the very extensive libraries
more that sweat the language details. The libraries include numpy, astropy,
scipy etc.

\sphinxAtStartPar
Other languages include “bash”, the Linux/Unix shell. Beware, bash on
Win10 IS NOT CONSISTENT. A handful of useful Linux utilities were used
to develop code \textendash{} not needed for run\sphinxhyphen{}time support.


\section{Science}
\label{\detokenize{software:science}}
\sphinxAtStartPar
Telescope and camera management:
\begin{itemize}
\item {} 
\sphinxAtStartPar
KStars/INDI/Ekos

\item {} 
\sphinxAtStartPar
Astroberry (augmented)

\end{itemize}

\sphinxAtStartPar
Reductions were performed with
\begin{itemize}
\item {} 
\sphinxAtStartPar
BASS

\item {} 
\sphinxAtStartPar
Demetra

\item {} 
\sphinxAtStartPar
ISIS

\item {} 
\sphinxAtStartPar
VSpec

\item {} 
\sphinxAtStartPar
IRAF

\item {} 
\sphinxAtStartPar
PlotSpectra

\end{itemize}

\sphinxAtStartPar
Tools include
\sphinxhyphen{} SAOImage ds9
\sphinxhyphen{} fitsverify

\sphinxAtStartPar
General tools:
\begin{itemize}
\item {} 
\sphinxAtStartPar
Anaconda Python env, including Jupyter, Spyder etc.

\end{itemize}


\section{Bokeh/Holoviz/Panel}
\label{\detokenize{software:bokeh-holoviz-panel}}
\sphinxAtStartPar
Bokeh \sphinxhref{https://www.psych.mcgill.ca/labs/mogillab/anaconda2/lib/python2.7/site-packages/bokeh/command/subcommands/serve.py}{parameters} at McGill University.

\sphinxAtStartPar
TODO: Token

\sphinxAtStartPar
TODO: Authentication

\sphinxAtStartPar
Bokeh \sphinxhref{https://docs.bokeh.org/en/latest/docs/user\_guide/server.html}{server}
instructions.

\sphinxAtStartPar
Some additional software packages need to be installed:

\begin{sphinxVerbatim}[commandchars=\\\{\},numbers=left,firstnumber=1,stepnumber=1]
apt\PYGZhy{}get update
apt\PYGZhy{}get \PYGZhy{}y install gcc
pip3 install numpy
pip3 install astropy\PYG{o}{[}all\PYG{o}{]}
pip3 install pandas
pip3 install bokeh
\end{sphinxVerbatim}

\sphinxAtStartPar
TODO: Cite the trademarks etc.


\section{Other Packages}
\label{\detokenize{software:other-packages}}
\sphinxAtStartPar
Here is a collectin or very random links to other coding projects.

\sphinxAtStartPar
Mostly in PERL
\sphinxhref{https://github.com/robertoabraham/ProjectDragonfly/tree/master/scripts}{DragonFly}


\chapter{CAD Computer Aided Design}
\label{\detokenize{cad:cad-computer-aided-design}}\label{\detokenize{cad::doc}}
\sphinxAtStartPar
The main design cycle consists of experiments, hand sketches, capture
of the sketches into one of the main CAD packages, to produce STL
files. STL files \index{STL files@\spxentry{STL files}}STL files are converted into \sphinxcode{\sphinxupquote{gcode}} using Cura \index{Cura@\spxentry{Cura}}Cura, moved to
the printer using \sphinxcode{\sphinxupquote{Octoprint}} \index{Octoprint@\spxentry{Octoprint}}Octoprint and printed. The 3D parts are
evaluated and (altered within this cycle) as needed.

\sphinxAtStartPar
Parts are easily shared across the world, via email or the GitHub
we maintained while developing the spectrograph.

\sphinxAtStartPar
Flex Spec 1 relied heavily on cad programs:
\begin{enumerate}
\sphinxsetlistlabels{\arabic}{enumi}{enumii}{}{.}%
\item {} 
\sphinxAtStartPar
SolidWorks Known to a few of the Authors from previous lives.

\item {} 
\sphinxAtStartPar
Autodesk’s Fusion/360 popular (indeed basis for LowSpec). Autodesk is not as freely accessible to the community as before.

\item {} 
\sphinxAtStartPar
TinkerCAD modify STL files, capture basic designs.

\item {} 
\sphinxAtStartPar
Cura convert STL into gcode

\item {} 
\sphinxAtStartPar
KiCAD for the electrical layout. The Power Supply Board was used.

\item {} 
\sphinxAtStartPar
Octoprint, used to readily transfer gcode to the 3D printer.

\end{enumerate}


\chapter{Electronics}
\label{\detokenize{electronics:electronics}}\label{\detokenize{electronics::doc}}
\sphinxAtStartPar
Clarke Yeager, used KiCad \sphinxcite{references:kicad-1} to create the schematic and layout for
several Printed Circuit Boards, emailed to China, made \textendash{} with silk
screen overlays and returned. The total cost was around 0.50USD per
board, shipping was more than the costs of the boards. This package
was emailed to Tony Rodda in England where he originated his own
order. Shipping time in Europe was less than for Clarke in the US.

\begin{figure}[htbp]
\centering
\capstart

\noindent\sphinxincludegraphics{{KiCadBoards1}.jpeg}
\caption{\sphinxhref{https://KiCad.org}{KiCad} boards. A completed high\sphinxhyphen{}voltage board, without the Arduino BLE Sense attached to bottom, shown over a top and bottom image for the board. To the side is a smaller board for stepper controllers.}\label{\detokenize{electronics:id2}}\end{figure}

\sphinxAtStartPar
TODO Add Schematics


\chapter{Arduino Chips}
\label{\detokenize{arduino:arduino-chips}}\label{\detokenize{arduino::doc}}
\sphinxAtStartPar
Follow the instructions:

\sphinxAtStartPar
\sphinxhref{https://wiki.seeedstudio.com/Seeeduino-XIAO/}{Seeeduino XIAO}

\sphinxAtStartPar
Arduino chips come in several flavors. Here we are using Arm \sphinxhyphen{}Core \index{Arm@\spxentry{Arm}}Arm processors.

\sphinxAtStartPar
Download the Arduino IDE (here 1.8.15)

\sphinxAtStartPar
Under \sphinxstylestrong{File \sphinxhyphen{}\textgreater{} Preferences}

\sphinxAtStartPar
Into the \sphinxstylestrong{Additional Boards Manager URLs} enter:

\sphinxAtStartPar
“\sphinxurl{https://files.seeedstudio.com/arduino/package\_seeeduino\_boards\_index.json}”

\sphinxAtStartPar
\sphinxstylestrong{Tools \sphinxhyphen{}\textgreater{} Board \sphinxhyphen{}\textgreater{} Boards Manager}

\sphinxAtStartPar
Search/select \sphinxstylestrong{Seeed SAMD Boards} (see that it includes
Seeeduino XAIO M0, and install it. (here 1.21.1) Takes a beat.

\sphinxAtStartPar
The real trick is to start with the raw IDE and add the \sphinxstylestrong{Blink}
program. This shows the ‘loop’ is closed with respect to development.


\section{Pin Outs}
\label{\detokenize{arduino:pin-outs}}
\begin{sphinxVerbatim}[commandchars=\\\{\},numbers=left,firstnumber=1,stepnumber=1]
Pin Funcion Type Description
1  D13          Digital GPIO
2  +3V3 Power Out Internally generated power output to external devices
3  AREF         Analog           Analog Reference; can be used as GPIO
4  A0/DAC0      Analog ADC in/DAC out; can be used as GPIO
5  A1           Analog ADC in; can be used as GPIO
6  A2           Analog ADC in; can be used as GPIO
7  A3           Analog ADC in; can be used as GPIO
8  A4/SDA       Analog ADC in; I2C SDA; Can be used as GPIO (*)
9  A5/SCL       Analog ADC in; I2C SCL; Can be used as GPIO(*)
10 A6           Analog ADC in; can be used as GPIO
11 A7           Analog ADC in; can be used as GPIO
12 VUSB Normally NC;
13 RST          Digital In Active low reset input (duplicate of pin 18)
14 GND Power Power Ground
15 VIN Power In Vin Power input
16 TX           Digital USART TX; can be used as GPIO
17 RX           Digital USART RX; can be used as GPIO
18 RST          Digital Active low reset input (duplicate of pin 13)
19 GND Power Power Ground
20 D2           Digital GPIO
21 D3/PWM       Digital GPIO; can be used as PWM
22 D4           Digital GPIO
23 D5/PWM       Digital GPIO; can be used as PWM
24 D6/PWM       Digital GPIO; can be used as PWM
25 D7           Digital GPIO
26 D8           Digital GPIO
27 D9/PWM       Digital GPIO; can be used as PWM
28 D10/PWM      Digital GPIO; can be used as PWM
29 D11/MOSI     Digital SPI MOSI; can be used as GPIO
30 D12/MISO     Digital SPI MISO; can be used as GPIO
\end{sphinxVerbatim}
\begin{description}
\item[{Note: (12)can be connected to VUSB pin of the USB connector by shorting}] \leavevmode
\sphinxAtStartPar
a jumper Power In/Out

\end{description}


\section{Logical}
\label{\detokenize{arduino:logical}}
\sphinxAtStartPar
Pins are “overloaded” in their ability; Use of some pins
precludes blocks of use.

\begin{sphinxVerbatim}[commandchars=\\\{\},numbers=left,firstnumber=1,stepnumber=1]
1  D13          Digital GPIO
3  AREF         Analog  GPIO  Analog Reference
4  A0/DAC0      Analog  GPIO  ADC in/DAC out
5  A1           Analog  GPIO  ADC in
6  A2           Analog  GPIO  ADC in
7  A3           Analog  GPIO  ADC in
8  A4/SDA       Analog  GPIO  ADC in; I2C SDA
9  A5/SCL       Analog  GPIO  ADC in; I2C SCL
10 A6           Analog  GPIO  ADC in
11 A7           Analog  GPIO  ADC in
20 D2           Digital GPIO
21 D3/PWM       Digital GPIO  ; can be used as PWM
22 D4           Digital GPIO
23 D5/PWM       Digital GPIO  ; can be used as PWM
24 D6/PWM       Digital GPIO  ; can be used as PWM
25 D7           Digital GPIO
26 D8           Digital GPIO
27 D9/PWM       Digital GPIO  ; can be used as PWM
28 D10/PWM      Digital GPIO  ; can be used as PWM
29 D11/MOSI     Digital GPIO  SPI MOSI
30 D12/MISO     Digital GPIO  SPI MISO


16 TX           Digital USART TX; can be used as GPIO
17 RX           Digital USART RX; can be used as GPIO


2  +3V3 Power Out Internally generated power output to external devices
12 VUSB Normally NC;
15 VIN Power In Vin Power input

14 GND Power Power Ground
19 GND Power Power Ground

13 RST          Digital In Active low reset input (duplicate of pin 18)
18 RST          Digital Active low reset input (duplicate of pin 13)
\end{sphinxVerbatim}
\begin{quote}

\sphinxAtStartPar
\sphinxurl{https://cdn.sparkfun.com/assets/0/d/8/4/9/DS-15580-Arduino\_Nano\_33\_BLE\_Sense.pdf}
\end{quote}


\chapter{Documentation}
\label{\detokenize{documentation:documentation}}\label{\detokenize{documentation::doc}}
\sphinxAtStartPar
This section holds the technical details to produce and extend
the Flex Spec 1 documentation.

\sphinxAtStartPar
It relies on the texlive docker container.

\sphinxAtStartPar
Make a subdir under the usual sphinx directory called documentation
and put the .tex files related to images etc there. Using the container
make the images. Some additional processing may be needed. Then
move the final images to doc/source/images and run the other
docker container sphinx\sphinxhyphen{}latex\sphinxhyphen{}fs1. This makes the overall
readthedocs files. The sphinx\sphinxhyphen{}latex\sphinxhyphen{}fd1 container has the regular
sphinx\sphinxhyphen{}latexpdf container augmented with astropy, numpy and other
necessary files.

\sphinxAtStartPar
Note: There is a way to fake sphinx into accepting inaccessible
modules and ignore them.

\sphinxAtStartPar
Next door to the docs dir, create a FS1Code directory and move
code copies there. This protects the files from other aspects
of the system. Judicious use of \_\_init\_\_.py files will create
the illusion of code in a way that Sphinx expects. This gets
complicated \textendash{} witness the astropy documentation approach.
Ugh.

\begin{sphinxVerbatim}[commandchars=\\\{\}]
docker run \PYGZhy{}it \PYGZhy{}\PYGZhy{}rm \PYGZhy{}v /home/\PYG{k}{\PYGZdl{}(}USER\PYG{k}{)}/git/FlexSpec1/docs/documentation:/home \PYG{l+s+se}{\PYGZbs{}}
   texlive/texlive /bin/bash

\PYG{n+nb}{cd} /home
latex ... \PYG{c+c1}{\PYGZsh{} output goes to /home/\PYGZdl{}(USER)/git/FlexSpec1/docs}
\end{sphinxVerbatim}

\sphinxAtStartPar
Then within the command prompt you can run commands.

\sphinxAtStartPar
Here the example is for making the art.

\sphinxAtStartPar
Use commands like:

\begin{sphinxVerbatim}[commandchars=\\\{\}]
\PYGZgt{} latex exa100.tex
\PYGZgt{} dvips exa100.ps
\end{sphinxVerbatim}

\sphinxAtStartPar
To produce a postscript file.

\sphinxAtStartPar
Currently, the docker texlive\sphinxhyphen{}latest has pstricks working.


\chapter{Using Docker}
\label{\detokenize{docker:using-docker}}\label{\detokenize{docker::doc}}
\sphinxAtStartPar
A docker container is a software module, loaded into memory and
executed within its own environment. It contains both the OS image %
\begin{footnote}[1]\sphinxAtStartFootnote
OS may include various Linux operating systems, some Windows released images.
%
\end{footnote},
and and executables %
\begin{footnote}[2]\sphinxAtStartFootnote
The executable must be compiled for the CPU architecture, ARM vs AMD64 for example.
%
\end{footnote} for the environment. Containers may be opened
in interesting ways to share files, environments and to interoperate
with graphics in limited ways. An excellent use of containers is
to manage toolchains (cross compilers; Latex) without any impact
on the native file system. Containers may be linked, for example
one container for PostgreSQL and a separate container for its
database file images. These two may be readily shared. Protection
allows modification to the container’s files, and unless actively
and overtly saved the changes are lost when the container shuts
down.

\sphinxAtStartPar
In the Flex Spec 1 project, a few containers are used.
\begin{enumerate}
\sphinxsetlistlabels{\arabic}{enumi}{enumii}{}{.}%
\item {} 
\sphinxAtStartPar
Sphinx Documentation \textendash{} extended a small amount for pstricks.

\item {} 
\sphinxAtStartPar
Arduino toolchain to permit a full toolchain to run easily in batch mode using make.

\end{enumerate}

\sphinxAtStartPar
To use containers:
\begin{enumerate}
\sphinxsetlistlabels{\arabic}{enumi}{enumii}{}{.}%
\item {} 
\sphinxAtStartPar
Create a \sphinxhref{https://www.docker.com/account}{Docker account}.
* Through the account one gains access to a vast number of prepared images like Ubuntu, Alpine (very small Linux OS), PostgreSQL, Arduino toolchain, Sphinx etc.

\item {} 
\sphinxAtStartPar
Download and install docker.
* The installation will tie the docker engine to your machine.

\item {} 
\sphinxAtStartPar
Use \sphinxcode{\sphinxupquote{docker pull ...}} command to fetch a container from the Dockerhub repository

\item {} 
\sphinxAtStartPar
Use a local Dockerfile to extend and modify, or

\item {} 
\sphinxAtStartPar
Run the container as a \sphinxcode{\sphinxupquote{bash}} command line; \sphinxcode{\sphinxupquote{apt install}} packages, and while the container is still running \sphinxcode{\sphinxupquote{commit}} the container.

\end{enumerate}

\sphinxAtStartPar
These rather terse basic steps are well explained in online documentation. Just be wary of any GUI requirements.


\section{Make the system}
\label{\detokenize{docker:make-the-system}}
\sphinxAtStartPar
Start working with the docker file.


\section{Docker Container Use}
\label{\detokenize{docker:docker-container-use}}
\begin{sphinxVerbatim}[commandchars=\\\{\},numbers=left,firstnumber=1,stepnumber=1]
docker run \PYGZhy{}it \PYGZhy{}\PYGZhy{}rm \PYGZhy{}v \PYG{k}{\PYGZdl{}(}\PYG{n+nb}{pwd}\PYG{k}{)}:/doc
                 \PYGZhy{}v \PYG{n+nv}{\PYGZdl{}HOME}:/root
                 \PYGZhy{}\PYGZhy{}env \PYG{n+nv}{SYSARCH}\PYG{o}{=}nano
                 arduino\PYGZus{}base:v2 bash
\end{sphinxVerbatim}


\subsection{Starting and Using Docker}
\label{\detokenize{docker:starting-and-using-docker}}
\begin{sphinxVerbatim}[commandchars=\\\{\}]
sudo systemctl start docker
\end{sphinxVerbatim}

\sphinxAtStartPar
Will start the docker services.


\section{Sphinx documentation}
\label{\detokenize{docker:sphinx-documentation}}
\begin{sphinxVerbatim}[commandchars=\\\{\},numbers=left,firstnumber=1,stepnumber=1]
docker run \PYGZhy{}it \PYGZhy{}\PYGZhy{}rm \PYGZhy{}v \PYG{l+s+si}{\PYGZdl{}\PYGZob{}}\PYG{n+nv}{GitREPO}\PYG{l+s+si}{\PYGZcb{}}/docs:/docs sphinxdoc/sphinx\PYGZhy{}latexpdf make clean latexpdf
docker run \PYGZhy{}it \PYGZhy{}\PYGZhy{}rm \PYGZhy{}v \PYG{l+s+si}{\PYGZdl{}\PYGZob{}}\PYG{n+nv}{GitREPO}\PYG{l+s+si}{\PYGZcb{}}/docs:/docs sphinxdoc/sphinx\PYGZhy{}latexpdf make clean html

Commands to rebuild the documentation.
\end{sphinxVerbatim}

\begin{sphinxVerbatim}[commandchars=\\\{\}]
\PYG{c+c1}{\PYGZsh{} Extend the sphinxdoc/sphinx\PYGZhy{}latexpdf\PYGZhy{}astropy to support Python docs}
\PYG{c+c1}{\PYGZsh{} add the gcc too \PYGZhy{}\PYGZhy{} this permits docs of c++ code.}
\PYG{c+c1}{\PYGZsh{} astropy[all] drags in its dependencies}
\PYG{c+c1}{\PYGZsh{} docker build \PYGZhy{}t sphinx\PYGZhy{}latexpdf\PYGZhy{}astropy .}
\PYG{c+c1}{\PYGZsh{} docker ps \PYGZsh{} look for the container to commit}
\PYG{c+c1}{\PYGZsh{} docker stop xxxxxxxxx}
\PYG{c+c1}{\PYGZsh{} docker commit sphinxdoc/sphinx\PYGZhy{}latexpdf sphinxdoc/sphinx\PYGZhy{}latexpdf\PYGZhy{}astropy}


\PYG{n}{FROM} \PYG{n}{sphinx}\PYG{o}{\PYGZhy{}}\PYG{n}{latexpdf}

\PYG{n}{WORKDIR} \PYG{o}{/}

\PYG{n}{RUN} \PYG{n}{apt}\PYG{o}{\PYGZhy{}}\PYG{n}{get} \PYG{n}{update}
\PYG{n}{RUN} \PYG{n}{apt}\PYG{o}{\PYGZhy{}}\PYG{n}{get} \PYG{n}{install} \PYG{n}{gcc}
\PYG{n}{RUN} \PYG{n}{pip3} \PYG{n}{install} \PYG{n}{astropy}\PYG{p}{[}\PYG{n+nb}{all}\PYG{p}{]}
\PYG{n}{RUN} \PYG{n}{pip3} \PYG{n}{install} \PYG{n}{pandas}
\PYG{n}{RUN} \PYG{n}{pip3} \PYG{n}{install} \PYG{n}{bokeh}



\end{sphinxVerbatim}


\section{Arduino Toolchain}
\label{\detokenize{docker:arduino-toolchain}}
\sphinxAtStartPar
The toolchain is a layered container you build:
\begin{itemize}
\item {} 
\sphinxAtStartPar
The Ubuntu base/Docker27 file \index{Dockerfile@\spxentry{Dockerfile}!toolchain;intermediate;base@\spxentry{toolchain;intermediate;base}}Dockerfile;toolchain;intermediate;base brings down the main Ubuntu OS

\item {} 
\sphinxAtStartPar
The intermediate/Dockerfile \index{Dockerfile@\spxentry{Dockerfile}!toolchain;intermediate@\spxentry{toolchain;intermediate}}Dockerfile;toolchain;intermediate, adds basic compiler and environmental support

\item {} 
\sphinxAtStartPar
The support/Dockerfile \index{Dockerfile@\spxentry{Dockerfile}!toolchain;support@\spxentry{toolchain;support}}Dockerfile;toolchain;support adds support: git,zip,automake,subversion etc.

\item {} 
\sphinxAtStartPar
The toolchain/Dockerfile \index{Dockerfile@\spxentry{Dockerfile}!toolchain@\spxentry{toolchain}}Dockerfile;toolchain adds the actual toolchain.

\end{itemize}

\sphinxAtStartPar
Each docker step requires a rather large amount of data transfer. Each
intermediate step, allows for mistakes and the ability to make small
corrections without a major ‘from the top’ repeat of previous downloads.

\sphinxAtStartPar
Each dockerfile has a comment about how to perform the step.

\begin{sphinxVerbatim}[commandchars=\\\{\},numbers=left,firstnumber=1,stepnumber=1]
FROM ubuntu

ARG \PYG{n+nv}{TOOLCHAIN\PYGZus{}ARCHIVE}\PYG{o}{=}https://github.com/arduino/toolchain\PYGZhy{}avr/archive/master.zip

RUN apt\PYGZhy{}get update \PYGZhy{}y \PYG{o}{\PYGZam{}\PYGZam{}} \PYG{l+s+se}{\PYGZbs{}}
    apt\PYGZhy{}get install \PYGZhy{}y \PYG{l+s+se}{\PYGZbs{}}
        build\PYGZhy{}essential \PYG{l+s+se}{\PYGZbs{}}
        gperf \PYG{l+s+se}{\PYGZbs{}}
        bison \PYG{l+s+se}{\PYGZbs{}}
        subversion \PYG{l+s+se}{\PYGZbs{}}
        texinfo \PYG{l+s+se}{\PYGZbs{}}
        zip \PYG{l+s+se}{\PYGZbs{}}
        automake \PYG{l+s+se}{\PYGZbs{}}
        flex \PYG{l+s+se}{\PYGZbs{}}
        libtinfo\PYGZhy{}dev \PYG{l+s+se}{\PYGZbs{}}
        pkg\PYGZhy{}config \PYG{l+s+se}{\PYGZbs{}}
        git \PYG{l+s+se}{\PYGZbs{}}
        wget

RUN wget \PYG{n+nv}{\PYGZdl{}TOOLCHAIN\PYGZus{}ARCHIVE} \PYGZhy{}O /tmp/toolchain\PYGZhy{}avr.zip \PYG{o}{\PYGZam{}\PYGZam{}} \PYG{l+s+se}{\PYGZbs{}}
    unzip /tmp/toolchain\PYGZhy{}avr.zip \PYG{o}{\PYGZam{}\PYGZam{}} \PYG{l+s+se}{\PYGZbs{}}
    rm /tmp/toolchain\PYGZhy{}avr.zip \PYG{o}{\PYGZam{}\PYGZam{}} \PYG{l+s+se}{\PYGZbs{}}
    mv toolchain\PYGZhy{}avr*/* . \PYG{o}{\PYGZam{}\PYGZam{}} \PYG{l+s+se}{\PYGZbs{}}
    rm \PYGZhy{}rf toolchain\PYGZhy{}avr*/ \PYG{o}{\PYGZam{}\PYGZam{}} \PYG{l+s+se}{\PYGZbs{}}
    ./tools.bash \PYG{o}{\PYGZam{}\PYGZam{}} \PYG{l+s+se}{\PYGZbs{}}
    ./binutils.build.bash \PYG{o}{\PYGZam{}\PYGZam{}} \PYG{l+s+se}{\PYGZbs{}}
    ./gcc.build.bash \PYG{o}{\PYGZam{}\PYGZam{}} \PYG{l+s+se}{\PYGZbs{}}
    ./avr\PYGZhy{}libc.build.bash \PYG{o}{\PYGZam{}\PYGZam{}} \PYG{l+s+se}{\PYGZbs{}}
    ./gdb.build.bash \PYG{o}{\PYGZam{}\PYGZam{}} \PYG{l+s+se}{\PYGZbs{}}
    ./package\PYGZhy{}avr\PYGZhy{}gcc.bash

ENV \PYG{n+nv}{PATH}\PYG{o}{=}/objdir/bin:\PYG{n+nv}{\PYGZdl{}PATH}
\end{sphinxVerbatim}

\begin{sphinxVerbatim}[commandchars=\\\{\},numbers=left,firstnumber=1,stepnumber=1]
\PYG{c+c1}{\PYGZsh{} /home/git/external/docker\PYGZhy{}arduino\PYGZhy{}toolchain/toolchain/Dockerfile}
\PYG{c+c1}{\PYGZsh{} Build on top of the container made ../base: arduino\PYGZus{}base:v1}
\PYG{c+c1}{\PYGZsh{} docker build \PYGZhy{}t arduino\PYGZus{}base:v2 .}
FROM arduino\PYGZus{}base:v1

ARG \PYG{n+nv}{TOOLCHAIN\PYGZus{}ARCHIVE}\PYG{o}{=}https://github.com/arduino/toolchain\PYGZhy{}avr/archive/master.zip

RUN apt\PYGZhy{}get install \PYGZhy{}y \PYG{l+s+se}{\PYGZbs{}}
        build\PYGZhy{}essential \PYG{l+s+se}{\PYGZbs{}}
        gperf \PYG{l+s+se}{\PYGZbs{}}
        bison \PYG{l+s+se}{\PYGZbs{}}
        subversion \PYG{l+s+se}{\PYGZbs{}}
        texinfo \PYG{l+s+se}{\PYGZbs{}}
        zip \PYG{l+s+se}{\PYGZbs{}}
        automake \PYG{l+s+se}{\PYGZbs{}}
        flex \PYG{l+s+se}{\PYGZbs{}}
        libtinfo\PYGZhy{}dev \PYG{l+s+se}{\PYGZbs{}}
        pkg\PYGZhy{}config \PYG{l+s+se}{\PYGZbs{}}
        git \PYG{l+s+se}{\PYGZbs{}}
        wget

RUN wget \PYG{n+nv}{\PYGZdl{}TOOLCHAIN\PYGZus{}ARCHIVE} \PYGZhy{}O /tmp/toolchain\PYGZhy{}avr.zip \PYG{o}{\PYGZam{}\PYGZam{}} \PYG{l+s+se}{\PYGZbs{}}
    unzip /tmp/toolchain\PYGZhy{}avr.zip \PYG{o}{\PYGZam{}\PYGZam{}} \PYG{l+s+se}{\PYGZbs{}}
    rm /tmp/toolchain\PYGZhy{}avr.zip \PYG{o}{\PYGZam{}\PYGZam{}} \PYG{l+s+se}{\PYGZbs{}}
    mv toolchain\PYGZhy{}avr*/* . \PYG{o}{\PYGZam{}\PYGZam{}} \PYG{l+s+se}{\PYGZbs{}}
    rm \PYGZhy{}rf toolchain\PYGZhy{}avr*/ \PYG{o}{\PYGZam{}\PYGZam{}} \PYG{l+s+se}{\PYGZbs{}}
    ./tools.bash \PYG{o}{\PYGZam{}\PYGZam{}} \PYG{l+s+se}{\PYGZbs{}}
    ./binutils.build.bash \PYG{o}{\PYGZam{}\PYGZam{}} \PYG{l+s+se}{\PYGZbs{}}
    ./gcc.build.bash \PYG{o}{\PYGZam{}\PYGZam{}} \PYG{l+s+se}{\PYGZbs{}}
    ./avr\PYGZhy{}libc.build.bash \PYG{o}{\PYGZam{}\PYGZam{}} \PYG{l+s+se}{\PYGZbs{}}
    ./gdb.build.bash \PYG{o}{\PYGZam{}\PYGZam{}} \PYG{l+s+se}{\PYGZbs{}}
    ./package\PYGZhy{}avr\PYGZhy{}gcc.bash

ENV \PYG{n+nv}{PATH}\PYG{o}{=}/objdir/bin:\PYG{n+nv}{\PYGZdl{}PATH}
\end{sphinxVerbatim}

\begin{sphinxVerbatim}[commandchars=\\\{\},numbers=left,firstnumber=1,stepnumber=1]
\PYG{c+c1}{\PYGZsh{} /home/git/external/docker\PYGZhy{}arduino\PYGZhy{}toolchain/toolchain/Dockerfile}
\PYG{c+c1}{\PYGZsh{} Build on top of the container made ../base: arduino\PYGZus{}base:v1}
\PYG{c+c1}{\PYGZsh{} Add the system support for non\PYGZhy{}arduino support.}
\PYG{c+c1}{\PYGZsh{} docker build \PYGZhy{}t arduino\PYGZus{}base:v2 .}
FROM arduino\PYGZus{}base:v1

ARG \PYG{n+nv}{TOOLCHAIN\PYGZus{}ARCHIVE}\PYG{o}{=}https://github.com/arduino/toolchain\PYGZhy{}avr/archive/master.zip

RUN apt\PYGZhy{}get install \PYGZhy{}y build\PYGZhy{}essential \PYG{l+s+se}{\PYGZbs{}}
                       gperf \PYG{l+s+se}{\PYGZbs{}}
                       bison \PYG{l+s+se}{\PYGZbs{}}
                       flex \PYG{l+s+se}{\PYGZbs{}}
                       automake \PYG{l+s+se}{\PYGZbs{}}
                       libtinfo\PYGZhy{}dev \PYG{l+s+se}{\PYGZbs{}}
                       pkg\PYGZhy{}config \PYG{l+s+se}{\PYGZbs{}}
                       zip \PYG{l+s+se}{\PYGZbs{}}
                       wget\PYG{l+s+se}{\PYGZbs{}}
                       git \PYG{l+s+se}{\PYGZbs{}}
                       subversion 

ENV \PYG{n+nv}{PATH}\PYG{o}{=}/objdir/bin:\PYG{n+nv}{\PYGZdl{}PATH}
\PYG{n+nb}{echo} docker run \PYGZhy{}it \PYGZhy{}\PYGZhy{}rm build \PYGZhy{}t arduino\PYGZus{}base:v2 /bin/bash
\PYG{n+nb}{echo} and apt\PYGZhy{}get install \PYGZhy{}y texinfo to answer questions.
\PYG{n+nb}{echo} \PYG{k}{then} docker commit \PYGZlt{}container\PYGZgt{} arduino\PYGZus{}base:v3
\end{sphinxVerbatim}

\begin{sphinxVerbatim}[commandchars=\\\{\},numbers=left,firstnumber=1,stepnumber=1]
FROM arduino\PYGZus{}base:v2

ARG \PYG{n+nv}{TOOLCHAIN\PYGZus{}ARCHIVE}\PYG{o}{=}https://github.com/arduino/toolchain\PYGZhy{}avr/archive/master.zip

RUN wget \PYG{n+nv}{\PYGZdl{}TOOLCHAIN\PYGZus{}ARCHIVE} \PYGZhy{}O /tmp/toolchain\PYGZhy{}avr.zip \PYG{o}{\PYGZam{}\PYGZam{}} \PYG{l+s+se}{\PYGZbs{}}
    unzip /tmp/toolchain\PYGZhy{}avr.zip \PYG{o}{\PYGZam{}\PYGZam{}} \PYG{l+s+se}{\PYGZbs{}}
    rm /tmp/toolchain\PYGZhy{}avr.zip \PYG{o}{\PYGZam{}\PYGZam{}} \PYG{l+s+se}{\PYGZbs{}}
    mv toolchain\PYGZhy{}avr*/* . \PYG{o}{\PYGZam{}\PYGZam{}} \PYG{l+s+se}{\PYGZbs{}}
    rm \PYGZhy{}rf toolchain\PYGZhy{}avr*/ \PYG{o}{\PYGZam{}\PYGZam{}} \PYG{l+s+se}{\PYGZbs{}}
    ./tools.bash \PYG{o}{\PYGZam{}\PYGZam{}} \PYG{l+s+se}{\PYGZbs{}}
    ./binutils.build.bash \PYG{o}{\PYGZam{}\PYGZam{}} \PYG{l+s+se}{\PYGZbs{}}
    ./gcc.build.bash \PYG{o}{\PYGZam{}\PYGZam{}} \PYG{l+s+se}{\PYGZbs{}}
    ./avr\PYGZhy{}libc.build.bash \PYG{o}{\PYGZam{}\PYGZam{}} \PYG{l+s+se}{\PYGZbs{}}
    ./gdb.build.bash \PYG{o}{\PYGZam{}\PYGZam{}} \PYG{l+s+se}{\PYGZbs{}}
    ./package\PYGZhy{}avr\PYGZhy{}gcc.bash

ENV \PYG{n+nv}{PATH}\PYG{o}{=}/objdir/bin:\PYG{n+nv}{\PYGZdl{}PATH}
\end{sphinxVerbatim}


\section{Issues}
\label{\detokenize{docker:issues}}
\sphinxAtStartPar
There are issues with cross\sphinxhyphen{}platform sharing of docker containers.
In this case, the sphinx\sphinxhyphen{}latexpdf\sphinxhyphen{}fs1 container is a docker image
pulled from dockerhub, and augmented with a few tools to make
tikz/pstricks work with latex. The attempt to \sphinxtitleref{docker save} the container
from Linux and \sphinxtitleref{docker load} the container in Win10 failed. Consultation
with a guru friend revealed his krew implements a “Docker in Docker”
way to build containers for export \textendash{} thus nailing the external dependencies.
The base of this issue is the vast size of a container and by using
the libraries in the “local” machine (user of the container’s machine)
the overall size of a container could be reduced. Ha! this is right
back to why we want to use containers in the first place, and it
also driving some people to the insane idea that flatpacks, snap etc
will acutaully work. \sphinxhref{http://jpetazzo.github.io/2015/09/03/do-not-use-docker-in-docker-for-ci/}{Here} is an article to that effect.


\section{Docker Basics}
\label{\detokenize{docker:docker-basics}}
\sphinxAtStartPar
The file \sphinxcode{\sphinxupquote{\$HOME/.docker/config.json}} holds your credentials, etc.
Adding \sphinxcode{\sphinxupquote{"experimental": "enabled"}} to file will enable the buildx
environment (docker 19.3/later).


\chapter{Bill of Materials}
\label{\detokenize{bom:bill-of-materials}}\label{\detokenize{bom:flexspec1-bom}}\label{\detokenize{bom::doc}}
\sphinxAtStartPar
TODO Flesh out this BOM etc.


\chapter{Calibration Lamps}
\label{\detokenize{callamps:calibration-lamps}}\label{\detokenize{callamps::doc}}
\sphinxAtStartPar
Calibration frames have long been a subject of discord within the
amateur spectroscopic community.  With the advent of CMOS cameras even
Bias and Dark calibration frames have come under scrutiny but for the
moment we focus here on wavelength calibration and “flat” lamps.

\sphinxAtStartPar
Wavelength Calibration
The former has usually been addressed by “the dangly bulb” method
of suspending a small Ar/Ne Relco bulb, or a much larger “Filly” bulb
over the end of the telescope object lens or aperture.  This, as we
have previously proven, has adverse affects on calibration accuracy
and the ability to provide repeatable calibration data.

\sphinxAtStartPar
The problem arises from the movement of calibration emission lines
caused by the relative position of the bulb itself.  This is also
described in Spectroscopy for Amateur Astronomers by Trypsteen and
Walker (Cambridge University Press).

\sphinxAtStartPar
Commercial spectrographs rely on the Relco bulb being placed within
the instrument (or a closely coupled Calibration Unit).  This does
a much better job notwithstanding the limitations of the Relco bulb
but the bulbs are generally inaccessible.

\sphinxAtStartPar
Flats
Spectroscopic “flats” have also been the source of problems.  The lack
of flux provided by Tungsten and Halogen bulbs in the blue regime
to enable pixel\sphinxhyphen{}to\sphinxhyphen{}pixel variations places a severe limitation on
wavelength calibration and target frames alike resulting in low
quality spectra in a region already troubled by loss of camera
sensitivity and optical component cut off.  The belief that
Flats should be generated by a black body source is outdated.

\sphinxAtStartPar
We have decided to provide a calibration unit that is housed immediately
in front of the FlexSpec fitted within its 2” nosepiece.  See The Kzin
Ring below. This consists of a circuit board housing a pair of Relco or
Neon Bulbs (for wavelength calibration), LED, Grains of Wheat and UV
Boost Leds used in combination to provide a Flat source enhanced in the
Blue and Red regimes.

\sphinxAtStartPar
Additional Red and Blue “Marker” LEDs provide a fallback method of
providing a visual indicator for positioning the grating.

\sphinxAtStartPar
Thus, all calibration can be carried out remotely with the Kzin.

\sphinxAtStartPar
Take reference calibration images, and maintain an archive and a PDF
of graphs of results. You may note changes, and account for
variations that will occur overtime.

\sphinxAtStartPar
Kzin \index{Kzin@\spxentry{Kzin}}Kzin ring \sphinxcite{references:niven-ellis} is the name assigned to the “ring\sphinxhyphen{}like”
in\sphinxhyphen{}line passive dispersing element equally illuminated by NeAr lamps,
LEDs, and small incandescent grain\sphinxhyphen{}of\sphinxhyphen{}wheat bulbs. The ring is shaped
in a way to reside inside the 2\sphinxhyphen{}inch nose\sphinxhyphen{}piece area outside of the
main housing of Flex Spec 1 as not to take up valuable back\sphinxhyphen{}focus area
for popular SCTs. The permanent position of the Kzin ring means
repeatability in injecting the calibration source\sphinxhyphen{}light into the
spectrograph.

\sphinxAtStartPar
It has its own Arduino \index{Kzin@\spxentry{Kzin}!Arduino@\spxentry{Arduino}}Kzin;Arduino, and is controlled by a
series of software switches tied to the main functions. Each important
lamp may be set to be on for a set interval.

\begin{figure}[htbp]
\centering
\capstart

\noindent\sphinxincludegraphics{{BlueGreenx2-1}.png}
\caption{Two LEDs listed as Blue\sphinxhyphen{}Green.}\label{\detokenize{callamps:id2}}\end{figure}

\begin{figure}[htbp]
\centering
\capstart

\noindent\sphinxincludegraphics{{OVLG-Comparison}.png}
\caption{OVLG Leds marked as white.}\label{\detokenize{callamps:id3}}\end{figure}

\begin{figure}[htbp]
\centering
\capstart

\noindent\sphinxincludegraphics{{WhiteFlashlight1}.png}
\caption{White Flashlight.}\label{\detokenize{callamps:id4}}\end{figure}

\begin{figure}[htbp]
\centering
\capstart

\noindent\sphinxincludegraphics{{660NK-Red}.png}
\caption{660nm Red LED As ‘finder’ marker for grating central wavelength selection.}\label{\detokenize{callamps:id5}}\end{figure}

\sphinxAtStartPar
A wavelength calibration frame from the FlexSpec/Kzin combination and the
RMS residual calculation from the software package ISIS (Buil) is shown below.
It can be seen that early results show very good calibration performance
of R = 1500+ and an RMS = 0.04.  These figures are better than similar
commercially available designs.

\begin{figure}[htbp]
\centering

\noindent\sphinxincludegraphics{{FS1Calibration}.png}
\end{figure}

\sphinxAtStartPar
.

\begin{figure}[htbp]
\centering

\noindent\sphinxincludegraphics{{ISIS}.png}
\end{figure}

\sphinxAtStartPar
Best Practice techniques such as always placing Reference star and
Target star on the same slit pixel should be followed in addition
to trimming all images to remove statistitcal contamination
from dead areas of the frames.  As always, an odd number of frames
should be recorded.

\sphinxAtStartPar
It is our position that binning, especially with CMOS cameras, should
always be avoided.


\chapter{Outstanding Documentation Issues}
\label{\detokenize{outstanding:outstanding-documentation-issues}}\label{\detokenize{outstanding:outstanding}}\label{\detokenize{outstanding::doc}}\begin{itemize}
\item {} 
\sphinxAtStartPar
Table of guider lenses, FOV, and guide issues with focal lengths etc.

\item {} 
\sphinxAtStartPar
Optical Ray\sphinxhyphen{}Trace of different guide lens scenarios (Greg? Wayne?)

\item {} 
\sphinxAtStartPar
Explanation of defects in the guide images: reflections etc.

\item {} 
\sphinxAtStartPar
Discuss Thermal Stability

\item {} 
\sphinxAtStartPar
Discuss Mechanical Stability

\item {} 
\sphinxAtStartPar
Remote Scenarios and Use Cases

\end{itemize}


\chapter{References}
\label{\detokenize{references:references}}\label{\detokenize{references::doc}}

\chapter{YouTube Resources}
\label{\detokenize{youtube:youtube-resources}}\label{\detokenize{youtube::doc}}
\sphinxAtStartPar
A word about YouTube: many tutorials are made my
non\sphinxhyphen{}American producers are are excellent in content.
Lately, many are just music with no words. Be patient
and shop around.

\sphinxAtStartPar
YouTube resources. Google the title, and you should
find a quick URL to the resource. Combine with terms
like Learning Resources, Tutorials, Examples, and
phrases related to your specific question. For
example “SolidWorks equation driven parabola”


\section{3D Printing}
\label{\detokenize{youtube:d-printing}}\begin{itemize}
\item {} 
\sphinxAtStartPar
3D Printing Nerd

\item {} 
\sphinxAtStartPar
CHEP Filament Friday

\item {} 
\sphinxAtStartPar
Technivorous

\item {} 
\sphinxAtStartPar
Thomas Sanladerer

\item {} 
\sphinxAtStartPar
Prusa 3D by Josef Prusa

\item {} 
\sphinxAtStartPar
Tech2C

\item {} 
\sphinxAtStartPar
Zack Freedman

\item {} 
\sphinxAtStartPar
Core Electronics 3D Printing

\item {} 
\sphinxAtStartPar
DIY BuilderDIY Builder

\end{itemize}


\section{Arduino}
\label{\detokenize{youtube:arduino}}\begin{itemize}
\item {} 
\sphinxAtStartPar
Adafruit Industries

\item {} 
\sphinxAtStartPar
Andreas Spiess

\item {} 
\sphinxAtStartPar
Core Electronics

\item {} 
\sphinxAtStartPar
Digi\sphinxhyphen{}Key

\item {} 
\sphinxAtStartPar
DIY Builder

\item {} 
\sphinxAtStartPar
Electronoobs

\item {} 
\sphinxAtStartPar
MERT Arduino \& Tech

\item {} 
\sphinxAtStartPar
Seeed Studio

\item {} 
\sphinxAtStartPar
Alexander Baran\sphinxhyphen{}Harper

\end{itemize}


\section{CAD/CAE}
\label{\detokenize{youtube:cad-cae}}\begin{itemize}
\item {} 
\sphinxAtStartPar
Hawk Ridge SolidWorks

\end{itemize}


\section{Computer Hardware}
\label{\detokenize{youtube:computer-hardware}}\begin{itemize}
\item {} 
\sphinxAtStartPar
Explaining Computers

\item {} 
\sphinxAtStartPar
Dronebot Workshop

\item {} 
\sphinxAtStartPar
The 8\sphinxhyphen{}Bit Guy

\item {} 
\sphinxAtStartPar
Ben Eater

\item {} 
\sphinxAtStartPar
Digi\sphinxhyphen{}Key

\end{itemize}


\section{Computer Languages}
\label{\detokenize{youtube:computer-languages}}\begin{itemize}
\item {} 
\sphinxAtStartPar
Tech with Tim

\item {} 
\sphinxAtStartPar
Traversy Media

\item {} 
\sphinxAtStartPar
Adafruit Industries

\item {} 
\sphinxAtStartPar
Ben Eater

\item {} 
\sphinxAtStartPar
Caleb Curry

\item {} 
\sphinxAtStartPar
Core Electronics

\item {} 
\sphinxAtStartPar
TechGumbo

\item {} 
\sphinxAtStartPar
Alexander Baran\sphinxhyphen{}Harper

\end{itemize}


\section{DIY}
\label{\detokenize{youtube:diy}}\begin{itemize}
\item {} 
\sphinxAtStartPar
Jake’s Workshop

\item {} 
\sphinxAtStartPar
Make it Extreme

\item {} 
\sphinxAtStartPar
MickMake

\item {} 
\sphinxAtStartPar
Ralph S Bacon

\item {} 
\sphinxAtStartPar
Zack Freedman

\item {} 
\sphinxAtStartPar
Andreas Spiess

\item {} 
\sphinxAtStartPar
Core Electronics

\item {} 
\sphinxAtStartPar
DIY Builder

\item {} 
\sphinxAtStartPar
Electronoobs

\item {} 
\sphinxAtStartPar
GreatScott!!

\item {} 
\sphinxAtStartPar
learnelectronics

\item {} 
\sphinxAtStartPar
MERT Arduino \& Tech

\item {} 
\sphinxAtStartPar
The Thought Emporium

\end{itemize}


\section{Electronics}
\label{\detokenize{youtube:electronics}}\begin{itemize}
\item {} 
\sphinxAtStartPar
Dronebot Workshop

\item {} 
\sphinxAtStartPar
Kevin Darrah

\item {} 
\sphinxAtStartPar
Ralph S Bacon

\item {} 
\sphinxAtStartPar
Adafruit Industries

\item {} 
\sphinxAtStartPar
Andreas Spiess

\item {} 
\sphinxAtStartPar
Core Electronics

\item {} 
\sphinxAtStartPar
Ben Eater

\item {} 
\sphinxAtStartPar
Crosstalk Solutions

\item {} 
\sphinxAtStartPar
Digi\sphinxhyphen{}Key

\item {} 
\sphinxAtStartPar
DIY Builder

\item {} 
\sphinxAtStartPar
Electronoobs

\item {} 
\sphinxAtStartPar
GreatScott!!

\item {} 
\sphinxAtStartPar
learnelectronics

\item {} 
\sphinxAtStartPar
MERT Arduino \& Tech

\item {} 
\sphinxAtStartPar
Seeed Studio

\item {} 
\sphinxAtStartPar
Alexander Baran\sphinxhyphen{}Harper

\item {} 
\sphinxAtStartPar
TheRaspberryPiGuy

\end{itemize}


\section{Math}
\label{\detokenize{youtube:math}}\begin{itemize}
\item {} 
\sphinxAtStartPar
3Blue1Brown

\item {} 
\sphinxAtStartPar
Corey Schafer

\item {} 
\sphinxAtStartPar
patrickJMT

\end{itemize}


\section{Networking}
\label{\detokenize{youtube:networking}}\begin{itemize}
\item {} 
\sphinxAtStartPar
CWNE88

\item {} 
\sphinxAtStartPar
Crosstalk Solutions

\end{itemize}


\section{Physics}
\label{\detokenize{youtube:physics}}\begin{itemize}
\item {} 
\sphinxAtStartPar
patrickJMT

\end{itemize}


\section{Raspberry Pi}
\label{\detokenize{youtube:raspberry-pi}}\begin{itemize}
\item {} 
\sphinxAtStartPar
Ralph S Bacon

\item {} 
\sphinxAtStartPar
Explaining Computers

\item {} 
\sphinxAtStartPar
Dronebot Workshop

\item {} 
\sphinxAtStartPar
Adafruit Industries

\item {} 
\sphinxAtStartPar
Andreas Spiess

\item {} 
\sphinxAtStartPar
Core Electronics

\item {} 
\sphinxAtStartPar
Crosstalk Solutions

\item {} 
\sphinxAtStartPar
Digi\sphinxhyphen{}Key

\item {} 
\sphinxAtStartPar
learnelectronics

\item {} 
\sphinxAtStartPar
Seeed Studio

\item {} 
\sphinxAtStartPar
Alexander Baran\sphinxhyphen{}Harper

\item {} 
\sphinxAtStartPar
TheRaspberryPiGuy

\end{itemize}


\chapter{Printer}
\label{\detokenize{printer:printer}}\label{\detokenize{printer::doc}}
\sphinxAtStartPar
3D Printer Calibration
Jerry Foote Oct 21, 2021

\sphinxAtStartPar
This procedure assumes that OctoPrint is attached and running. To send commands select
“Terminal” at the top of the screen and use the “Send” button to send the command to the
printer.

\sphinxAtStartPar
\# Send M503 command. Look for the M92 line and and note Current values for X, Y, Z and E.

\sphinxAtStartPar
\# Using printer controls heat up extruder temp to 190C for PLA or 245C for PETG.

\sphinxAtStartPar
\# Mark length of filament at 100mm \& 120mm from the filament guide.

\sphinxAtStartPar
\# When hot, send command: G1 E100 F100

\sphinxAtStartPar
\# After the filament is extruded measure how much was extruded. If less than 100mm note how much, if more than 100, use 120 mark and compute how much. This is “ActualE”.

\sphinxAtStartPar
\# From the M92 command get the value for “E”= CurrentE.

\sphinxAtStartPar
\# Compute the new value for “E”, NewE= (100/ActualE)* CurrentE

\sphinxAtStartPar
\# Send M92 ENewE

\sphinxAtStartPar
\# Print 20 cm cube. \sphinxurl{https://www.thingiverse.com/thing:1278865}

\sphinxAtStartPar
\# Measure X, Y, Z dimensions (X is right to left, Y is front to back and Z is verticle)

\sphinxAtStartPar
\# Compute New X = (20/ActualX) * CurrentX

\sphinxAtStartPar
\# Compute New Y = (20/ActualY) * CurrentY

\sphinxAtStartPar
\# Compute New Z = (20/ActualZ) * CurrentZ

\sphinxAtStartPar
\# Send M92 XNewX YNewY ZNewZ This command should format like the M92 command previously but with the new values.

\sphinxAtStartPar
\# Send M500 which saves these values.

\sphinxAtStartPar
\# Repeat process and fine tune values.


\chapter{MERLIN GCODE Tricks}
\label{\detokenize{merlin:merlin-gcode-tricks}}\label{\detokenize{merlin::doc}}
\sphinxAtStartPar
GCode “Gerber” code is widely used in Computer Numerical Code (CNC) machining.
Most 3D printers us Merlin’s implementation. Full documentation is elsewhere.

\sphinxAtStartPar
Here are a few practical tips:

\sphinxAtStartPar
We use \sphinxhref{https://octoprint.org/}{Octoprint} to interface the printer
to the recommended Raspberry Pi 3B, 3B+ or 4B computer. This is a
35USD simple solution, web based and easy to use. This is all open\sphinxhyphen{}source
code.

\sphinxAtStartPar
The basics of GCode include a “letter” command and parameters.
Comments are enclosed in parenthesis (this is a comment).

\sphinxAtStartPar
OctoPrint and slicer programs allow users to create \sphinxstyleemphasis{scripts} to prepend
or append to the produced gcodes. Octoprint allows additinal scripts
(or the only ones) to be stored within the Octoprint computer. This
allows fine\sphinxhyphen{}tuning of gcode output to fit genaral or specific (one time)
needs.

\sphinxAtStartPar
Hand gcode scripts include one to make a small side print prior to
starting (test feed, bed adhesion); to raise/move the print head/bed
after the print \textendash{} etc.

\sphinxAtStartPar
M503 \sphinxhyphen{} dump the salient printer internal values. A screen dump from
OctoPrint via the Terminal tab:

\begin{sphinxVerbatim}[commandchars=\\\{\}]
\PYG{n}{Send}\PYG{p}{:} \PYG{n}{M503}
\PYG{n}{Recv}\PYG{p}{:} \PYG{n}{echo}\PYG{p}{:}\PYG{n}{Steps} \PYG{n}{per} \PYG{n}{unit}\PYG{p}{:}
\PYG{n}{Recv}\PYG{p}{:} \PYG{n}{echo}\PYG{p}{:}  \PYG{n}{M92} \PYG{n}{X81}\PYG{o}{.}\PYG{l+m+mi}{00} \PYG{n}{Y81}\PYG{o}{.}\PYG{l+m+mi}{00} \PYG{n}{Z400}\PYG{o}{.}\PYG{l+m+mi}{50} \PYG{n}{E94}\PYG{o}{.}\PYG{l+m+mi}{30}
\PYG{n}{Recv}\PYG{p}{:} \PYG{n}{echo}\PYG{p}{:}\PYG{n}{Maximum} \PYG{n}{feedrates} \PYG{p}{(}\PYG{n}{mm}\PYG{o}{/}\PYG{n}{s}\PYG{p}{)}\PYG{p}{:}
\PYG{n}{Recv}\PYG{p}{:} \PYG{n}{echo}\PYG{p}{:}  \PYG{n}{M203} \PYG{n}{X450}\PYG{o}{.}\PYG{l+m+mi}{00} \PYG{n}{Y450}\PYG{o}{.}\PYG{l+m+mi}{00} \PYG{n}{Z5}\PYG{o}{.}\PYG{l+m+mi}{00} \PYG{n}{E25}\PYG{o}{.}\PYG{l+m+mi}{00}
\PYG{n}{Recv}\PYG{p}{:} \PYG{n}{echo}\PYG{p}{:}\PYG{n}{Maximum} \PYG{n}{Acceleration} \PYG{p}{(}\PYG{n}{mm}\PYG{o}{/}\PYG{n}{s2}\PYG{p}{)}\PYG{p}{:}
\PYG{n}{Recv}\PYG{p}{:} \PYG{n}{echo}\PYG{p}{:}  \PYG{n}{M201} \PYG{n}{X3000} \PYG{n}{Y3000} \PYG{n}{Z100} \PYG{n}{E3000}
\PYG{n}{Recv}\PYG{p}{:} \PYG{n}{echo}\PYG{p}{:}\PYG{n}{Acceleration}\PYG{p}{:} \PYG{n}{S}\PYG{o}{=}\PYG{n}{acceleration}\PYG{p}{,} \PYG{n}{T}\PYG{o}{=}\PYG{n}{retract} \PYG{n}{acceleration}
\PYG{n}{Recv}\PYG{p}{:} \PYG{n}{echo}\PYG{p}{:}  \PYG{n}{M204} \PYG{n}{S1000}\PYG{o}{.}\PYG{l+m+mi}{00} \PYG{n}{T800}\PYG{o}{.}\PYG{l+m+mi}{00}
\PYG{n}{Recv}\PYG{p}{:} \PYG{n}{echo}\PYG{p}{:}\PYG{n}{Advanced} \PYG{n}{variables}\PYG{p}{:} \PYG{n}{S}\PYG{o}{=}\PYG{n}{Min} \PYG{n}{feedrate} \PYG{p}{(}\PYG{n}{mm}\PYG{o}{/}\PYG{n}{s}\PYG{p}{)}\PYG{p}{,} \PYG{n}{T}\PYG{o}{=}\PYG{n}{Min} \PYG{n}{travel} \PYG{n}{feedrate} \PYG{p}{(}\PYG{n}{mm}\PYG{o}{/}\PYG{n}{s}\PYG{p}{)}\PYG{p}{,} \PYG{n}{B}\PYG{o}{=}\PYG{n}{minimum} \PYG{n}{segment} \PYG{n}{time} \PYG{p}{(}\PYG{n}{ms}\PYG{p}{)}\PYG{p}{,} \PYG{n}{X}\PYG{o}{=}\PYG{n}{maximum} \PYG{n}{XY} \PYG{n}{jerk} \PYG{p}{(}\PYG{n}{mm}\PYG{o}{/}\PYG{n}{s}\PYG{p}{)}\PYG{p}{,}  \PYG{n}{Z}\PYG{o}{=}\PYG{n}{maximum} \PYG{n}{Z} \PYG{n}{jerk} \PYG{p}{(}\PYG{n}{mm}\PYG{o}{/}\PYG{n}{s}\PYG{p}{)}\PYG{p}{,}  \PYG{n}{E}\PYG{o}{=}\PYG{n}{maximum} \PYG{n}{E} \PYG{n}{jerk} \PYG{p}{(}\PYG{n}{mm}\PYG{o}{/}\PYG{n}{s}\PYG{p}{)}
\PYG{n}{Recv}\PYG{p}{:} \PYG{n}{echo}\PYG{p}{:}  \PYG{n}{M205} \PYG{n}{S0}\PYG{o}{.}\PYG{l+m+mi}{00} \PYG{n}{T0}\PYG{o}{.}\PYG{l+m+mi}{00} \PYG{n}{B20000} \PYG{n}{X10}\PYG{o}{.}\PYG{l+m+mi}{00} \PYG{n}{Z0}\PYG{o}{.}\PYG{l+m+mi}{40} \PYG{n}{E1}\PYG{o}{.}\PYG{l+m+mi}{00}
\PYG{n}{Recv}\PYG{p}{:} \PYG{n}{echo}\PYG{p}{:}\PYG{n}{Home} \PYG{n}{offset} \PYG{p}{(}\PYG{n}{mm}\PYG{p}{)}\PYG{p}{:}
\PYG{n}{Recv}\PYG{p}{:} \PYG{n}{echo}\PYG{p}{:}  \PYG{n}{M206} \PYG{n}{X0}\PYG{o}{.}\PYG{l+m+mi}{00} \PYG{n}{Y0}\PYG{o}{.}\PYG{l+m+mi}{00} \PYG{n}{Z0}\PYG{o}{.}\PYG{l+m+mi}{00}
\PYG{n}{Recv}\PYG{p}{:} \PYG{n}{echo}\PYG{p}{:}\PYG{n}{PID} \PYG{n}{settings}\PYG{p}{:}
\PYG{n}{Recv}\PYG{p}{:} \PYG{n}{echo}\PYG{p}{:}   \PYG{n}{M301} \PYG{n}{P33}\PYG{o}{.}\PYG{l+m+mi}{41} \PYG{n}{I1}\PYG{o}{.}\PYG{l+m+mi}{47} \PYG{n}{D189}\PYG{o}{.}\PYG{l+m+mi}{27}
\end{sphinxVerbatim}


\section{Initial Tests}
\label{\detokenize{merlin:initial-tests}}
\sphinxAtStartPar
The initial testing should include printing a known good piece, and fine
tuning the M92 codes. To do this download a block like:

\sphinxAtStartPar
\sphinxhref{https://www.youtube.com/watch?v=UUelLZvDelU}{CHEP}


\chapter{Indices and tables}
\label{\detokenize{index:indices-and-tables}}\begin{itemize}
\item {} 
\sphinxAtStartPar
\DUrole{xref,std,std-ref}{genindex}

\item {} 
\sphinxAtStartPar
\DUrole{xref,std,std-ref}{modindex}

\item {} 
\sphinxAtStartPar
\DUrole{xref,std,std-ref}{search}

\end{itemize}

\begin{sphinxthebibliography}{Polyansk}
\bibitem[Polyanskiy\sphinxhyphen{}2021]{references:polyanskiy-2021}\begin{enumerate}
\sphinxsetlistlabels{\Alph}{enumi}{enumii}{}{.}%
\setcounter{enumi}{12}
\item {} \begin{enumerate}
\sphinxsetlistlabels{\Alph}{enumii}{enumiii}{}{.}%
\setcounter{enumii}{13}
\item {} 
\sphinxAtStartPar
Polyanskiy, “Refractive index database,” \sphinxurl{https://refractiveindex.info}.

\end{enumerate}

\end{enumerate}
\bibitem[Kahan\sphinxhyphen{}1999]{references:kahan-1999}
\sphinxAtStartPar
Kahan, W. (1999) “Miscalculating area and angles of a needle\sphinxhyphen{}like triangle”, \sphinxurl{https://people.eecs.berkeley.edu/~wkahan/Triangle.pdf} .
\bibitem[Niven\sphinxhyphen{}Ellis]{references:niven-ellis}\begin{enumerate}
\sphinxsetlistlabels{\Alph}{enumi}{enumii}{}{.}%
\setcounter{enumi}{11}
\item {} 
\sphinxAtStartPar
Niven,  D. Ellis, “Ringworld”, A Del Rey book SF classic, \sphinxurl{https://books.google.com/books?id=uv4vqKYsyawC}

\end{enumerate}
\bibitem[Kicad\sphinxhyphen{}1]{references:kicad-1}
\sphinxAtStartPar
Stambaugh, Wayne (2018\sphinxhyphen{}02\sphinxhyphen{}03) {[}2018{]}. “KiCad Version 5 New Feature Demo”. Archived from the original on 2018\sphinxhyphen{}07\sphinxhyphen{}29. \sphinxurl{https://web.archive.org/web/20180729032002/https://fosdem.org/2018/schedule/event/cad\_kicad\_v5/}
\end{sphinxthebibliography}



\renewcommand{\indexname}{Index}
\printindex
\end{document}