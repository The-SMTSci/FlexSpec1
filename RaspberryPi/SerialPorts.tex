\section{RPi Serial Ports} \label{sec:RPiSerialPorts}

The Raspberry Pi has 6 serial ports, available in various ways, including
over BlueTooth connections. The ports use GPIO pins, and may be disabled
to allow secondary functions for the GPIO pins to be permitted.

\begin{table}[h!]
\centering
\begin{tabular}{| l | l | l | l | l | l | l |}
%\MakeShortVerb{\|}
%\multicolumn{n}{fmt}{text for merged cols}
\hline
Name   &  Type       &  Models     &  Enabled    &  GPIO (TX)  &  GPIO (RX)  &  CTS/RTS    \\
\hline
UART0  &  PL011      &  All        &  Yes        &  14,32,36   &  15,33,37   &      \\
UART1  &  mini-UART  &  All        &  No         &  14,32,40   &  15,33,41   &      \\
UART2  &  PL011      &  Pi4 Only   &  No         &         0   &         3   &   2,3    \\
UART3  &  PL011      &  Pi4 Only   &  No         &         4   &         7   &   6,7    \\
UART4  &  PL011      &  Pi4 Only   &  No         &         8   &        11   &   10,11    \\
UART5  &  PL011      &  Pi4 Only   &  No         &        12   &        15   &   14,15    \\
%% ones-based: \cline{a-b}
\hline
%%\DeleteShortVerb{|}
\end{tabular}
%%\end{minipage}    %% for footnotes  r@{.}l
\caption[WAN Firewall]{Stellarmate WAN Firewall Ports}
\label{table:StellarmateWANFirewallPorts}
%%} % end small etc
\end{table}

\subsection{Connecting FlexSpec1 via Serial Ports}

The Serial Ports for the RPi are 3.3V and require a level shifter to be
on the safeside. The FlexSpec1 design calls for a 5-wire cable between
the RPi and the Arduino SBC. These include:

\begin{table}[h!]
\centering
\begin{tabular}{| l | l |}
\hline
RPi      &  RPi                         \\
+5 V     &  Power                       \\ 
GND      &  Signal/Electrical Ground    \\ 
Tx       &  Commands                    \\ 
Rx       &  Response                    \\ 
Reset    &  Reboot FlexSpec1            \\ 
%% ones-based: \cline{a-b}
\hline
\end{tabular}
\caption[RPi/Ardiuno Cable]{The Cable has a footprint on the FlexSpec
PCBA. Pins are shown for programming clarity. The pins on the Arduino
are lables and NOT traditional pin numbers. Be careful.}
\label{table:RPi/ArdiunoCable}
\end{table}


