\newpage
\section{WAN Firewall}

Never open a large range of ports. Stellarmate uses ports 5624-8624,
that includes (Internet Relay Chat RFC-1469) at 6667.


\begin{table}[h!]
\centering
\begin{tabular}{| l | l | l |}
\hline
Service           & Port            &  WAN/Internet Control        \\ 
\hline
SSH               & 5624            & Required for Flexspec        \\ 
INDI Web Manager  & 8624            & Not required                 \\ 
INDI Server       & 7624            & Not required                 \\ 
EkosLive Server   & 3000            & Not required                 \\ 
Web VNC           & 6080            & Not required                 \\ 
VNC               & 5900            & Not required                 \\ 
Serial            & /dev/ttyS<n>    & Arduino Serial1 pins on RPi  \\ 
Serial            & /dev/ttyACM0    & Arduino Serial USB           \\ 
\hline
\end{tabular}
\caption[Stellarmate Wan Ports]{Stellarmate Wan Firewall Requirements}
\end{table}


\subsection{Handy Network Commands }

The original program to manage and display information about networks
was called \dhl{ifconfig} for ``interface configuration''. This has
been replaced recently with the newer \dhl{ip} to ``show / manipulate
routing, network devices, interfaces and tunnels``. 

\begin{table}[h!]
\centering
\begingroup % \fontsize{6pt}{6pt}
\selectfont
\begin{tabular}{| l | l |}
\hline
command       & action \\
\hline
arp           & address resolution protocol                                       \\
dig           & DNS lookup                                                        \\
netstat       & Print network connections, routing tables, interface statistics,  \\
              & masquerade connections, and multicast memberships                 \\
nslookup      & lookup DNS information                                            \\
ping          & see if path to machine exists                                     \\
tcpdump       & view traffic on local machine                                     \\
tracepath     & tracepath to a remote machine: eg: \dhl{tracepath google.com}     \\
\hline
\end{tabular}
\endgroup
\caption[Handy Net Tools]{A few of many tools for network tracing.}
\label{table:handynetcommands}
\end{table}

\newpage
\subsection{ip vs ifconfig}
The mapping from old ifconfig commands to new ip commands.

\begin{table}[h!]
\centering
\begingroup % \fontsize{6pt}{6pt}
\selectfont
\begin{tabular}{| l | l |}
\hline
ifconfig command & New ip command                                                           \\
\hline
arp -a                              & ip neigh                                              \\ 
arp -v                              & ip -s neigh                                           \\ 
arp -s 192.168.1.1 1:2:3:4:5:6      & ip neigh add 192.168.1.1 lladdr 1:2:3:4:5:6 dev eth1  \\ 
arp -i eth1 -d 192.168.1.1          & ip neigh del 192.168.1.1 dev eth1                     \\ 
ifconfig -a                         & ip addr                                               \\ 
ifconfig eth0 down                  & ip link set eth0 down                                 \\ 
ifconfig eth0 up                    & ip link set eth0 up                                   \\ 
ifconfig eth0 192.168.1.1           & ip addr add 192.168.1.1/24 dev eth0                   \\ 
ifconfig eth0 netmask 255.255.255.0 & ip addr add 192.168.1.1/24 dev eth0                   \\ 
ifconfig eth0 mtu 9000 i            & p link set eth0 mtu 9000                              \\ 
ifconfig eth0:0 192.168.1.2         & ip addr add 192.168.1.2/24 dev eth0                   \\ 
iptables                            & show and manipulate IPTABLEs \dhl{iptables -S}        \\
netstat                             & ss                                                    \\ 
netstat -neopa                      & ss -neopa                                             \\ 
netstat -g                          & ip maddr                                              \\ 
route                               & ip route                                              \\ 
\hline
\end{tabular}
\endgroup
\caption[Net-tools vs IProute2]{A ifconfig vs ip command quick summary.}
\label{table:ifconfigvsipcommand}
\end{table}

\subsection{Modem Configuration}

Each model model will use a different scheme to enable Network Address
Translation (NAT) to connect the port from the WAN side to a
``server'' within the LAN. Information usually includes:

Start Port \\
End   Port \\
Server IP4 Address \\
Server IP6 Address \\

