\documentclass[letter,11pt,oneside]{article}
%%% HEREHEREHERE
%%% APPENDIX
%%% ADDAPPENDIX
%%% (insert (format "\n%s\n" (buffer-file-name)))
%%% (occur "\\(\\\\[a-z]*section\\|appendix\\|input\\|\\<include\\>\\)")

%%\documentclass[11pt,twocolumn]{article}
%%\usepackage[inline]{asymptote}       %% Inline asymptote diagrams
%%\usepackage{wglatex}                 %% Use this one and kill others.
\usepackage{color}                     %% colored letters {\color{red}{{text}}
\usepackage{fancyhdr}                  %% headers/footers
\usepackage{fancyvrb}                  %% headers/footers
\usepackage{datetime}                  %% pick up tex date time 
\usepackage{lastpage}                  %% support page of ...lastpage
\usepackage{times}                     %% native times roman fonts
\usepackage{textcomp}                  %% trademark
\usepackage{amssymb,amsmath}           %% greek alphabet
\usepackage{parskip}                   %% blank lines between paragraphs, no indent
\usepackage{shortvrb}                  %% short verb use for tables
\usepackage{lscape}                    %% landscape for tables.
\usepackage{longtable}                 %% permit tables to span pages wg-longtable
\usepackage{multicol}                  %% Enhance footnotes/endnotes
\usepackage{url}                       %% Make URLs uniform and links in PDFs
\usepackage{enumerate}                 %% Allow letters/decorations for enumerations
\usepackage{endnotes}                  %% Enhance footnotes/endnotes
\usepackage{listings}                  %% Make URLs uniform and links in PDFs
\pdfadjustspacing=1                    %% force LaTeX-like character spacing
%%\usepackage{geometry}                  %% allow margins to be relaxed
%%\usepackage{wrapfig}                 %% permit wrapping figures.
%%\usepackage{subfigure}               %% images side by side.
%%\geometry{margin=1in}                  %% Allow narrower margins etc.
\usepackage[T1]{fontenc}               %% Better Verbatim Font.
\renewcommand*\ttdefault{txtt}        %% 
\usepackage[colorlinks=true]{hyperref} %% Make huperlinks within a PDF
\usepackage{natbib}                    %% bibitems
\usepackage{upquote}                   %% make programmer's quoetes in verbatim sections.

%% include background image (wg-document-page-background) 

\usepackage{graphicx}            %% Include pictures into a document
%% (wg-texdoc-inserttikz)


\def\documentisdraft{NOTDRAFT}

%% (wg-texdoc-isdraft)
%% (wg-texdoc-insert-fancy-headers)

%\usepackage[bookmarks]{hyperref} %% Make huperlinks within a PDF
\usepackage{makeidx}             %% Make an index uncomment following line
\makeindex                       %%.. yeah this one, too. index{key} in text




\definecolor{verbcolor}{rgb}{0.6,0,0}
\definecolor{darkgreen}{rgb}{0,0.4,0}
\newcommand\debate[1]{\textcolor{darkgreen}{DEBATE: #1} \marginpar{\textcolor{red}{DEBATE} }}
\newcommand{\ltodo}[2]{\marginpar{\textcolor{red}{ACTION: #1}\endnote{#2}}}
\renewcommand{\thefigure}{\thesection-\arabic{figure}}
\newcommand{\menu}{\ensuremath{\;\rightarrow\;}}
\newcommand{\dhl}[1]{{\color{verbcolor}{\texttt#1}}}
\definecolor{wglightgreen}{rgb}{0.88, 0.58, 0.88}
\newcommand{\wgtextbox}[1]{\noindent\fcolorbox{darkgreen}{wglightgreen}{%
    \minipage[t]{\dimexpr0.80\linewidth-2\fboxsep-2\fboxrule\relax}
        {#1}
    \endminipage}}

\usepackage[listings,breakable]{tcolorbox}

% \snippet{Caption}{file with body}
\newcommand\snippet[2]{%
\begin{figure}
\begin{tcolorbox} [breakable,colback=yellow!99!black!20]
\begingroup \fontsize{10pt}{10pt}
\selectfont
\VerbatimInput{snippets/#2}    % INPUT THIS FILE
\endgroup
\end{tcolorbox}
\caption{#1}
\label{fig:KStarsPreliminary}
\end{figure}
}

\newcommand\inlinesnippet[1]{%
\begin{figure}
\begin{tcolorbox} [breakable,colback=yellow!99!black!20]
\begingroup \fontsize{10pt}{10pt}
\selectfont
\begingroup \fontsize{10pt}{10pt}
\selectfont
\begin{Verbatim} [commandchars=\\\{\}]
#1
\end{Verbatim}
\endgroup

\endgroup
\end{tcolorbox}
\caption{Install KStars and other handy programs.}
\label{fig:KStarsPreliminary}
\end{figure}
}


%%(wg-add-inline-images)  %% add inline images to the mix




%%Begin User Definitions: Hint: ~/.latex.defs and  latex.defs  
%%End User Definitions:

%% (wg-texdoc-adjust-paper-width)
%% (wg-texdoc-insert-hypersetup)
%% (wg-latex-tablet-page)
%%%%%%%%%%%%%%%%%%%%%%%%%%%%%%%%%%%%%%%%%%%%%%%%%%%%%%%%%%%%%%%%%%%%%%%%%%%%%
%%%%%%%%%%%%%%%%%%%%%%%%%%%      PAGE SIZE      %%%%%%%%%%%%%%%%%%%%%%%%%%%%%
%%%%%%%%%%%%%%%%%%%%%%%%%%%%%%%%%%%%%%%%%%%%%%%%%%%%%%%%%%%%%%%%%%%%%%%%%%%%%
%%%%%%%%%%%%%%%%%%%%%%% comment usepackage geometry above %%%%%%%%%%%%%%%%%%%
% \usepackage{fancyhdr}            %% headers/footers
% remove references to package geometry
\pagestyle{fancy}
\usepackage[paperheight=7.125in,paperwidth=9.5in,footskip=.05in,margin=.75in,heightrounded]{geometry}

% (iv (setq tmp (/ (* 3.0 9.5) 4.0 )))   7.125
\fancyhf{}
%\cfoot{{\tiny Page \thepage \hspace{1pt}}}




%%%%%%%%%%%%%%%%%%%%%%%%%%%%%%%%%%%%%%%%%%%%%%%%%%%%%%%%%%%%%%%%%%%%%%%%%%%%%


\begin{document}


%% (wg-latex-pretty-title-page)
%%%%%%%%%%%%%%%%%%%%%%%%%%%% START PRETTY TITLE PAGE %%%%%%%%%%%%%%%%%%%%%%%%%%%%%%%%%

\vskip 1.5cm



 {\Huge FlexBerry}


\begin{figure}[h!]
\centering
\includegraphics[width=.65\textwidth]{images/FlexBerry.png}\\
\hskip 4cm {\tiny AI art via nightcafe.studio for ``Spectrograph Telegraph Wonder''}
\end{figure}


%%%%%%%%%%%%%%%%%%%%%%%%%%%% END PRETTY TITLE PAGE %%%%%%%%%%%%%%%%%%%%%%%%%%%%%%%%%

%% (wg-texdoc-titleblock)
\pagenumbering{gobble}
\newpage
\section*{Preamble}

The cover art was generated at \url{https://images.nightcafe.studio/}
using the keywords ``Spectrograph Telegraph Wonder''. It was further
processed in a way to reduce toner use with Gimp. Presented here is
the AI Generated first cover art I've ever used for my notes!

\newpage

\title{FlexSpec1 Raspberry Pi Startup Guide}
\author{FS1 Team}
\date{\today}
\maketitle

\begin{abstract}
Acquire the latest image of Ubuntu 22.04 from Canonical for the
Raspberry Pi.  Burn the ISO image onto a high quality SD Card.  Start
the system and 'take the quiz': language, keyboard, geographic
location, user name and password, machine name etc.

Follow the \dhl{FollowMe.sh} script to do a full installation.

In this case, the user is \dhl{fred}, the machine runs DHCP and has
the name of \dhl{pier15}. See Section \ref{sec:Namesplaces}

\end{abstract}

%%%%%%%%%%%%%%%%%%%%%%%%%%%%%%%%%%%%%%%%%%%%%%%%%%%%%%%%%%%%%%%%%%%%%%%%%%%%%
%% table of contents
%%%%%%%%%%%%%%%%%%%%%%%%%%%%%%%%%%%%%%%%%%%%%%%%%%%%%%%%%%%%%%%%%%%%%%%%%%%%%
\pagenumbering{gobble}
\clearpage
\pagenumbering{roman}   % i,ii,etc
%%\pagenumbering{gobble}   %ignore page numbers for a while
\pdfbookmark[0]{Table of Contents}{MyTOC} % if usepackage{hyperref} in use.
\tableofcontents
\listoffigures
\listoftables
\newpage


\setcounter{section}{1}
\pagenumbering{arabic}

\ifx\documentisdraft\drafttest
\linenumbers    %%%%%%%%%%%%% DRAFT
\fi
\cfoot{{\tiny Page \thepage \hspace{1pt}}}

\section*{Overview}
\addcontentsline{toc}{section}{Overview}
\setcounter{page}{1}

This FlexBerry implementation uses Ubuntu 22.04 and NOT the Raspbian
image based on Ubuntu programs. The prescription for making this
system is contained in the GitHub repo TheSMTSci/FlexSpec1 repo. This
must be installed on the FlexBerry early on during its creation. This
document steps you through the install. See Section \ref{sec:RaspberryPi}

This section builds the first working image, copy this image to an
iso file and make duplicate SD cards for use through out the system.
A small script to adjust each macnine's host name and IP address is
run for second/subsequent Raspberry Pis in the system.

The main script is situated at \dhl{\$HOME/git/FlexSpec1/code/HOME/FollowMe.sh}
in the git repository.

The basic steps:
\vspace{-.15cm}
\begin{enumerate}\addtolength{\itemsep}{-0.5\baselineskip}
%\setcounter{enumi}{N}
   \item   Get the Raspberry Imager if you don't have it.
   \item   Dwonload the Ubuntu 22.04 image.
   \item   Make the SD image.
   \item   Insert the SD into a Raspberry Pi 4b with attached Monitor/keyboard/mouse.
  This is needed to do a few very early steps.
   \item   Powerup/Boot the Raspberry Pi.
   \item   Take the usual Ubuntu quiz:\\
        username and password; machine name; country details etc.
   \item   The machine will reboot.
   \item   On the FlexBerry, open a terminal and start opening the new system up.
   \item   Download the rasp-config materials.
   \item   Run rasp-config and enable the serial port.
   \item   Now you get to reboot.
\end{enumerate}

Reboot with \dhl{sudo restart} command.

\snippet{Initial Commands. Bring install to current released content.}{01.txt}

\snippet{Add System User, get the FlexSpec1 git repository, run the script.}{02.txt}


The machine will be 'headless' -- meaning no keyboard/mouse/terminal
on the telescope. In order to communicate with the device, you will want
to start a ``ssh terminal session'' from the PC. This is standard with
Linux desktops, but needs additional packages for Win10/11 machines.

One of the best is PuTTY.

On your Windows machine, go to the Microsoft store and acquire PuTTY. Microsoft
perports this to be the safest way to install things on your machine.

\subsection{Cast of Characters}

Review the Scenario in Section \ref{Scenario} below.

For the local machine on a local lan,  assume:

\vspace{-.15cm}
\begin{enumerate}\addtolength{\itemsep}{-0.5\baselineskip}
%\setcounter{enumi}{N}
   \item   the Raspberry Pi is called \dhl{pier15}
   \item   We assume the default user's name is fred. (Note this is unique!)
   \item   It is found at \dhl{http://pier15.local}
   \item   It may be accessed with ssh: \\
  \dhl{ssh fred@pier15.local}
   \item   In addition to your login on the machine, we install a special user
  called \dhl{flex}. 
\vspace{-.15cm}
\begin{enumerate}\addtolength{\itemsep}{-0.5\baselineskip}
%\setcounter{enumi}{N}
   \item   The \dhl{flex} user holds all of our materials in its home directory.
   \item   These are critical to operation of FlexBerry.
\end{enumerate}
\end{enumerate}

Set Ubuntu's opinion of the  username for this machine. Here we use \dhl{pier15}.
Add the Country locale information, keyboard type, and we recommend
you think about boot without login.

At the console you want to add a few initial packages. The goal
is to use the \dhl{git} facility to download a copy of
\url{https://github.com/The-SMTSci/FlexSpec1} repository with
the comprehensive script of all the packages to load and configurations
to be made.


\section{The FlexBerry in action} \label{sec:FlexBerryInAction}

The FlexBerry has two main services running: the bokeh server for the GIU,
and a flexdispatch socket service to attach to the Arduino. These are
started and maintained by the OS and should not require any thought.
The FlexBerry runs the \dhl{nginx} web-server. This listens on port
80 -- per usual -- and offers access to documentation etc on each of
the FlexBerry Raspberry Pi machines. 




A special user \dhl{flex} will be added with a default password.

\vspace{-.15cm}
\begin{enumerate}\addtolength{\itemsep}{-0.5\baselineskip}
%\setcounter{enumi}{N}
   \item  Acquire the latest image of Ubuntu 22.04 from Canonical for the
Raspberry Pi.  

   \item  Burn the ISO image onto a high quality SD Card.

   \item  Insert the SD Card into the Pi and power on.

   \item  Start the system and 'take the quiz': language, keyboard, geographic
location, user name and password, machine name etc.
\end{enumerate}

\subsection{The Quiz} \label(sec:TheQuiz)

\dhl{The Quiz} consists of all the details a fresh Ubuntu system
needs to know. In particular is the user name and initial password.


A special user has been added: "flex". We recommend you keep this
user, as it retains some key "user level" data.

%%% HEREHEREHERE


\section{Scenario} \label{sec:Scenario}

The system is designed to run standalone without a router. It is
also designed to be part of a world-wide collaboration of like minded
sites; complete site with its domain name, subnets into observatories,
each observatory with its own subnet for one or more piers, each
pier with one or more OTAs; each OTA with one or more payloads.

The site's domain name:  example.com, with one observatory with
one telescope called \dhl{pier15}.

This  may be accessed at pier15.example.com in our scenario.

To make this happen DNS needs to be added to one Raspberry Pi within
the pier15 subnet. Instructions for this are included.

\section{Nginx Install}

We will add https to Nginx. This requires adding to \dhl{nginx.conf}

Nginx lives at \dhl{/etc/nginx}.

The pages for nginx lives at \dhl{/var/www/html} deep.

The HTML manual for the FlexSpec1 is at \dhl{http://pier15.local/flexhelp}.

The Arduino GUI is at \dhl{http://pier15.local/flexspec}. 


\subsection{Nginx Administration} \label{sec:NginxAdministration}

The \dhl{/etc/nginx/sites-available} has all the config information.

The \dhl{/etc/nginx/sites-enabled}   has the subset of files we're using.

Please don't mess around with this area unless you understand it, and need to.



\begingroup \fontsize{10pt}{10pt}
\selectfont
%%\begin{Verbatim} [commandchars=\\\{\}]
\begin{verbatim} 
TBD
\end{verbatim}
\endgroup
%% \end{Verbatim}



%%% APPENDIX
\appendix
\renewcommand \thesection{\Alph{section}}
\section{Names and Places} \label{sec:Namesplaces}

The design is one of using a network-enabled device on one of many
\dhl{piers} in an \dhl{observatory} at a \dhl{site}. There may be
one or more sites within a collaboration. One or more \dhl{collaborations}
within a \dhl{federation}\footnote{OK Too much late night Star Trek\copyright}.)

The observatory's Domain name will be referred to herein as the venerable
\url{https://example.com}. (Try it!) 

We assume multiple piers, each pier with one or more OTAs, each OTA with
one or more instruments under control of one or more Raspberry Pi's or
other hardware.

\section{Components of FlexBerry}

The \dhl{nginx} web engine is installed to provide the gateway to
the Bokeh code to communicate with the FlexSpec1's Arduino for motors,
etc. It also serves as the gateway to the interactive blog for observing
sessions. You may add additional pages to the \dhl{/var/www/html} section.

A bind9 DNS package is installed, and is used to provide subnet names
for the net behind the FlexBerry. There only should be one per subnet.
The bind9 package may be simple but it is often manages multi-site
corporate networks. We keep it simple.

In additon to nominal things installed as part of an Ubuntu 20.04 system,
we add the \dhl{supervisord} and \dhl{gunicorn} daemons to manage
flask/boheh web apps. This complexity permits access from outside the 
network.

\ltodo{Other Packages}{Explain other packages and subsytems.}


Details for the Raspberry Pi.

\newpage
\section{Raspberry Pi}

Details related to the Raspberry Pi.


\subsection{SSH on Raspberry Pi}

The StellarMate image has all necessary ports open and ready for use.
The firewall has been set to pass all relevant ports.

See subsection \ref{sec:StellarMate} below.

\subsection{Non-StellarMate Raspberry Pi's}

Note: this section is complicated by the details needed to open
home firewalls to the internet using the local Telco's interface
routers. Each router is different, and the Telco changes its mind
frequently.

This command will list the full name for hosts that are on the network.
Here we're looking for our machine \dhl{pier15} that will appear
as \dhl{pier15.hsd1.co.comcast.net}:

{\color{verbcolor}{\verb={sudo nmap -sP 10.1.10.0/24 | awk -e '/^Nmap/ {print $5;}'}=}}

If ssh issues a:

\dhl{ssh: connect to host <machine>.local port 22: Connection refused}

use the keyboard/mouse for the raspbian machine and run the command
\dhl{sudo raspi-config}, then under Interface options, \dhl{I2 SSH} \llbox{{TAB}}, enable.


SSH (Secure SHell), originates on port 22. The port needs to be allowed
through the Raspbian's ``uncomplicated firewall'' \dhl{ufw} task. Root
permissions are required.

\begingroup \fontsize{10pt}{10pt}
\selectfont
%%\begin{Verbatim} [commandchars=\\\{\}]
\begin{verbatim} 
sudo ufw list       # see all the current ports
sudo ufw allow ssh  # allow the port to work.
\end{verbatim}
\endgroup
%% \end{Verbatim}

\subsection{Raspberry Pi -- Raspbian SSH}

Enable SSH through the Raspberry Pi Configuration menu:
\dhl{Preferences $\rightarrow$ Raspberry Pi Configuration} Click on ``Interfaces'':
and select \dhl{Enabled} next to SSH.

Use the command \dhl{sudo nano /etc/ssh/sshd\_config} find
and X11Forwarding set the value to yes 

\dhl{ X11Forwarding yes}.

Save.

\dhl{sudo systemctl restart ssh}

% https://techsphinx.com/raspberry-Pi/enable-x11-forwarding-on-raspberry-Pi/

\subsection{Serial Ports}

At end of file:

\begingroup \fontsize{10pt}{10pt}
\selectfont
%%\begin{Verbatim} [commandchars=\\\{\}]
\begin{verbatim} 
dtoverlay=disable-bt
dtoverlay=uart1
dtoverlay=uart2
droverlay=uart3
dtoverlay=uart4
dtoverlay=uart5
\end{verbatim}
\endgroup
%% \end{Verbatim}

Pinouts:

\begingroup \fontsize{10pt}{10pt}
\selectfont
%%\begin{Verbatim} [commandchars=\\\{\}]
\begin{verbatim} 
0/1    14/15   Same Uart
2       0/1
3       4/5
4       8/9
5      12/13
\end{verbatim}
\endgroup
%% \end{Verbatim}


To disable the Bluetooth 

\dhl{sudo systemctl disable hciuart}

\dhl{/boot/cmdline.txt:}

add \dhl{console=ttyS0,115200} to the main line.

\newpage
\subsection{Raspberry External Pin Functions}

\begingroup \fontsize{10pt}{10pt}
\selectfont
%%\begin{Verbatim} [commandchars=\\\{\}]
\begin{verbatim} 
  ALT0       ALT1       ALT2       ALT3       ALT4       ALT5      
 0 SDA0       SA5        PCLK       SPI3_CE0_N TXD2       SDA6      
 1 SCL0       SA4        DE         SPI3_MISO  RXD2       SCL6      
 2 SDA1       SA3        LCD_VSYNC  SPI3_MOSI  CTS2       SDA3      
 3 SCL1       SA2        LCD_HSYNC  SPI3_SCLK  RTS2       SCL3      
 4 GPCLK0     SA1        DPI_D0     SPI4_CE0_N TXD3       SDA3      
 5 GPCLK1     SAO        DPI_D1     SPI4_MISO  RXD3       SCL3      
 6 GPCLK2     SOE_N      DPI_D2     SPI4_MOSI  CTS3       SDA4      
 7 SPI0_CE1_N SWE_N      DPI_D3     SPI4_SCLK  RTS3       SCL4      
 8 SPI0_CE0_N SDO        DPI_D4     _          TXD4       SDA4      
 9 SPI0_MISO  SD1        DPI_D5     _          RXD4       SCL4      
10 SPI0_MOSI  SD2        DPI_D6     _          CTS4       SDA5      
11 SPI0_SCLK  SD3        DPI_D7     _          RTS4       SCL5      
12 PWM0       SD4        DPI_D8     SPI5_CE0_N TXD5       SDA5      
13 PWM1       SD5        DPI_D9     SPI5_MISO  RXD5       SCL5      
14 TXD0       SD6        DPI_D10    SPI5_MOSI  CTS5       TXD1      
15 RXD0       SD7        DPI_D11    SPI5_SCLK  RTS5       RXD1      
16 FL0        SD8        DPI_D12    CTS0       SPI1_CE2_N CTS1      
17 FL1        SD9        DPI_D13    RTS0       SPI1_CE1_N RTS1      
18 PCM_CLK    SD10       DPI_D14    SPI6_CE0_N SPI1_CE0_N PWM0      
19 PCM_FS     SD11       DPI_D15    SPI6_MISO  SPI1_MISO  PWM1      
20 PCM_DIN    SD12       DPI_D16    SPI6_MOSI  SPIl_MOSI  GPCLK0    
21 PCM_DOUT   SD13       DPI_D17    SPI6_SCLK  SPI1_SCLK  GPCLK1    
22 SD0_CLK    SD14       DPI_D18    SD1_CLK    ARM_TRST   SDA6      
23 SD0_XMD    SD15       DPI_D19    SD1_CMD    ARM_RTCK   SCL6      
24 SD0_DATO   SD16       DPI_D20    SD1_DAT0   ARM_TDO    SPI3_CE1_N
25 SD0_DAT1   SD17       DPI_D21    SD1_DAT1   ARM_TCK    SPI4_CE1_N
26 SD0_DAT2   TE0        DPI_D22    SD1_DAT2   ARM_TDI    SPI5_CE1_N
27 SD0_DAT3   TE1        DPI_D23    SD1_DAT3   ARM_TMS    SPI6_CE1_N
\end{verbatim}
\endgroup
%% \end{Verbatim}



\newpage
\section{Certificates of Authority}

You can use a commercial Certifying Agency to manage \dhl{Certificates of Authority}
for you site/subnets or self-sign certificates. Here we play with self-signed certificates.
\index{Certificates of Authority}.

%\snippet{Self Signed Certificates}{ca1.txt}


\section{CertMe.sh}   \label{sec:CertMe}

This section covers issuing a 'self-signed-certificate' for nginx.
It is designed for local network use. We will enable ssh certificate
login AND keep PasswordAuthentication. PasswordAuthentication leaves
the machine open to brute force attacks -- we'll live with that for
now.

In order to ssh into the FlexBerry, adding a certificate mechanism
to the user's directory is not a bad or difficult thing to do.
Here are the steps. \index{Certificates of Authority!ssh}.

The files:

\begingroup \fontsize{10pt}{10pt}
\selectfont
%%\begin{Verbatim} [commandchars=\\\{\}]
\begin{verbatim} 
cd ~/.ssh
ssh-keygen
ssh-copy-id  wayne@pier15
\end{verbatim}
\endgroup
%% \end{Verbatim}


\snippet{Nginx Files}{certme1.txt}

%%% \begingroup \fontsize{10pt}{10pt}
%%% \selectfont
%%% %%\begin{Verbatim} [commandchars=\\\{\}]
%%% \begin{verbatim} 
%%% certs/
%%%    # nginx-selfsigned.crt dhparam.pem"
%%% openssl.cnf
%%% private/
%%%    # nginx-selfsigned.key  
%%% \end{verbatim}
%%% \endgroup
%%% %% \end{Verbatim}


\snippet{nginx sample file}{certme2.txt}

\section{Filesharing}

The FlexSpec, running on a Raspberry Pi, takes science images and may
store them locally in the Pi filesystem. It may also copy these files
to remote machine elsewhere. To work with file processing/viewing utilities
one has two choices:
\vspace{-.15cm}
\begin{enumerate}\addtolength{\itemsep}{-0.5\baselineskip}
%\setcounter{enumi}{N}
   \item   Move the file to the Remote machine, and use sofware there.
   \item   Install the package on the RPi
\end{enumerate}

Of the two, doint all processing/viewing on the remote machine makes the most
sense. This shifts the CPU load away from the instrumentation, and allows
programs native to a mix of operating systems to be used.

\subsection{NFS}



\begingroup \fontsize{10pt}{10pt}
\selectfont
%%\begin{Verbatim} [commandchars=\\\{\}]
\begin{verbatim} 
sudo apt install nfs-common
/mnt/share 10.0.0.0/24(rw,sync,no_subtree_check)
/export/flex 192.168.0.0/24(rw,async,no_subtree_check,anonuid=1000,anongid=1000)
sudo exportfs -a
sudo systemctl restart nfs-kernel-server
sudo ufw allow from 10.0.2.15/24 to any port nfs
sudo ufw enable
# sudo ufw status # look for 2049
\end{verbatim}
\endgroup
%% \end{Verbatim}



/subsection{SMB Windows Filesharing}

A SMB server is added to promote pushing image files to remote
servers during observing.

\begingroup \fontsize{10pt}{10pt}
\selectfont
%%\begin{Verbatim} [commandchars=\\\{\}]
\begin{verbatim} 
sudo apt install -y samba samba-tools smbclient cifs-utils
sudo systemctl enable --now smbd              # retister for all reboots
sudo ufw allow samba
sudo usermod -aG sambashare flex
sudo smbpasswd -a "flex%time has come"
sudo mkdir -p /samba/{$USER,flex}             # make shares for the two main users
sudo chgrp -R sambashare /samba
sudo usermod -aG sambashare $USER             # add these users to group
sudo usermod -aG sambashare flex
#smb://winhost/shared-folder-name
# TODO mod /etc/samba/smb.conf
sudo systemctl restart smbd
sudo systemctl restart nmbd
\end{verbatim}
\endgroup
%% \end{Verbatim}


Gnome's file manager has built in SMB support.

\url{https://wiki.samba.org/index.php/User_and_Group_management} has decent
examples.

/etc/samba/smb.conf:

   server role = standalone server
interfaces = 127.0.0.0/8 eth0
bind interfaces only = yes


[flex]
    path = /samba/flex
    browseable = no
    read only = no
    force create mode = 0660
    force directory mode = 2770
    valid users = flex @sadmin

\subsection{Flex Nginx Configuration}

This is a rambling collection of notes about the Nginx
install. This is initial setup:

\begingroup \fontsize{10pt}{10pt}
\selectfont
%%\begin{Verbatim} [commandchars=\\\{\}]
\begin{verbatim} 
FILE 50-mod-http-geoip2.conf
load_module modules/ngx_http_geoip2_module.so;
FILE 50-mod-http-image-filter.conf
load_module modules/ngx_http_image_filter_module.so;
FILE 50-mod-http-xslt-filter.conf
load_module modules/ngx_http_xslt_filter_module.so;
FILE 50-mod-mail.conf
load_module modules/ngx_mail_module.so;
FILE 50-mod-stream.conf
load_module modules/ngx_stream_module.so;
FILE 70-mod-stream-geoip2.conf
load_module modules/ngx_stream_geoip2_module.so;
\end{verbatim}
\endgroup
%% \end{Verbatim}

\subsection{DNS Details}

This is where it gets sticky.

access-control-list

trust-anchors in bind.keys

Block these zones from routing out.

\begingroup \fontsize{10pt}{10pt}
\selectfont
%%\begin{Verbatim} [commandchars=\\\{\}]
\begin{verbatim} 
/etc/bind:
    bind.keys  -- trust anchors.
    db.0       -- keep these routes very local
    db.127
    db.255
    
    db.empty   -- only on the machine
    db.local
    named.conf - simply include the named.conf.xxxx files
       named.conf.default-zones
       named.conf.local
       named.conf.options
    rndc.key      -secret key chmod c640 chown bind:bind
    zones.rfc1918 - the list of socalled non-routable IP domains

    db-{site}-local
    db-{site}-reverse
\end{verbatim}
\endgroup
%% \end{Verbatim}

Files we need to configure:




Different machines call the ethernet interface by different
names. Usually OS vendor dependent. You will find \dhl{eno1} and
\dhl{eth0}. Currently Ubuntu 22.04 installs with \dhl{eth0}.

This document is on a Comcast Business SOHO network, so I've left the
IP addresses where they are. Comcast uses the 10 net. Not the best
idea.  Most homes use the 192.168/13 net. The /13 refers to \dhl{CIDR
  (Classless Inter-Domain Routing)
  notation}\footnote{\url{https://en.wikipedia.org/wiki/Classless_Inter-Domain_Routing}}.

Scenario: The home router connects the WAN (telco) to a LAN (local area network).
Most home accounts do not have a fixed IP address, they are assigned from a pool
of static IP addresses assigned by ICANN to the telco. Some use an internal
network address that randomly ties to a pool of external (WAN) addresses -- making
penetrating a home network from the outside difficult. Telco's 'rent' these blocks
to you for a fee. You can hit external websites that will bounce back your IP
address -- and you can 'borrow' that for a while.
When Google was asked ``whats my ip address'' it gets
back \dhl{2603:300b:509:500:dc35:2027:1f78:a426} -- this is an IP6 address.
Uuf da. The IP4 address is 71.237.89.11. Lets see how that fares over the day.

Thus we have to setup DNS with both addresses.




FlexSpec uses the nginx engine with an external ``Certificate of Authority''
(obtained from a CA provider). The nginx web server challenges the CA with
its credentials and is able to establish a SSH connection with the remote
caller. All login in steps are handled by nginx at the final address.




In the FlexSpec1 scenario, there are one or many  pi's per OTA/Mount; one or many 
mounts per observatory; one or many observatories per site and one or more sites
per collaboration; and one or more collaborations per federation. Or just a observer
in his warm room.

To facilitate all of this we assume the Site is called
example.com\footnote{Remember, this one does not route and we use it,
  well, as an example.}. The observatory, obs1 is a \dhl{subnet} of
example.com and may be reached at the \dhl{obs1.example.com} domain
name. In this the person far away just enters that name, example.com
is undergoes a DNS lookup within the real DNS system, and a query is
made of example.com to see if it can resolve a name, and if it can
resolve the \dhl{obs1.example.com} name.  If, so packets are sent
to the main machine, which in turn passes the packets on to the
obs1 router.

Lets have pier15 with two OTAs, one with a flexspec and the other
performing \dhl{piggyback} duty. let there be two Pi's:  myflex and mypiggy
machines there. Now the domain name will be
mypiggy.pier15.example.com.

Each dot (.) is a router! Each one needs to know the 'names' of the
individual machines under the router's control.

This is were \dhl{bind9} comes into play.

The FlexSpec1/Code/ directory has a script for each Pi to manage itself,
and help with each router as we can.

\subsection{ARPA}

This is a ``backronym'', where the Advanced Research Project Agency (ARPA)
originated ARPANET the early precursor-model for the modern Internet, it
now simply means \dhl{Address and Routing Parameter Area}, where top level
machines manage/delegate the addressing management. The critical idea
of this is the single, and unmentioned ``.'' that is the ``top of the Internet''.
This lone dot appears at the very end of every named address, for example
\dhl{example.com.} (note the dot after com). This is critical, as it must
append in bind (other) DNS management files.

\begingroup \fontsize{10pt}{10pt}
\selectfont
\begin{verbatim} 


lsb_release -a  # get a short description of the machine 'properties'
ip r
# default via 10.1.10.1 dev eno1 proto dhcp metric 100 
# 10.1.10.0/24 dev eno1 proto kernel scope link src 10.1.10.232 metric 100 
# 169.254.0.0/16 dev docker0 scope link metric 1000 linkdown 
# 172.17.0.0/16 dev docker0 proto kernel scope link src 172.17.0.1 linkdown 


echo $(ip r | gawk -- '/kernel/ {printf("%s", $1);}')
export mylocalnet=$(ip r | gawk -- '/kernel/ {printf("%s", $1);}')

forward/reverse lookup zones
SOA and A record forward
SOA and A reverse 

Edit /etc/named.conf
forwarders {
  192.168.1.254;   # private (us)
  8.8.8.8;         # public here google
};

sudo systemctl restart bind9
sudo systemctl status bind9


edit named.conf:

zone "class.local" {
  type master;
  file "/etc/bind/db.class.local";
};

sudo cp db.local db.class.local  # our domain local

edit db.class.local

%%%%%%%%%%%%%%%%%%%%%%%%%%%%%%%%%%%%%%%%%%%%%%%%%%%%%%%%%%%%%%%%%%%%%%%%%%%%%
$TTL   604800
@          IN       SOA     class.local. root.class.local. (
                    2          ; serial  
                     604800    ; refresh
                     86400     ; Retry
                     2419200   ; Expire
                     604800 )  ; Netative Cache TTL
;
@    IN           NS     flex.class.local.
@    IN           A      10.1.10.70
@    IN           AAAA   ::1
flex              10.1.10.70
%%%%%%%%%%%%%%%%%%%%%%%%%%%%%%%%%%%%%%%%%%%%%%%%%%%%%%%%%%%%%%%%%%%%%%%%%%%%%
sudo systemctl restart bind9
sudo systemctl status bind9    # check for typos/syntax errors

edit named.conf.local insert lookup zone

//  add to end
// reverse lookup zone
zone "1.254.10.in-addr.arpa"  {
    type       master;
    file       "/etc/bind.db.10";
};

sudo systemctl restart bind9
sudo systemctl status bind9    # check for typos/syntax errors

edit db.10 file



; BIND reverse data file for local 10.01.10.xx network

$TTL   604800
@          IN       SOA     flex.class.local. root.class.local. (
                    2          ; serial  
                     604800    ; refresh
                     86400     ; Retry
                     2419200   ; Expire
                     604800 )  ; Netative Cache TTL
;
@    IN           NS     flex.
10   IN           PTR    flex.class.local

sudo systemctl restart bind9
sudo systemctl status bind9    # check for typos/syntax errors

edit resolv.conf and add log






nameserver 10.01.10.70   
options eth0 trust-ad
search  class.local


sudo systemctl restart bind9
sudo systemctl status bind9    # check for typos/syntax errors


ping 10.01.10.70
dig -x                 # look at us
dig du.edu             # lool at something far away
ping pier15.class.local

\end{verbatim}
\endgroup
%% \end{Verbatim}


\subsection{Notes}



\begingroup \fontsize{10pt}{10pt}
\selectfont
%%\begin{Verbatim} [commandchars=\\\{\}]
\begin{verbatim} 

ip r | gawk --  '/proto/ {print $1;}'

lsb_release -a
No LSB modules are available.
Distributor ID:	Ubuntu
Description:	Ubuntu 22.04.1 LTS
Release:	22.04
Codename:	jammy
\end{verbatim}
\endgroup
%% \end{Verbatim}

\begingroup \fontsize{10pt}{10pt}
\selectfont
%%\begin{Verbatim} [commandchars=\\\{\}]
\begin{verbatim} 
ip r
default via 10.1.10.1 dev eth0 proto dhcp metric 100 
10.1.10.0/24 dev eth0 proto kernel scope link src 10.1.10.70 metric 100 
169.254.0.0/16 dev eth0 scope link metric 1000 
\end{verbatim}
\endgroup
%% \end{Verbatim}



\url{https://www.youtube.com/watch?v=MPDXVkwUehs}


\subsection{DNS Background}\section{Domain Name Server}

Up-front things:

Certain classes of internet addresses will not route.\footnote{\url{https://en.wikipedia.org/wiki/IP_address}}.

\begingroup \fontsize{8pt}{8pt}
\selectfont
%%\begin{Verbatim} [commandchars=\\\{\}]
\begin{verbatim} 
10./24         ( Range: 10.0.0.0    : 10.255.255.255 )  instrument net
169.254/16     ( Range: 169.254.0.0 : 169.254.255.255)  private network
172.16.0.0/12  ( Range: 172.16.0.0  : 172.31.255.255 )  lan routable
192.168.0.0/16 ( Range: 192.168.0.0 : 192.168.255.255)  lan routable
\end{verbatim}
\endgroup
%% \end{Verbatim}

The ``Domain Name System''. originating in 1984 \cite{BIND_Manual1},
developed under DARPA funds at Berkeley as \dhl{Berkeley Internet Name
  Domain (BIND)}, serves as ``the'' top level domain name manager. The
ICANN\footnote{Internet Corporation for Assigned Names and Numbers}
DNS System uses a large number of secondary caches to resolve the
human readable \dhl{domain name} into a
network-routable\footnote{Routable protocol allows packets to be
sa  forwarded from one network to another} (hardware) numeric
\dhl{Internet Protocol Address}.  The addresses are either IP4, and
lately IP6. Lookups worldwide take on the order of 10s to 100s of
milliseconds -- event down the a public-facing IP address running on a
FlexSpec telescope.

To serve such a wide number of domain names, (est. 341.7 million) in 2022
it is very obvious that the ``Domain'' needs to be divided into ``zones'' each
divided into sub-zones. An example:  \dhl{pier.observatory.example.com} can
be used to reference a sub-net for an observatory at <example.com>; that has 
smaller subdomain called <pier>. The main router at example.com manages ``zones
within it''. Here, perhaps one of several buildings -- or ``observatory''
in this scenario. The router within each ``observatory'' (or the main
site's router) has further zones -- here <pier> in this scenario. World
governing infrastructure manages these ``Top Level Domains'' (TLDs). 
The traffic into the reference machine at the end of that IP address,
has a (extended) nameserver associated with it. The site administrators maintain
that extended nameserver. It is presented with the fill name, parses that
name, resolves what it can. That resolution may result in a further 
request of a deeper-nested nameserver for more information. Eventually
the local machine will return a name, perhaps with a different port,
whereby a running program at the other end of this mess is engaged.


%%%ADDAPPENDIX

%% use a bibitem approach to the references publications etc.
%% (wg-bibitem)

%%\clearpage
\addcontentsline{References}{section}{References}
\renewcommand*{\refname}{Bibliography and References}
\bibliographystyle{apalike}	% bibliographystyle{apalike} and \usepackage{natbib}
\bibliography{FollowMe}	% expects file "MasterBib.bib"



%%\begin{thebibliography}{80}
%%\usepackage{natbib}   %% bibitems
%%\end{thebibliography}

%%\clearpage
\addcontentsline{toc}{section}{Index}
\printindex %% www.cs.usask.ca/resources/tutorials/latex/notes/toc-index.pdf

% /home/git/external/FlexSpec1/Code/rpi/tex/FollowMe.tex

%% (wg-texdoc-endnotes)

%%%%%%%%%%%%%%%%%%%%%%%%%%%%%%%%%%%%%%%%%%%%%%%%%%%%%%%%%%%%%%%%%%%%%%%%%%%%%
% Support for endnotes
\begingroup
\renewcommand{\notesname}{\textcolor{red} {Action Items:}}
\parindent 0pt
\parskip 2ex
\phantomsection
\addcontentsline{toc}{section}{	extcolor{red} {Action Items:}}
\def\enotesize{\normalsize}
\theendnotes
\endgroup

\end{document}

Ksk Royal -- basic install for Ubuntu 22.04
https://www.youtube.com/watch?v=YRgNcBdUHbo


