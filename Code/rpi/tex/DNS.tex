\subsection{DNS Details}

This is where it gets sticky.

access-control-list

trust-anchors in bind.keys

Block these zones from routing out.

\begingroup \fontsize{10pt}{10pt}
\selectfont
%%\begin{Verbatim} [commandchars=\\\{\}]
\begin{verbatim} 
/etc/bind:
    bind.keys  -- trust anchors.
    db.0       -- keep these routes very local
    db.127
    db.255
    
    db.empty   -- only on the machine
    db.local
    named.conf - simply include the named.conf.xxxx files
       named.conf.default-zones
       named.conf.local
       named.conf.options
    rndc.key      -secret key chmod c640 chown bind:bind
    zones.rfc1918 - the list of socalled non-routable IP domains

    db-{site}-local
    db-{site}-reverse
\end{verbatim}
\endgroup
%% \end{Verbatim}

Files we need to configure:




Different machines call the ethernet interface by different
names. Usually OS vendor dependent. You will find \dhl{eno1} and
\dhl{eth0}. Currently Ubuntu 22.04 installs with \dhl{eth0}.

This document is on a Comcast Business SOHO network, so I've left the
IP addresses where they are. Comcast uses the 10 net. Not the best
idea.  Most homes use the 192.168/13 net. The /13 refers to \dhl{CIDR
  (Classless Inter-Domain Routing)
  notation}\footnote{\url{https://en.wikipedia.org/wiki/Classless_Inter-Domain_Routing}}.

Scenario: The home router connects the WAN (telco) to a LAN (local area network).
Most home accounts do not have a fixed IP address, they are assigned from a pool
of static IP addresses assigned by ICANN to the telco. Some use an internal
network address that randomly ties to a pool of external (WAN) addresses -- making
penetrating a home network from the outside difficult. Telco's 'rent' these blocks
to you for a fee. You can hit external websites that will bounce back your IP
address -- and you can 'borrow' that for a while.
When Google was asked ``whats my ip address'' it gets
back \dhl{2603:300b:509:500:dc35:2027:1f78:a426} -- this is an IP6 address.
Uuf da. The IP4 address is 71.237.89.11. Lets see how that fares over the day.

Thus we have to setup DNS with both addresses.




FlexSpec uses the nginx engine with an external ``Certificate of Authority''
(obtained from a CA provider). The nginx web server challenges the CA with
its credentials and is able to establish a SSH connection with the remote
caller. All login in steps are handled by nginx at the final address.




In the FlexSpec1 scenario, there are one or many  pi's per OTA/Mount; one or many 
mounts per observatory; one or many observatories per site and one or more sites
per collaboration; and one or more collaborations per federation. Or just a observer
in his warm room.

To facilitate all of this we assume the Site is called
example.com\footnote{Remember, this one does not route and we use it,
  well, as an example.}. The observatory, obs1 is a \dhl{subnet} of
example.com and may be reached at the \dhl{obs1.example.com} domain
name. In this the person far away just enters that name, example.com
is undergoes a DNS lookup within the real DNS system, and a query is
made of example.com to see if it can resolve a name, and if it can
resolve the \dhl{obs1.example.com} name.  If, so packets are sent
to the main machine, which in turn passes the packets on to the
obs1 router.

Lets have pier15 with two OTAs, one with a flexspec and the other
performing \dhl{piggyback} duty. let there be two Pi's:  myflex and mypiggy
machines there. Now the domain name will be
mypiggy.pier15.example.com.

Each dot (.) is a router! Each one needs to know the 'names' of the
individual machines under the router's control.

This is were \dhl{bind9} comes into play.

The FlexSpec1/Code/ directory has a script for each Pi to manage itself,
and help with each router as we can.

\subsection{ARPA}

This is a ``backronym'', where the Advanced Research Project Agency (ARPA)
originated ARPANET the early precursor-model for the modern Internet, it
now simply means \dhl{Address and Routing Parameter Area}, where top level
machines manage/delegate the addressing management. The critical idea
of this is the single, and unmentioned ``.'' that is the ``top of the Internet''.
This lone dot appears at the very end of every named address, for example
\dhl{example.com.} (note the dot after com). This is critical, as it must
append in bind (other) DNS management files.

\begingroup \fontsize{10pt}{10pt}
\selectfont
\begin{verbatim} 


lsb_release -a  # get a short description of the machine 'properties'
ip r
# default via 10.1.10.1 dev eno1 proto dhcp metric 100 
# 10.1.10.0/24 dev eno1 proto kernel scope link src 10.1.10.232 metric 100 
# 169.254.0.0/16 dev docker0 scope link metric 1000 linkdown 
# 172.17.0.0/16 dev docker0 proto kernel scope link src 172.17.0.1 linkdown 


echo $(ip r | gawk -- '/kernel/ {printf("%s", $1);}')
export mylocalnet=$(ip r | gawk -- '/kernel/ {printf("%s", $1);}')

forward/reverse lookup zones
SOA and A record forward
SOA and A reverse 

Edit /etc/named.conf
forwarders {
  192.168.1.254;   # private (us)
  8.8.8.8;         # public here google
};

sudo systemctl restart bind9
sudo systemctl status bind9


edit named.conf:

zone "class.local" {
  type master;
  file "/etc/bind/db.class.local";
};

sudo cp db.local db.class.local  # our domain local

edit db.class.local

%%%%%%%%%%%%%%%%%%%%%%%%%%%%%%%%%%%%%%%%%%%%%%%%%%%%%%%%%%%%%%%%%%%%%%%%%%%%%
$TTL   604800
@          IN       SOA     class.local. root.class.local. (
                    2          ; serial  
                     604800    ; refresh
                     86400     ; Retry
                     2419200   ; Expire
                     604800 )  ; Netative Cache TTL
;
@    IN           NS     flex.class.local.
@    IN           A      10.1.10.70
@    IN           AAAA   ::1
flex              10.1.10.70
%%%%%%%%%%%%%%%%%%%%%%%%%%%%%%%%%%%%%%%%%%%%%%%%%%%%%%%%%%%%%%%%%%%%%%%%%%%%%
sudo systemctl restart bind9
sudo systemctl status bind9    # check for typos/syntax errors

edit named.conf.local insert lookup zone

//  add to end
// reverse lookup zone
zone "1.254.10.in-addr.arpa"  {
    type       master;
    file       "/etc/bind.db.10";
};

sudo systemctl restart bind9
sudo systemctl status bind9    # check for typos/syntax errors

edit db.10 file



; BIND reverse data file for local 10.01.10.xx network

$TTL   604800
@          IN       SOA     flex.class.local. root.class.local. (
                    2          ; serial  
                     604800    ; refresh
                     86400     ; Retry
                     2419200   ; Expire
                     604800 )  ; Netative Cache TTL
;
@    IN           NS     flex.
10   IN           PTR    flex.class.local

sudo systemctl restart bind9
sudo systemctl status bind9    # check for typos/syntax errors

edit resolv.conf and add log






nameserver 10.01.10.70   
options eth0 trust-ad
search  class.local


sudo systemctl restart bind9
sudo systemctl status bind9    # check for typos/syntax errors


ping 10.01.10.70
dig -x                 # look at us
dig du.edu             # lool at something far away
ping pier15.class.local

\end{verbatim}
\endgroup
%% \end{Verbatim}


\subsection{Notes}



\begingroup \fontsize{10pt}{10pt}
\selectfont
%%\begin{Verbatim} [commandchars=\\\{\}]
\begin{verbatim} 

ip r | gawk --  '/proto/ {print $1;}'

lsb_release -a
No LSB modules are available.
Distributor ID:	Ubuntu
Description:	Ubuntu 22.04.1 LTS
Release:	22.04
Codename:	jammy
\end{verbatim}
\endgroup
%% \end{Verbatim}

\begingroup \fontsize{10pt}{10pt}
\selectfont
%%\begin{Verbatim} [commandchars=\\\{\}]
\begin{verbatim} 
ip r
default via 10.1.10.1 dev eth0 proto dhcp metric 100 
10.1.10.0/24 dev eth0 proto kernel scope link src 10.1.10.70 metric 100 
169.254.0.0/16 dev eth0 scope link metric 1000 
\end{verbatim}
\endgroup
%% \end{Verbatim}



\url{https://www.youtube.com/watch?v=MPDXVkwUehs}


\subsection{DNS Background}\section{Domain Name Server}

Up-front things:

Certain classes of internet addresses will not route.\footnote{\url{https://en.wikipedia.org/wiki/IP_address}}.

\begingroup \fontsize{8pt}{8pt}
\selectfont
%%\begin{Verbatim} [commandchars=\\\{\}]
\begin{verbatim} 
10./24         ( Range: 10.0.0.0    : 10.255.255.255 )  instrument net
169.254/16     ( Range: 169.254.0.0 : 169.254.255.255)  private network
172.16.0.0/12  ( Range: 172.16.0.0  : 172.31.255.255 )  lan routable
192.168.0.0/16 ( Range: 192.168.0.0 : 192.168.255.255)  lan routable
\end{verbatim}
\endgroup
%% \end{Verbatim}

The ``Domain Name System''. originating in 1984 \cite{BIND_Manual1},
developed under DARPA funds at Berkeley as \dhl{Berkeley Internet Name
  Domain (BIND)}, serves as ``the'' top level domain name manager. The
ICANN\footnote{Internet Corporation for Assigned Names and Numbers}
DNS System uses a large number of secondary caches to resolve the
human readable \dhl{domain name} into a
network-routable\footnote{Routable protocol allows packets to be
sa  forwarded from one network to another} (hardware) numeric
\dhl{Internet Protocol Address}.  The addresses are either IP4, and
lately IP6. Lookups worldwide take on the order of 10s to 100s of
milliseconds -- event down the a public-facing IP address running on a
FlexSpec telescope.

To serve such a wide number of domain names, (est. 341.7 million) in 2022
it is very obvious that the ``Domain'' needs to be divided into ``zones'' each
divided into sub-zones. An example:  \dhl{pier.observatory.example.com} can
be used to reference a sub-net for an observatory at <example.com>; that has 
smaller subdomain called <pier>. The main router at example.com manages ``zones
within it''. Here, perhaps one of several buildings -- or ``observatory''
in this scenario. The router within each ``observatory'' (or the main
site's router) has further zones -- here <pier> in this scenario. World
governing infrastructure manages these ``Top Level Domains'' (TLDs). 
The traffic into the reference machine at the end of that IP address,
has a (extended) nameserver associated with it. The site administrators maintain
that extended nameserver. It is presented with the fill name, parses that
name, resolves what it can. That resolution may result in a further 
request of a deeper-nested nameserver for more information. Eventually
the local machine will return a name, perhaps with a different port,
whereby a running program at the other end of this mess is engaged.
